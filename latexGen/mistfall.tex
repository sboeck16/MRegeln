\documentclass{article}
\usepackage[utf8]{inputenc}
\usepackage[T1]{fontenc}
\usepackage[ngerman]{babel}
\usepackage[a4paper, left=2cm, bottom=15mm, top=2cm, right=2cm]{geometry}
\usepackage{hyperref}
\usepackage{multicol}
\usepackage{xcolor}
\usepackage[framemethod=tikz]{mdframed}
\usepackage{array}
\begin{document}
\tableofcontents
\newpage
\begin{center}
\section{Einleitung und Überblick}
\end{center}

Klassische Pen and Paper Role Playing Games (oder wie auch immer sie genannt und abgekürzt werden) haben das Ziel
eine Geschichte um eine Gruppe von Charakteren zu erzählen. In der Gruppe der Spielenden übernimmt einer die Rolle
des Geschichtenerzählers (Spielleiter) die anderen Spieler steuern ihren Spielercharakter in der Erzählung. Das
Bewerten der Handlung und Aktionen der Spieler obliegt dem Spielleiter, er muss daraus die Geschichte weitererzählen.
Zum Thema Gestaltung des Rollenspiels gibt es vielfältige Literatur und Ratgeber. Kernzweck ist hier aber
ein Regelwerk bereit zu stellen auf das sich Spieler und Spielleiter verlassen können um erzählerische Konflikte zu
lösen (schafft der Charakter den Sprung über den Graben?). Als Regelwerk für klassisches Rollenspiel sind die Regeln
für Kampf und Handlungsmöglichkeiten der Charaktere dort detaillierter ausgearbeitet.

\begin{center}
\subsection{Für Wen ist das Regelwerk?}
\end{center}

Es richtet sich an erfahrene Spielgruppen und ist für Einsteiger weniger geeignet. Das vorliegende Regelwerk ist aus
verschiedenen Modulen zusammengesetzt (und bezieht sich auf ein Setting). Kernelemente sind ein Attribut und
Fertigkeitssystem um möglichst alle Situationen oder Konflikte zu bewerten. Trotzdem sollte es ruhig um
Hausregeln ergänzt werden, bzw geändert werden. Im Regelwerk gibt es Spielleiterhinweise die dazu gedacht sind den
Spielleiter bzw. die Spielrunde anzustoßen ob die vorgestellte Regel (oder alternative Regel) so gespielt werden soll.
Hier wäre gleich so eine Empfehlung: Es empfiehlt sich die Hausregeln vor der Kampagne zu besprechen (damit alle auf
dem gleichen Stand sind). Sollte es im Spielverlauf dazu kommen das eine Regel nicht passt so sollte die Szene oder
dasAbenteuer zu Ende gespielt werden und erst dann diskutiert werden.

\begin{center}
\subsection{Spielleiter hat immer Recht!}
\end{center}

Falls ein Spieler das lesen muss... Aber im Ernst, Ziel ist das alle Spass haben, die Aufgabe des Spielleiters ist
dabei aber komplizierter. Er muss Handlungs und Spannungsbögen planen können deswegen hat er immer Recht. Er sollte
aber zu seinen Entscheidungen stehen (konsequent) und wenn möglich erklären. Diskussionen sollten wenn sie länger
dauern nicht im Abenteuer stattfinden.

\begin{center}
\section{Würfelproben}
\end{center}

Früher oder später kommt man in der Geschichte an eine Stelle an der nicht klar ist ob ein Charakter die vor ihm
liegende Aufgabe bewältigen kann. Wir haben einen erzählerischen Konflikt, falls der Charakter die Aufgaben schafft
so nimmt die Geschichte einen anderen Verlauf als wenn er sie nicht schafft. Dieser Konflikt wird durch das Regelwerk
gelöst. Der Charakter (und möglicherweise weiter Beteiligte) haben Stärken und Schwächen sowie antrainierte
Fertigkeiten. All das muss auf die Situation angewand werden um zu bestimmen ob der Charakter Erfolg hat oder nicht.
Da Rollenspielregelwerke nur Abstraktionen eines realistischen (oder auch eines fantastischen, etc.) Vorgangs sind,
bedienen sich die meisten Regelwerke (so auch dieses) dem Zufall um zu ermitteln ob der Charakter den Konflikt für
sich entscheidet.

\paragraph{Beispiel}

Vex hatte mal wieder Pech, mit knurrenden Magen ins erstbeste Herrenhaus einzubrechen war prinzipiell nicht die
dämlichste Idee. Nur das der ''Herr'' ein Hundenarr ist und Vex durchs Fenster im ''Hundesalon'' landete.
Der Spielleiter hatte ein Glücksprobe gefordert (Vex Spielerin hätte diese vielleicht umgehen können wenn sie sich
über das Haus informiert hätte), diese viel sehr schlecht aus... Jetzt wird der Spielleiter eine Heimlichkeitsprobe
fordern. Schliesslich soll man schlafende Hunde nicht wecken.

\begin{center}
\subsection{Wie und Was würfelt man?}
\end{center}

Wenn Charaktere vor Aufgaben stehen die sie mit ihren Fähigkeiten bewältigen müssen so wird das weiter unten stehende
System über Würfelpools verwendet. Aber für Neulinge eine kurze Erinnerung für allgemeine (additive) Würfelwürfe in
Rollenspielsystemen.

\subsubsection{Allgemeine und sonstige Würfelproben}

In Rollenspielregelwerken hat es sich durchgesetzt Würfelproben mit XWY+C zu beschreiben. Das bedeutet es wird eine
Anzahl von 'X' Würfeln (im englischen Dice dort wird mit XDY+C abgekürzt) mit 'Y' Seiten geworfen. Das Ergebniss der
einzelnen Würfe wird zusammen mit C addiert und bilden das Ergebniss.

Die W4, W6, W8, W10, W12 und W20 (Würfel mit entsprechender Seitenzahl) sind erwerbbar und sollten vorrätig sein.
Da in dem System Würfelpools aus W6 gebildet werden empfiehlt es sich diese in größere Zahl vorrätig zu haben
(10 pro Spieler und Spielleiter sind nicht schlecht, unterschiedliche Farben können hilfreich sein).

\paragraph{Beispiel}

Ein Spieler erwirbt einen Heiltrank und notiert das dieser 3W4+5 Schadenspunkte neutralisiert. Als sein Charakter
diesen nutzt so würfelt er drei vierseitige Würfel, diese zeigen das Ergebnis 2, 4 und 1 an. Zusammen mit der ''5''
ist das Ergebnis also 12.

Um andere Würfel zu improvisieren kann man folgende Methoden verwenden (vor dem Wurf festlegen!):

\begin{itemize}
\item W100 oder auch W\% Das Ergebnis wird durch zwei W10 gebildet (für diesen Zweck gilt die 10 als 0). Jeder W10 legt eine Stelle des Wurfes fest (unterschiedliche Farben, bzw es gibt passende Würfel die 10er Ergebnisse abgebildet haben). Eine doppelte Null bezeichnet dann die 100. Durch weitere W10 kann man auch W1000 oder größer darstellen)
\item W2, W3, W5 lassen sich durch halbieren und aufrunden eines Würfels doppelter Augenzahl abbilden. (Beispiel W3 durch einen W6: 1-2 ist eine 1, 3-4 ist eine 2 und 5-6 ist eine 3). Auch W4 und W10 lassen sich zur Not so würfeln. Prinzipiell kann auch ein W4 durch einen W20 abgebildet werden (durch 5 teilen und aufrunden)
\item Beliebiger anderer Würfel der größer ist. Sollte eine Zahl gewürfelt werden die größer ist als der simulierte Würfel kann so wird neu geworfen. (Einen W31 könnte man mit einem W100 würfeln -> durch 3 teilen und aufrunden, ist das Ergebnis dann eine 32, 33 oder 34 so wird erneut gewürfelt)
\end{itemize}

\paragraph{Spielleiteroption allgemeine Glücksprobe}

Auch wenn dieses Regelwerk ein Glücksattribut bereit hält gibt es viele Situation die zu allgemein sind um sie
den speziellen Glück eines Charakters zu überlassen. Insbesondere da das Attribut sonst zu übermächtig und zu sehr
erforderlich ist). Hier kann man 2W6 würfeln. Je höher das Ergebnis desto schlechter für die Spieler oder für die
Partei für die gewürfelt wurde. Zwei Würfel bieten den Vorteil das die extremen Ergebnisse (2 und 12) selten und
durchschnittliche Ergebnisse (6 bis 8) häufig vorkommen. Manchmal kann der Wurf als Inspiration für das weitere
Erzählen dienen, allerdings empfiehlt sich vorher grob festzulegen was bei welchem Ergebnis passieren wird.

\paragraph{Beispiel}

Bob hat den Befehl über ein Regiment königlicher Bogenschützen bekommen. Durch erfolgreiche Ausspähaktionen hat er
eine äusserst günstige Position für den Feuerüberfall auf die feindlichen Nachschublinien ermittelt. Durch gutes
Timing und Vorbereitung sind die Bogenschützen unendeckt in Position gegangen, allerdings wird der feindliche
Nachschub durch eine Lanze leichter Kavallerie und weiterer Kämpfer begleitet. Da die Entlastung die dringend
benötigt wird und die Position erfolgsversprechend ist entschliesst sich Bob schweren Herzens dazu den Angriff zu
befehlen.
Bobs Spieler und der Spielleiter wollen den Kampf aber nicht im Detail spielen und einigen sich auf einen Glückswurf
der ermitteln soll welchen Ausgang die Schlacht nimmt. Hier werden 2W6 geworfen. Je niedriger desto besser für den
Spielercharakter (hier Bobs Bogenschützen). Der Spielleiter bewertet Bobs Vorbereitungen als sehr gut und gib einen
Bonus von 2. Es wird allso 2W6-2 gewürfelt. Bobs Spieler würfelt zwei sechseitige Würfel und würfelt eine 4 und
eine 1. Alles zusammen ist das Ergebniss eine 3 (und somit sehr glücklich/erfolgreich). Der Spielleiter beschreibt
dann den Schlachtverlauf anhand diesem Ergebnis. Die Kavallerie schafft es nicht geordnet den Hügel auf dem die
Schützen stehen zu stürmen, diese reagieren schnell und decken die Pferde mit einem Geschoßhagel ein das diese
liegen bleiben und die existierende Unordnung der nachfolgenden Reiter nur vergrößern. Auch die restlichen Kämpfer
bleiben desorganisiert und entschliessen sich rasch zur Flucht unter Zurücklassung der Wagen.

\begin{center}
\subsection{Würfelpool}
\end{center}

Für Proben zur Konfliktlösung werden Würfelpools von sechsseitigen Würfeln gebildet (W6). Diese werden im Regelwerk
meist nur mit einem großen W bezeichnet. Als Würfelpool bezeichnen wir hier eine gewisse Anzahl an Würfeln. Diese
Anzahl setzt sich aus verschiedenen Faktoren zusammen die nachfolgend aufgelistet werden.

Hat man die Größe des Pools bestimmt so wird dieser geworfen (sprich man würfelt alle im Pool befindlichen Würfel).

Man legt die 4 besten Würfel raus und addiert diese. Das ist das Ergebnis.
Ist unter den nicht gewerteten Würfeln, Würfel mit dem Ergebnis 6 so erhöhen diese den Wurf um eins.

\subsubsection{Grundlage eines Würfelpools}

Prinzipiell werden immer zwei Charakterwerte addiert um die Größe des Würfelpool zu bilden!

\paragraph{Attribut und Fertigkeit}

Für die meisten Proben werden Attribut und Fertigkeitswert addiert. Je nach Aufgabe wird ein Attribut und eine
Fertigkeit gewählt und bilden die Grundlage des Pools. Für eine spezielle Aufgabe wird immer dasselbe Attribut
mit derselben Fertigkeit kombiniert. Allerdings kann eine Fertigkeit durchaus mit verschiedenen Attributen kombiniert
werden, je nach Situation. Z.B. beim Angreifen mit Pfeil und Bogen werden das Attribut Wahrnehmung und die Fertigkeit
Bogenschießen addiert. Beim Schätzen des Wertes eines frisch gefunden Bogens könnte aber auch z.B. Intelligenz und
Bogen addiert werden. Oder beim Wettschiessen bei der es nur um die Entfernung geht könnte der Spielleiter Stärke und
Bogenschießen verlangen.

\paragraph{zwei Attribute}

Ein paar grundlegende Würfe werden häufig durch die Addition zweier Attribute beschrieben. Zwei passende Attribute
werden ausgewählt und bilden den Würfelpool. Meist sind diese Kombinationen sehr eindeutig und die Größe des Pools
kann ruhig auf dem Charakterblatt festgehalten werden. So werden Reaktion und Initiativeproben auf Wahrnehmung und
Geschick gewürfelt.

\paragraph{doppeltes Attribut}

Ganz allgemeine Konflikte lassen sich lösen in dem das zu Grunde liegende Attribut zu Rate gezogen wird. Bei einer
Attributsprobe wird das gewählte Attribut verdoppelt (Also Attribut und Attribut um die Regel der zwei Werte gerecht
zu werden) und bildet die Grundlage des Pools. Der Charakter könnte stolpern und hält sich nur bei einer erfolgreichen
Geschicklichkeitsprobe auf den Beinen. Oder der Charakter versucht einer Versuchung zu widerstehen (Willenskraftprobe).

\subsubsection{Modifikation des Würfelpools}

Hat man die Grundlage des Pool gebildet so kann dieser modifiziert werden. Das bedeutet das Würfel hinzugefügt oder
abgezogen werden können.

Modifikation des Pools werden durch Situationen und Gegebenheiten die zentriert um den ausführenden Charakter sind.
Insbesondere Kontrolle (oder der Verlust dieser) der Situation oder Vorbereitung und Werkzeuge, etc. Alles was
prinzipiell vom Charakter beeinflußt werden könnte (nicht unbedingt zum Zeitpunkt der Probe). Umstände zählen dazu
schliesslich entscheidet sich ein Charakter dazu gerade hier an diesem Ort die Probe abzulegen (Auch wenn die
Geschichte logischerweise nur an diesem Ort die Probe verlangt).

\paragraph{Erleichterungen}

Hier ein paar Beispiele die die Aufgabe erleichtern:

\begin{itemize}
\item Langsames und gründliches Arbeiten: Der Zeitraum wird verdoppelt, die Probe bekommt 2W
\item Hilfe anderer Charaktere: Halbe abgerundete Fertigkeit gibt es als Bonuswürfel. (Hier sollte man realistisch bleiben 10 Diebe werden nicht gemeinsam ein Schloss knacken. Wenn aber drei Charaktere Erste Hilfe bei einem verblutendenden Charakter leisten so kann man durchaus akzeptieren das zwei assistieren und einer die eigentliche Arbeit macht). Als Faustregel sollte gelten das der ausführende Charakter nicht mehr Bonuswürfel durch assistierende Charaktere bekommt als er selbst Fertigkeit in der Aufgabe hat (Spielleiter hat das letzte Wort).
\item Vorbereitung und Umstände: Der Charakter soll auf eine Wissensfertigkeit würfeln und hat passende Nachschlagewerke oder gar eine Bibliothek? Hier kann ein Bonus von 1W bis 3W gewährt werden. Er weiß was ihn erwartet und hat Gelegenheit sich sinnvoll vorzubereiten? Ein Bonus von 1W bis 2W ist gerechtfertigt. Meisterhafte Werkzeuge oder solche die ungewöhnlich gut für die Aufgabe sind können ebenfalls 1W oder 2W Bonus gewähren. Solche Werkzeuge sollten einen generischer Bonus geben der auch in anderen Situation zum tragen kommen kann). Umstände können additiv sein sofern sie sich nicht ersetzen (So wird der Wissbegierige nicht 1W fürs Nachschlagewerk bekommen wenn er gleichzeitig 3W für die umfangreiche Bibliothek bekommt. Der Dieb der sich auf das zu knackende Schloss vorbereitet hat und Meisterwerkzeuge dabei hat bekommt aber für beide Umstände je einen Bonuswürfel).
\end{itemize}

\paragraph{Erschwernisse}

Hier ein paar Beispiele die die Aufgabe erschweren:

\begin{itemize}
\item Schnelles und hastiges Arbeiten: Der Zeitraum wird halbieren, die Probe erleidet einem Malus von 3W.
\item Beinträchtigung des Charakters: Der Charakter ist verwundet oder unter dem Einfluß eines Giftes oder einer Krankheit. Der Malus wird an den entsprechenden Stellen erklärt und kann gravierend sein. Zu Beachten ist das er nicht für jede Probe angewanded oder zumindest nicht in voller Höhe angewandedt werden sollte. Der Sänger ist leicht betrunken? Er würde beim Klettern sicherlich einen Malus bekommen aber beim Singen oder Geschichten erzählen sollte ihn das erstmal nicht stören. Der Charakter hat eine Verwundung, beeinträchtigt das seine Fähigkeit sich an Sachen zu erinnern? Es sollte in der Spielrunde geklärt und festgehalten werden ob wie die einzelnen körperlichen Mali angewandt werden. Als Faustregel gilt körperlich aktive Proben sind immer in voller Höhe betroffen. Wahrnehmungs, Wissens und soziale Proben nur eingeschränkt wenn überhaupt. Die schwammige Formulierung ist notwendig, genauer wirds beim Abschnitt Verwundung und Heilung.
\item Umstände: Alles was den Erfolg des Charakters schmälert. Z.B. ein schaukelndes Schiff beim Holzschnitzen. Der Malus wird vom SL festgelegt und ist von 1W (leichtes Ärgernis z.B. leisten von Erster Hilfe im Regen) bis 3W (sehr schlechte Umstände wie leisten von erster Hilfe im Sturm auf einem tanzenden Boot) oder mehr. Der Malus aus unterschiedlichen Quellen ist additiv! (Also beim Verarzten auf einem Boot im Sturm bei strömenden Regen währen das 4W). Auch fehlende Vorbereitung oder besser formuliert fehlende unzureichende Vorbereitung kann die Probe erschweren oder unmöglich machen. Die Mali hier hängen von der Aufgabe ab und werden entweder in der Fertigkeitsbeschreibung oder vom Spielleiter festgelegt (z.B. ohne zu Lernen in die Prüfung gehen).
\end{itemize}

\paragraph{Sonstige Modifikation, Charakterboni}

Der Pool wird häufig durch spezielle Charakterboni oder Situationsregeln modifiziert. (Wege, Spezialisierungen oder
sonstige Vorteile durch Geburt oder Beruf, etc.). Diese Modifikation werden an der jeweiligen Stelle im Regelwerk
erklärt.


\paragraph{Beispiel}

Schauen wir uns wieder die hungrige Vex an. Mit einem Arm voll Hundefutter und einer sehr(!) wütenden Hundemeute auf
den Fersen läuft sie zurück zur Umfriedung des Herrenhaus (Vex hat die Heimlichkeitsprobe geschafft, aber Hunde haben
ein überirdisches Gespür wenn es um ihr Futter geht, leider ist Vex wirklich hungrig). Diese muss überklettert werden.
Der Spielleiter legt die Schwierigkeit auf 16 fest. Als Probe soll Geschick und Klettern gewürfelt werden. Als 
Meisterdiebin hat Vex Geschicklichkeit auf 4 und Klettern auf 5. Der Würfelpool ist also 9, allerdings verhängt der
Spielleiter zwei Mali, einmal hat Vex nur einen Arm zum Klettern und bekommt einen Malus von 2W zum anderen muss es
jetzt schnell gehen, also nochmal einen Malus von 2W. Die Spielerin von Vex erinnert daran das Vex ein gut
verstecktes Seil zurück gelassen hat das ihr bei der Flucht behilflich sein kann und bekommt einen Bonus von 1W.
Alles zusammen werden jetzt 6W geworfen. Diese zeigen 6,4,1,2,5,5. Die vier besten Werte (6,5,5,4) geben als Ergebnis
eine 20. Vex entkommt also sehr gekonnt der Hundemeute.
Ein weniger geschickter Dieb hätte an der Stelle die Beute fallen lassen können um den Malus zu umgehen. Auch das
fallen lassen der Würde (langsames Klettern und dafür einen Hundebiß im Allerwertesten zu riskieren) wäre möglich
gewesen um den anderen Malus zu neutralisieren.

Allerdings ist Vex damit nicht in Sicherheit, wie der Zufall es will landet sie genau der Stadtwache vor den Füßen
diese sind verdutzt und Vex nutzt die Gelegenheit sofort auf sie einzureden. Erklärt ihnen sie sei für die Qualität
der Hundefutterherstellung verantwortlich und prüft nun eben wie gut Hunde darauf ansprechen. Schnell fügt
sie mit schelmischen Lächeln hinzu das sie wohl besonders Erlesenes gerade prüft. (Unnötig zu erwähnen das die Hunde
auf der anderen Seite der Umfriedung furchtbaren Lärm machen). Der Spielleiter ist sich nicht sicher um seine tumben
Wachen das glauben sollen und lässt die Spielerin Persönlichkeit und Überzeugen würfeln, legt aber die Schwierigkeit
auf 22 fest da die Geschichte äußerst unglaubwürdig ist. Der Spielerin ist das zu hoch (der Pool ist nur 7 Würfel
groß) und läßt Vex schnell noch die ungläubig guckenden Wachen fragen, was sie wohl sonst hier macht? Einen Einbruch
um Hundefutter zu stellen, das glaubt auf Wache doch niemand? Der Spielleiter ist gnädig und senkt den Mindestwurf
auf 20. Die Wachen sind nun vollends verwirrt und wissen garnicht mehr was sie glauben sollen. Bei einer 20 glauben
sie zwar nicht die Geschichte von Vex werden sie aber gehen lassen.
Die 7W (Persönlichkeit 4, Überzeugen 3 keine weiteren Modifikationen) werden geworfen, das Ergebnis 6,2,6,6,6,6,6.
Ein absolut überragender Wurf. Das Ergebniss von 24 kann durch zwei zusätzliche 6er auf 26 erhöht werden. Die beiden
Wachen sind absolute Hundeliebhaber helfen Vex auf die Füße und bedanken sich mit tränenschwacher Stimme das endlich
mal jemand auch an die Hunde denkt und insbesondere auch ans Hundefutter. Gerührt gehen die Wachen weiter und lassen
eine leicht perplexe Vex zurück...

\subsubsection{Mindestwürfe}

Nachdem geklärt ist wie ein Würfelpool gebildet und der Wurf ausgewertet wird wird in diesem Kapitel wird erklärt wie
sich die Mindestwürfe allgemein zusammensetzen. Für spezielle Situationen gibt es aber spezielle Regeln die die
Mindestwürfe festsetzen (zum Beispiel werden die Kampfmindestwürfe im Kapitel Kampf beschrieben).

Prinzipiel gibt es zwei Arten von Mindestwürfen:

\paragraph{Fester Mindestwurf}

Hier sind feste Mindestwürfe und eine knappe verbale Einschätzung:

\begin{itemize}
\item 12: sehr einfach
\item 14: einfach
\item 16: normal
\item 18: schwer
\item 20: sehr schwer
\item 22: extrem kompliziert
\item 24: eigentlich unmöglich
\end{itemize}

Es wird ein Mindestwurf anhand der Schwierigkeit festgelegt. Das eignet sich für alltäglich und unspezifische
Situationen. Der Vorteil ist das die Schwierigkeit verbal festgelegt werden kann und damit kurz erläutert werden
kann (Der Spieler ist verwirrt das die Kletternprobe so schwer sein soll, der Spielleiter weißt nochmal auf die
glatten Marmorwände hin). Prinzipiell können sich sowohl der Spielleiter als auch die Spieler auf ihr Bauchgefühl
verlassen was die Mindestwürfe anbelangt. Im Zweifel sollte das kommuniziert sein bevor der Spieler sich für einen
Weg entscheidet der die Probe erforderlich macht. Auch Zwischenschritte können verwendet werden (ist keine normale
Aufgabe aber so wirklich schwer ist sie auch noch nicht d.h. Mindestwurf 17)

\paragraph{Vergleichende Probe}

Stehen Charakter als Gegner in einem Konflikt so kann dieser mit einer vergleichenden Probe gelöst werden. Dazu
werden für beide Charaktere passende Würfelpools gebildet und modifiziert. Diese Pools müssen nicht die gleichen
sein und können gravierend abweichen. Z.B. würde Wahrnehmung und Aufmerksamkeit gegen Geschick und Heimlichkeit
gewürfelt beim Konflikt ob der Schleichende entdeckt wird.

Beide Charaktere würfeln die Pools und modifizieren die Ergebnisse durch anzuwendende Regeln (z.B. könnte der Bonus
durch Sichtdeckung beim Schleichen auf das Ergebnis addiert werden).

Der Charakter mit dem höheren Ergebnis entscheidet den Wettstreit. Falls ein Unentschieden nicht auswertbar ist (z.B.
der Taschendieb wird entweder ertappt oder nicht) so sollte in der Regel zu Gunsten des passiven oder defensiven
Charakters entschieden werden. Alternativ kann die Probe solange wiederholt werden bis ein Sieger feststeht (
allerdings gilt die Probe trotzdem nur als knapp geschafft oder verfehlt unabhängig von Ergebnis der folgenden Würfe).

\paragraph{Modifikation}

Die Modifikation die den Mindestwurf beeinflussen, sprich ihn erhöhen oder senken sind vom Ziel abhängig.
Meist ist eine Modifikation nicht nötig bzw. werden sie durch spezifische Regeln abgedeckt. Z.B. gibt es in den
Kampfregeln verschiedene Modifikationen für die Verteidigung (der Mindestwurf der Angriffsprobe). Bei generellen
Proben sollte sich eher an den verbalen Beschreibungen orientiert werden.

\paragraph{Zusätzliche Auswertung}

In vielen Fällen reicht es bei der Konfliktlösung zwischen Erfolg und Misserfolg zu unterscheiden. Manchmal ist
für die Auswertung aber die Güte des Erfolg notwendig (und sei es aus rein erzählerischer Weise). Dazu kann das
Ergebnis der Probe rangezogen werden (eine Ergebnis von 24 bei einer Schwierigkeit von 24 sollte durchaus kurz
beschrieben werden, der Charakter hat was unmögliches versucht und trotzdem irgendwie knapp geschafft). Auch die
Differenz zwischen Ergebnis der Probe und Mindestwurf sagt was über die Güte aus. (Ergebnis von 20 bei Mindestwurf
von 14, beschreibt eine einfache Aufgabe die routiniert und ohne Fehler ausgeführt wurde).

An einigen Stellen wird im Regelwerk werden Erfolgsstufen verwendet (zum Beispiel beim Laufen zur Bestimmung der
Geschwindigkeit). Erfolgsstufen sind die abgerundete halbe Differenz zwischen Ergebnis und Mindestwurf. Sprich eine
Aufgabe ist geschafft wenn der Mindeswurf geschafft ist, die erste Erfolgsstufe gibts aber erst wenn man das Ergebnis
um zwei überbietet). Erfolgsstufen werden explizit in den Regeln oder bei Bedarf vom Spielleiter ausgewertet. Falls
benötigt können auch negative Erfolgsstufen gewertet werden (unebener Boden der Charakter kommt langsamer voran wenn
die Probe um zwei oder mehr daneben geht).

Von einem kritischen Erfolg oder Fehlschlag redet man wenn das Ergebnis um 3 Erfolgsstufen oder mehr abweicht (Ein
Patzer bei einem Mindestwurf von 16 wäre also bei einem Ergebnis von 10 oder weniger). Auch hier reicht es meist aus
das Ergebnis erzählerisch einzuflechten oder über die Erfolgsstufen auszuwerten (Ein kritischer Treffer im Kampf
macht eben 3 Punkte mehr Schaden). Allerdings können Regeloption oder Hausregeln durchaus kritische Erfolge oder
Misserfolge auswerten.


\paragraph{Beispiel}

Der Kämpfer Osric sieht sich auf einer einsamen staubigen Landstraße einer Gruppe von drei Banditen gegenüber.
Erfreut das seine Langweile auf der Reise unterbrochen wird zieht er seinen geliebten Zweihänder und stellt sich den
Banditen. Schnell wird ihm klar das er vor sich nur halbverhungerte Bauern hat die ihm keine Herausforderung bieten.
Aus Mitleid entschliesst er sich sie lieber mit Drohungen zu verjagen und schildert ihn bildlich was er mit ihren
Eingeweiden anstellen wird. Der Spielleiter entschliesst das hier Stärke oder Persönlichkeit nach Wahl des Spielers
zusammen mit Einschüchtern gewürfelt wird. Für den Anführer wird er mit Willenskraft dagegen würfeln. Da die Banditen
in Überzahl sind und halb wahnsinnig entschliesst er sich ihnen zwei Boni zu geben. Als erstes erhöht er den Pool des
Anführers um je 1W durch die beiden Begleiter (Umständebonus). Das Würfelergebnis wird er um 2 erhöhen da die
Banditen extrem verzweifelt sind (Der Spieler initiert die Probe gegen das Ziel Banditen und die haben hier extreme
Umstände). Allerdings legt er fest das die Banditen auch bei einem normalen Mißerfolg nicht angreifen sondern
ihrerseits nur Forderungen stellen werden. Auf weitere Modifikation (z.B. bessere Bewaffnung Osrics, seine anfängliche
nicht gespielte Freude) wird verzichtet. Die Banditen erziehlen bei ihrer Probe eine stolze 13 die um 2 erhöht zur 15
wird. Osric seinerseits hat als Ergebnis des Einschüchterns eine 21 anzubieten, ein kritischer Erfolg! Während die
Möchtegernbanditen einnäßen nehmen sie die Beine in die Hand und fliehen. Währe Osrics Ergebnis eine 9 (kritischer
Fehlschlag) so wären sie wohl zum Angriff übergegangen. (Hunde die bellen und so...). Bei einem normalen Erfolg hätten
sie sich nur normal entfernt bei einem Misserfolg nur eine bescheidene Maut verlangt.

\begin{center}
\section{Kampf}
\end{center}

In diesem Kapitel wird beschrieben wie Charaktere mit physischen Auseinandersetzungen und Verletzungen umgehen. Auch
enthalten ist wie Magie die physischen bzw körperlichen Schaden verursacht, verhindert oder heilt zu bewerten ist.

\begin{center}
\subsection{Zustandsmonitor}
\end{center}

Um die Verletzung die ein Charakter erleidet und aushalten kann zu notieren gibt es zwei Zustandsmonitore. Einen für
den körperlichen Schaden (Physische oder Verletzungen des Körpers) und einen für den geistigen Schaden (Mentale oder
Verletzung des Geistes z.B. durch Magieanwendung). Zum Bestimmen des Aussehens der Monitore gelten folgende Regeln

\begin{itemize}
\item Jeder ist 5 Reihen groß
\item Der körperlich Monitor hat pro Reihe ein Anzahl von Kästchen die 5 + Kon entspricht.
\item Der Geistige hat pro Reihe 5 + Will Kästchen.
\end{itemize}

\subsubsection{Schaden und Mali}

\begin{itemize}
\item Wenn ein Charakter Verletzungen erleidet so notiert er diese im entsprechenden Monitor. Dabei gibt es Strichschaden (einfache schnell heilende Verletzung), Kreuzschaden (normaler Schaden, Verletzungen) und Sternschaden (Verkrüppelungen die wenn überhaupt nur sehr langsam heilen).
\item Der Monitor wird wie beim Lesen abgekreuzt von oben links Reihe für Reihe abgekreuzt. Dabei ist so abzukreuzen das der Sternschaden so weit es geht vorne steht, danach kommt der Kreuz- und zuletzt der Strichschaden. Falls bereits Schaden auf dem Monitor ist so kann ein Strichschaden beim Erhalt von Kreuz oder Sternschaden zu selbigen gemacht werden und am Ende des erlittenen Schadens wieder hinzugefügt werden. Ähnlich wird mit Kreuz und Sternschaden verfahren.
\item Nach Erhalt von Schaden muss auf einem Monitor immer erst der Stern-, dann der Kreuz- und dann der Strichschaden kommen
\item Jede Reihe hat einen Modifikator (0 für die Erste, 1 für die Zweite, usw.).
\item Dieser Modifikator kommt als Malus als Würfelabzug auf alle aktive Proben, sobald ein ausgefülltes Schadenskästchen in der entsprechenden Reihe ist.
\item Sollte ein Charakter mehr Schaden haben als auf seinen Monitor passt so ist er bewusstlos.
\item Schaden der nicht auf den Monitor passt wird zu Schaden einer höheren Kategorie.
\item Ein Charakter stirbt sobald ein Monitor voll mit Sternschaden ist.
\end{itemize}

\paragraph{Beispiel}

Bob der Bogenschütze hat einen schlechten Tag, die Witze seines Kameraden sorgten dafür das er bereits Kästchen in
der dritten Reihe seines geistigen Monitors abstreichen musste (Malus von 2). Der feindliche Schütze hat ihn bereits
so verletzt das er in der zweite körperlichen Reihe ist (Malus von 1). Beim zurückschiessen verringert sich die
Anzahl seiner Würfel um 3.

Osric hatte sich in der Barschlägerei gründlich verschätzt. Die Bardame zu begrabschen während ihr drei Brüder da
waren, war nicht die beste Idee. Etwas später ist bis auf zwei Kästchen sein Monitor voll mit Strichschaden. Da
erwischt ihn ein Tritt für ganze 7 Schadenspunkte. Der Spieler macht die letzen beiden Kästchen voll, mit den
verbleibend Schaden von 5 macht er Strich zu Kreuzschaden. Sollten die Brüder weiter auf ihn eintreten so würde er
weiter Strich zu Kreuzschaden machen. Sollten sie sogar solange auf ihn eintreten bis er nur Kreuzschaden auf dem
Monitor hat, so würde er anfangen Kreuz in Sternschaden umzuwandeln.

\subsubsection{Heilung von Schaden}

Hier wird die natürliche Heilung eines Charakters beschrieben. Je nach Stil der Kampagne kann der Spielleiter
fordern das Wunden versorgt werden müssen bevor sie heilen können. Sternschaden heilt nur wenn kein anderer Schaden
auf dem Monitor ist und die Ursache beseitigt ist. Damit Schaden heilt benötigt der Charakter Ruhe und evtl Schlaf.

Für körperlichen Schaden (Kreuz und Strich) kann ein passende Fertigkeit zur Wundversorgung gewürfelt werden,
Mindeswurf ist 16 (kann und sollte vom Spielleiter nach Umständen modifiziert werden). Versorgte Wunden heilen
doppelt so schnell.


\begin{small}
\begin{tabular}{|m{3cm}|m{5cm}|m{5cm}|}
\hline
\textbf{Schaden}&\textbf{Beispiel}&\textbf{nat. Heilung}\\
\hline
\hline
geist. -&magischer Effekt, mentale Erschöpfung&1 Reihe pro Minute\\
\hline
geist. +&magische Erschöpfung, lange geistige Arbeit&1 Reihe pro Stunde\\
\hline
geist. *&Geisterpakt, Seelenschaden&1 Kästchen pro Tag\\
\hline
körp. -&stumpfe Waffen / Fäuste&halbe Reihe pro Stunde\\
\hline
körp. +&Lethale Waffen, Sturz&halbe Reihe pro Tag\\
\hline
körp. *&Todesmagie, schwere Krankheit&1 Kästchen pro Tag\\
\hline
\end{tabular}
\end{small}

\begin{mdframed}[hidealllines=true, backgroundcolor=black!10]
\paragraph{Spielleiterhinweis}

Hier ein paar Vorschläge für Hausregeln um Verletzungen gefährlicher zu machen.

\begin{itemize}
\item Charaktere die durch Kreuzschaden bewusstlos sind, sterben weiter es sei es wird eine Probe auf erste Hilfe für den Charakter angewendet. Der Mindestwurf ist 18. Bei Erfolg ist der Charakter stabil. Die Probe dauert ein Minute. Solange der Charakter nicht stabil ist bekommt er ein weiteres Kästchen KreuzSchaden pro Minute. Der Spielleiter kann dem Charakter erlauben sich mit einer Konstitutionsprobe gegen 18 selbst zu stabilisieren. (Möglicherweise auch nur als Folge einer misslungenen ersten Hilfe Probe)
\item Erste Hilfe und Wundversorgungsproben werden durch die körperlichen Abzüge des Ziels modifiziert. Hier könnte entweder der Pool des Erstversorgers modifiziert werden (Umstände) oder der Mindestwurf um den Abzug erhöht werden.
\item Wenn eine Reihe körperlich mit Sternschaden gefüllt ist verliert der Charakter ein Körperteil. Entweder ganz oder es ist nur eingeschränkt nutzbar) und einen passenden Attributspunkt. Sollte nur in Kampangen verwendet werden wo der Charakter mittelfristig an Ersatz / Behebung kommt (z.B. durch Magie).
\end{itemize}

\end{mdframed}
\begin{center}
\subsection{Kampfregeln}
\end{center}

Folgende Stichpunkte beschreiben den Kampf:

\begin{itemize}
\item Er findet rundenweise statt, Zugreihenfolge siehe Initiative.
\item Eine Runde dauert 3s.
\item Charaktere bewegen sich auf Feldern (Hexfelder sind empfohlen), ein solches Feld ist 1,5m groß.
\item Es wird i.d.R. ein Angriff gemacht (unabhängig von der Anzahl verwendeter Waffen)
\item Es kann ein Manöver verwendet werden, sofern nicht anders angegeben hält es bis zu Beginn der nächsten eigenen Runde.
\item Nur ein Manöver pro Runde möglich sofern nicht explizit anders beschrieben.
\item Nahkämpfer mit einsatzbereiter Nahkampfwaffe haben eine Kontrollzone um sich herum (Zone des Nahkampfes). Verschieden Aktionen die von Feinden in dieser Zone gemacht werden können Gelegenheitsangriffe provozieren (z.B Bewegungen, Zaubern in der Tasche kramen und ähnliches).
\end{itemize}

\begin{center}
\subsection{Initiative}
\end{center}

Sollte es zu einem Kampf kommen so wird rundenweise agiert. Um zu bestimmen wer wann handeln darf gibt es folgende
Regeln:

\begin{itemize}
\item Reaktionsproben werden mit Geschick und Wahrnehmung gewürfelt. Reaktionsproben sind in Überraschungsmomenten angebracht und können auch ausserhalb des Kampfes verwendet werden (Jemanden rechtzeitig festhalten der abrutscht).
\item Initiative ist eine Reaktionsprobe (Geschick und Wahrnehmung). Das höchste Ergebnis beginnt gefolgt vom Zweithöchsten usw. Haben zwei Charaktere das gleiche Ergebnis so können sie falls möglich gleichzeitig handeln. Dann werden erst die Ergebnisse aller Handlungen bestimmt und erst danach angewendet (Zwei Charaktere schlagen gleichzeitig im Nahkampf aufeinander ein und verletzen sich. Die Wunden und ihre Abzüge kommen aber erst nach den gleichzeitigen Handlungen zur Geltung).
\item Nach Möglichkeit sollte sich der Kampf aber so sortieren das eine Partei komplett zieht und dann die Nächste. (Spieler und Verbündete gehen hier vor).
\item Dann kann die handelnde Seite die Aktivierung ihrer Charaktere in beliebiger Reihenfolge machen.
\item Eine Runde beginnt für einen Charakter sobald er beginnt zu handeln (z.B. erleidet er erst Schaden durch Gift und handelt dann, hier empfiehlt sich falls möglich einen anderen Charakter vorher handeln zu lassen der das Gift heilt).
\end{itemize}

\begin{mdframed}[hidealllines=true, backgroundcolor=black!10]
\paragraph{Spielleiterhinweis}

Der hier vorgeschlagene Hinweis eine Seite komplett handeln zu lassen ist natürlich vom Setting und insbesondere
vom Stil der Kampagne abhängig. Für Spielrunden die ihre Handlungen absprechen wollen und ihre Handlungen
entpannt planen ist diese Regel sinnvoll, der Kampf kann dadurch sogar taktischer und strategischer werden. Für
Spielrunden die das Absprechen untereinander an die Situation anpassen und spontan auf unplanbare Situation regieren
wollen ist es empfohlen einfach alle Handlungen in Initiativreihenfolge durchzuführen. Diese Kämpfe sind eher
handlungsorientiert (actionreicher) erfordern aber auch mehr Organisation durch die Spielleitung.

\end{mdframed}
\subsubsection{Verzögern von Handlungen}

Manchmal ist es nicht opportun für einen Charakter jetzt schon zu handeln, er kann seine Handlung verzögern:

\begin{itemize}
\item Die Initiative des Charakters „rutscht“ bis zum Ende des Kampfes
\item Verzögern kann nach einer anderen Handlung abgebrochen werden
\item Verzögern kann Handlungen anderer unterbrechen (notfalls vergleichende Reaktionsprobe der beiden Kontrahenten) falls eng gefasste Bedingungen eintreffen die beim Verzögern genannt wurden.(z.B. Armbrust abfeuern wenn jemand um die Ecke kommt.)
\end{itemize}

\begin{center}
\subsection{Handlungen pro Kampfrunde}
\end{center}

Pro Runde kann der Charakter folgende Handlung ausführen:

\subsubsection{passive Handlung}

Sachen die der Charakter nebenbei macht:

\begin{itemize}
\item Der Charakter kann sich Bewegen. Er legt eine Anzahl Felder in Höhe der Geschwindigkeit zurück.
\item Er kann eine Waffe oder einen Gegenstand bereitmachen. Weiter passive Aktivierung sind dabei möglich, wenn passend.
\item Eine Waffe nachladen, wieviele Handlungen benötigt werden hängt von der Waffe ab.
\item prinzipiell Handlungen die „nebenbei“ passieren und routiniert/unterbewusst gesteuert werden
\end{itemize}

\subsubsection{aktive Handlung}

die aktive Handlung finded statt wenn der Charakter ''dran'' ist (bestimmt durch Aktivierung und Initiative). Sie
findet nach der Bewegung statt. Hier ein paar Beispiele für eine aktive Handlung:

\begin{itemize}
\item Angreifen, Manöver oder Talent einsetzen. Einige haben spezielle Regeln und gehen über mehrere Handlungen!
\item Nutzen eines Gegenstands. z.B. Aktivieren eines magischen Gegenstands, sofern er nicht in die Kategorie Angriff fällt. Hier werden Gelegenheitsangriffe provoziert.
\item Aktiv eine Fertigkeit einsetzen. Insbesondere wenn eine Fertigkeitsprobe gewürfelt wird (z.B. Klettern).
\item Sollten zwei Fertigkeitsproben oder mehr in einer Runde gewürfelt werden (z.B. Klettern und Zaubern) so werden alle Proben um eine Anzahl Würfel erschwert die der Anzahl der verwendeten Proben entspricht. Zwei Sachen gleichzeitig machen bedeutet also -2W auf beide Proben.
\item Zusätzliche passive Handlung durchführen. Z.B. das Nachladen einer Waffe weiter durchführen oder Laufen. Beim Laufen hat der Charakter doppelte Bewegungsreichweite diese Runde. Hier kann der Charakter zusätzlich noch sprinten um seine Bewegung wie im Kapitel Bewegung beschrieben ist weiter zu verbessern).
\end{itemize}

\subsubsection{freie Handlungen}

Sehr einfache Sachen die der Charakter ungestört nebenbei machen kann. Im Gegensatz zu den anderen Handlungen kann
ein Charakter mit Zustimmung des Spielleiters mehrere machen.

\begin{itemize}
\item Er könnte passiv Wahrnehmen (nicht Suchen oder Umschauen!) und so z.B. eine bessere Kampfübersicht bekommen.
\item Eine freie Handlung (zum Reden, Kommandieren) steht dem Charakter in jedem seiner Züge zu.
\item Passive freie Handlungen (z.B Ausweichen bei Flächeneffekten, Festhalten beim Absturz) können zusätzlich dazu durchgeführt werden. Hier sollte die Anzahl aber begrenzt werden oder die Umstände die freie Handlung erlauben.
\end{itemize}

\subsubsection{reaktive Handlung}

Handlung für Charaktere die gerade nicht am Zug sind. In den meisten Fällen werden hier Nahkämpfer
Gelegenheitsangriffe durchführen. Dabei gilt zu beachten man hat nur eine reaktive Handlung pro Runde, d.h. man kann
ohne besondere Talente nur einen Gelegenheitsangriff durchführen.

\subsubsection{Bewegung}

Charaktere müssen sich im Kampf of repositionieren. Dabei gelten folgende Regeln:

\begin{itemize}
\item Geschwindigkeit gibt an wie viele Felder ein Charakter mit seiner Bewegunghandlung (passive Handlung) ziehen kann.
\item Die Geschwindigkeit ist 3 + (Geschick + Stärke) / 2 - Belastung (abrunden).
\item Charaktere dürfen durch Felder mit befreundeten Einheiten durchziehen
\item Bewegung provoziert Gelegenheitsangriffe ausser die aktive Handlung ist ''Absetzen'' (und entfällt damit für weitere Handlung wie z.B. Doppelzug)
\item Auch die aktive Handlung kann zum Bewegen genutzt werden. Der Charakter zieht eine Zahl von Feldern in Höhe seiner doppelten Geschwindigkeit
\item Zusätzlich kann der Charakter Sprinten. Beim Sprinten wird mit Sportlichkeit (und passendem körperlichen Attribut) gegen 14 gewürfelt. Jeder Erfolgsgrad erhöht den Bewegungswert um 1 (also erst ab 16 passiert etwas)
\item Wenn keine Bewegungshandlung als passive oder aktive Handlung durchgeführt wird so kann sich stattdessen 1 Feld mit einer freien Handlung bewegt werden. Das provoziert keinen Gelegenheitsangriff. Zwei Kämpfer können so ihre Positionen tauschen. Sie müssen dann aber auch sofort und nacheinander aktiviert werden!
\end{itemize}

\subsubsection{Angriff und Verteidigung}

Angriffe sind aktive Handlungen. Dabei gilt es mit der passenden Fertigkeitsprobe die Verteidigung des Ziels zu
erreichen. Erreicht der Angreifer diesen Wert so fügt er Waffenschaden zu. Für die Verteidigung gilt:

\begin{itemize}
\item Verteidigung ist 14 + Gesch - Belastung.
\item Liegende Gegner haben eine Verteidigung von 14 + Gesch/2 - Belastung (abrunden).
\item Überraschte Gegner haben eine Verteidigung von 14.
\item Verteidigung kann nicht unter 14 fallen.
\item Im Fernkampf kann die Verteidigung durch Deckung um 1 - 4 Punkte erhöht werden
\item Höchstmögliche Verteidgung ist 23.
\item Angriffsprobe wird für den Nahkampf mit Geschick oder bei schweren Waffen (Langschwert aufwärts) alternativ mit Stärke und der passenden Fertigkeit gewürfelt.
\item Fernkampfproben werden mit der Wahrnehmung und der Fertigkeit gewürfelt. In Kernschussreichweite kann alternativ die Geschicklichkeit verwendet werden.
\end{itemize}

Der zugefügte Schaden hägt von der Waffe ab. Es wird folgendes zusammengefasst:

\begin{itemize}
\item Grundschaden der Waffe (meist durch Würfel beschrieben)
\item Einhandwaffen und speziell angepassten Fernkampfwaffen bekommen die Stärke, zweihändigen Nahkampfwaffen die anderthalbfache Stärke (abrunden) als Bonus hinzu.
\item Erfolgsgrade (je 2 volle Punkte über der Verteidigung) bringen 1 Punkt Extraschaden.
\item Hinterhältige Angriffe verdoppeln den Schaden.
\item Gegner die eine Schwäche gegen den Angriffseffekt haben verdoppeln den entsprechenden Schaden.
\item Zwei Schadensverdopplungen werden zur Verdreifachung, Drei Verdopplungen zur Vervierfachung, usw.
\item Der Schaden wird um den passenden Schutz reduziert (z.B. durch den Rüstungschutz).
\end{itemize}

\subsubsection{Gelegenheitsangriffe}

Ein paar Beispiele für Aktionen die Gelegenheitsangriffe provozieren oder nicht provozieren. Der Spielleiter hat hier
wieder das letzte Wort. Für die Kampagne sollte aber eine einheitliche Regel gelten.

\begin{itemize}
\item Alle Art von Fernkampfangriffe provozieren Gelegenheitsangriffe, auch wenn sie sich auf ein Ziel im direkten Nahkampf richten!
\item Nutzen magischer oder alchemistischer Gegenstände die eine aktive oder passive Handlung benötigen.
\item Verzauberungen für den Nahkampf (explizite Nennung) sind davon ausgenommen.
\item Waffe bereitmachen provoziert keinen Gelegenheitsangriff. Kämpfer sollten ausgebildet genug sein...
\item Bewegung aus einem kontrollierten Feld provoziert. Das bedeutet das Betreten des ersten Feldes provoziert keinen Gelegenheitsangriff
\item Wenn der Charakter sich nur einfach die Runde bewegt (Er opfert die aktive Handlung für das ''Absetzen'') dann provoziert er keinen Gelegenheitsangriff.
\item Aufstehen falls liegend provoziert einen Gelegenheitsangriff, hier kann auch unter Eigenschutz aufgestanden werden, die aktive Handlung entfällt dann.
\item Generell alle Handlungen die es dem Charakter verbieten den Gegner im Auge zu behalten und oder sich zur Wehr zu setzen provozieren einen solchen Angriff.
\end{itemize}

\begin{center}
\subsection{Manöver}
\end{center}

Im Kampf kann ein Charakter ein Manöver durchführen. Ausser anders genannt kann nur ein Manöver gleichzeitig
durchgeführt werden. Falls nicht anders beeschrieben gelten die Boni und Mali eines Manövers die gesamte Runde.
Viele Manöver können über Wege gelernt werden, allerdings beherscht jeder Charakter folgende Manöver:

\begin{itemize}
\item Volle Abwehr: Kein Angriff → +2 Verteidigung. Hat der Verteidigende Platz um nach hinten auszuweichen so entfällt ein möglicher Waffenvorteil. Volle Abwehr kann auch durchgeführt werden wenn ein Charakter sich absetzt.
\item Sturmangriff: Erhöht die mögliche Reichweite beim Bewegen um 50\%. Läuft der Angreifende (mindesten 3 Felder Bewegung) so erhält er +1 Schaden. Sollte der Angriff fehlschlagen ist ein Stolperprobe gegen 18 fällig.
\item Mächtiger Hieb: -1 Verteidigung und -2W bei der Angriffsprobe. Ein Treffer richtet dafür 2 Schadenspunkte extra an.
\item Manöver brechen: passiv Handlung 1FP, Vergleichende Fertigkeitsprobe, gelingt diese so ist das feindliche Manöver beendet.
\end{itemize}

\begin{center}
\subsection{Waffenvorteil}
\end{center}

Waffen haben Waffenwerte um abzubilden das bessere Bewaffnung einen ernstzunehmenden Vorteil darstellt. Dieser Wert
ist wie folgt:

\begin{itemize}
\item 0: Unbewaffnet
\item 1: kurze Waffen (Messer, Dolche etc)
\item 2: Einhandwaffen (Kurz und Langschwerter)
\item 3: Waffen die mit zwei Händen geführt werden müssen
\end{itemize}

Wenn mehrere Kämpfer die zusammenstehen und am Nahkampf beteiligt sind (in ihrer letzten Handlung haben sie
angegriffen) addieren ihre Waffenwerte sowohl im Angriff als auch in der Verteidigung. Dieser Waffenwert gilt dann
für jedes Mitglied des ''Haufens.

Greift ein Angreifer mit einem besseren Waffenwert an sinkt die Verteidigung des Verteidigers. Ist sein Waffenwert
geringer oder gleich passiert allerdings nichts!


\begin{small}
\begin{tabular}{|m{3cm}|m{3cm}|}
\hline
\textbf{Waffennachteil des Verteidigers}&\textbf{Verteidigungsmalus}\\
\hline
\hline
1&1\\
\hline
3&2\\
\hline
6&3\\
\hline
10&4\\
\hline
15&5\\
\hline
\end{tabular}
\end{small}

\begin{mdframed}[hidealllines=true, backgroundcolor=black!10]
\paragraph{Spielleiterhinweis}

Wenn der Runde das zusammenrechnen des Waffenwerts zu viel ist so kann als Hausregel auch drauf verzichtet werden.
Der Spielleiter sollte dann aber nach eigenem Ermessen die Verteidigung von unterlegenen Gegnern um 1 bis 2 Punkte
reduzieren.

\end{mdframed}
\begin{center}
\subsection{Magie im Kampf}
\end{center}

Im Kampf kann Magie verwendet werden um Feinde zu schädigen. Geschoss und Strahlzauber (u.ä) haben folgende Regeln:

\begin{itemize}
\item Der Erfolgswurf beim Zaubern wird gleichzeitig als Angriffswurf gegen die Verteidigung verwendet. (Erfolgsgrade!)
\item Pro Punkt Effektstärke verursacht ein Zauber 1W4 Schaden.
\item Statt Stärke kommt die Willenskraft als Bonus hinzu wenn die Effektstärke mindestens 2 beträgt, oder die halbe Willenskraft bei Effektstärke 1
\item kontinuierliche Zauber verliehren den Bonus durch Willenskraft nach dem initialen Schaden. Sie müssen ebenfalls jede Runde mit einem neuen Angriffswurf treffen (aktive Handlung) können dabei aber Erfolgsgrade haben.
\end{itemize}

Für Flächenschaden (magisch oder nicht) gilt:

\begin{itemize}
\item Kein Schadensbonus durch Erfolgsgrade, außer bei dem Charakter der frontal vom Geschoss getroffen wird.
\item Auch die Willesnkraft erhöht nur den Schaden bei dem direkt Getroffenem.
\item Welches Feld welchen Schaden bekommt wird durch den Effekt beschrieben (z.B. siehe LEKO).
\item Charaktere dürfen sich gegen den Effekt verteidigen (keine Handlung nötig). Dazu machen sie ein Geschicklichkeitsprobe gegen 20. Bei Erfolg erleiden sie nur halben abgerundeten Schaden
\end{itemize}

Heilmagie hat keine Erfolgsgrade und heilt 1W4 pro Effektstärke. Sie bekommt einen Bonus durch Willenskraft wie
bei Schadenszaubern beschrieben.

Schildzauber können auf Charaktere gewirkt werden:

\begin{itemize}
\item Sie schützen den Charakter vor eintreffenden Schaden
\item Sie haben 5 Lebenspunkte pro Effektstärke (auch hier gibt es einen Bonus durch Willenskraft wie oben beschrieben)
\item Schaden geht erst ins Schild, das ist nicht durch die Rüstung des Charakters geschützt. Ist das Schild zerstört wird verbleibender Schaden durch die Rüstung des Charakters reduziert und dann auf seinem Monitor notiert.
\end{itemize}

\begin{center}
\subsection{Zustände}
\end{center}

Charaktere können von Zuständen betroffen sein. Diese haben meist negativen Auswirkungen. Charaktere die von einem
Zustand betroffen sind notieren sich diesen zusammen mit seiner Stärke. Jeder Zustand hat Regeln wie seine Stärke
abklingt. Ist die Stärke auf null oder weniger so verschwindet der Zustand wieder vom Charakter.

Sollte ein Charakter erneut einen bereits vorhandenen Zustand bekommen so addiert er die Stärken. Für jeden Zustand
gibt es Regeln die die Auswirkung beschreiben. Falls es Regeln gibt bei denen Proben fällig werden, so sind diese nur
bei Erhalt des Zustands zu würfeln und nicht beim Regenerieren.

\paragraph{Vergiftet}

Schadensgifte können durch schnellwirkende natürliche Gifte oder durch magische/alchemistische Effekte verursacht
werden. Mögliche magischen Effekte sind AN MANI KONFAR oder MORT KONFAR. Entsprechend heilt ein MANI KONFAR
Vergiftungen. Giftstärke beim magischen Vergiften oder Entgiften entspricht 3 * Effektstärke + Willenskraft. Ein
vergifteter Charakter erleidet zu Beginn seiner Runde Schaden und reduziert danach die Giftstärke um die Anzahl der
Schadenswürfel.


\begin{small}
\begin{tabular}{|m{3cm}|m{4cm}|m{5cm}|}
\hline
\textbf{Vergiftungsstärke}&\textbf{Schadenswürfel (W4)}&\textbf{Besonderheit beim Erhalten}\\
\hline
\hline
1&1&Konstitutionsprobe gegen 14, falls geschafft keine Vergiftung\\
\hline
2&1&Konstitutionsprobe gegen 18, falls geschafft keine Vergiftung\\
\hline
3&1&-\\
\hline
4&2&-\\
\hline
5&2&-\\
\hline
6&2&-\\
\hline
7&3&-\\
\hline
8&3&-\\
\hline
9&3&-\\
\hline
10&3&-\\
\hline
11&4&-\\
\hline
12&4&-\\
\hline
13&4&-\\
\hline
14&4&-\\
\hline
15&4&-\\
\hline
16&5&-\\
\hline
17+&5&Konstitutionsprobe gegen 20, Tod bei misslingen.\\
\hline
\end{tabular}
\end{small}

\paragraph{Gelähmt}

Der Charakter ist nicht in der Lage sich vernünftig zu bewegen. Er bekommt einen Malus auf alle körperlichen Aktionen.
Am Ende seiner Runde wird der Status ''Gelähmt'' um diesen Malus reduziert.


\begin{small}
\begin{tabular}{|m{2cm}|m{3cm}|m{6cm}|}
\hline
\textbf{Effektstärke}&\textbf{Handlungsmalus}&\textbf{Geschicklichkeitsmalus}\\
\hline
\hline
1&1&-\\
\hline
2&1&-\\
\hline
3&2&-\\
\hline
4&2&1\\
\hline
5&2&1\\
\hline
6&3&1\\
\hline
7&3&2\\
\hline
8&3&2\\
\hline
9&3&2\\
\hline
10&4&2\\
\hline
11+&4&3\\
\hline
\end{tabular}
\end{small}

Sollte ein Charakter von einem Lähmeffekt der Stärke 11 oder mehr bekommen so muss er eine Konstitutionsprobe gegen
10 + Effektstärke ablegen. Mißlingt diese ist er solang handlungsunfähig bis er nicht mehr gelähmt ist oder die Probe
gelingt (darf am Ende der Runde des Charakters widerholt werden)

\paragraph{Benommen}

Der Charakter ist nicht in der Lage sich zu konzentrieren. Er bekommt einen Malus auf alle geistigen Fertigkeiten und
solche die auf Wahrnehmung basieren. Am Ende seiner Runde wird der Status ''Gelähmt'' um diesen Malus reduziert.


\begin{small}
\begin{tabular}{|m{2cm}|m{3cm}|m{6cm}|}
\hline
\textbf{Effektstärke}&\textbf{Handlungsmalus}&\textbf{Intelligenz / Wahrnehmungsmalus}\\
\hline
\hline
1&1&-\\
\hline
2&1&-\\
\hline
3&2&-\\
\hline
4&2&1\\
\hline
5&2&1\\
\hline
6&3&1\\
\hline
7&3&2\\
\hline
8&3&2\\
\hline
9&3&2\\
\hline
10&4&2&\\
\hline
11+&4&3\\
\hline
\end{tabular}
\end{small}

Sollte ein Charakter von einem Benommeneffekt der Stärke 11 oder mehr bekommen so muss er eine Willenakraftprobe gegen
10 + Effektstärke ablegen. Mißlingt diese ist er solang handlungsunfähig bis er nicht mehr benommen ist oder die Probe
gelingt (darf am Ende der Runde des Charakters widerholt werden).

\paragraph{Geblendet}

Der Charakter bekommt einen Malus für alle Handlungen die irgendwie auf Sicht basieren. Das inkludiert Nahkampf und
Magie die auf nicht bereitwillige Ziele gewirkt werden soll. Die Effektstärke sinkt um 1 pro Runde (Am Ende der Runde
des betroffenen Charakters).


\begin{small}
\begin{tabular}{|m{2cm}|m{3cm}|}
\hline
\textbf{Effektstärke}&\textbf{Handlungsmalus}\\
\hline
\hline
1&1\\
\hline
2&1\\
\hline
3&2\\
\hline
4&2\\
\hline
5&2\\
\hline
6&3\\
\hline
7&3\\
\hline
8&3\\
\hline
9&3\\
\hline
10+&4+\\
\hline
\end{tabular}
\end{small}

Ein Charakter der von einer Effektstärke von 10 oder mehr betroffen so ist er blind. Sichtbasierte Wahrnehmungsproben
scheitern automatisch.

\begin{center}
\section{Attribute}
\end{center}

Attribute bezeichen die angeborenen Stärken und Schwächen eines Charakters. Sie beschreiben ihn sehr allgemein und
lassen sich nicht so ohne weiter verändern. Jeder Charakter bekommt eine Menge an Attributspunkten die bei der
Erschaffung verteilt werden.

Ein durchschnittliches Attribut sollte den Wert von 3 haben. Ein Wert von 1 ist sehr schlecht und könnte ein Hinweis
auf eine Behinderung sein. Ein Wert von 5 stellt eine ausergewöhnliche Begabung dar.

Fällt ein Attribut eines Charakters auf den Wert 0 so kann es nicht mehr gebraucht werden! Alle Fertigkeitsproben
mit diesem Attribut mislingen automatisch. Das hat drastisch und oft tödliche Konsequenzen!

\paragraph{Beispiel}

Die Spielerin von Vex will bei der Erschaffung ihrer Diebin zwei aussergewöhnliche Attribute verteilen. Sie wählt das
Attribut Glück auf 1. Vex ist also ein absoluter Pechvogel die in jeder Situation die irgendwie mit Glück zu tun hat
den kürzeren ziehen wird. Als Entschädigung bekommt sie eine Geschicklichkeit von 5. Sie ist gewand wie ein Katze und
wird damit hoffentlich vieler Situation in die sie auf der Suche nach dem besten Fettnäpfchen geraten ist entkommen.
Als Stärke bekommt Vex eine 2 verpasst was eher unterdurchschnittlich ist aber mit dem Bild das die Spielerin von Vex
hat (eher klein und sehnig). Eine überdurchschnittliche Persönlichkeit von 4 soll ihrem hübschen Aussehen sowie ihrer
gutgelaunten frech offensiven Art gerecht werden.

\begin{center}
\subsection{Charaktererschaffung}
\end{center}

Attribute werden zur Charaktererschaffung festgelegt.

Junge Charaktere erhalten pro Attribut 3 Verteilungspunkte. Sie dürfen diese
nach Belieben auf die Attribute verteilen. Folgende Regeln sind aber zu beachten.

\begin{itemize}
\item Kein Attribut darf auf 0 sein
\item Kein Attribut darf auf 6 oder höher sein.
\item Nur ein Attribut darf auf 1 sein
\item Nur ein Attribut darf auf 5 sein
\end{itemize}

Danach können bei Bedarf Modifikation durch Herkunft oder Rasse (je nach Setting) angewandt werden. Auch dabei ist zu
beachten das kein Attribut auf null sinken darf!

Sollte ein Spieler unzufrieden sein so sollte man eine ''Reparatur'' in gemeinsamer Absprache zu Beginn der Geschichte
durchführen (Spieler stellt fest das der Charakter für seine Vorstellung mehr Geschick braucht und fragt ob er diese
mit Punkten aus Persönlichkeit aufpäppeln kann)

\begin{mdframed}[hidealllines=true, backgroundcolor=black!10]
\paragraph{Spielleiterhinweis}

Ältere Charaktere (so in den 20er) sollten einen weiteren Attributspunkt haben, Kinder bzw. junge Teenager je nach
Alter deutlich weniger.

Eine Kampagne könnte die ersten Lebensjahre durchaus auspielen. So könnten die Charaktere im Alter von 10 Jahre
mit einem stark reduzierten Attributsset beginnen und für jeden Lebensabschnitt ein Abenteuer erleben. Die Charaktere
bekommen einen/weitere Attributspunkte und legen passend den Hintergrund ihrer Charaktere fest.
Auch Startfertigkeitspunkte können passend vergeben werden.

\end{mdframed}
\begin{center}
\subsection{Steigerung}
\end{center}

Charaktere können Attributspunkte erwerben. Der erste kostet 10 Erfahrungspunkte der zweite 20, der dritte 30 usw.
(Immer 10 Punkte mehr als der Vorherige). Beim Erhöhen von Attributen sind folgende Punkte zu berücksichtigen.

Ein Attribut auf 5 oder höher zu steigern kostet zwei Attributspunkt falls ein anderes Attribut bereits auf 5
oder höher ist.

Steigerung nur in Absprache mit dem Spielleiter. Je nach Art der Kampagne können sehr hohe Attribute zu Problemen
führen. Der Spielleiter sollte verstehen welche Auswirkung das neue Attribut auf die Geschichte hat und entsprechend
die Steigerung erlauben oder Verbieten.

\begin{mdframed}[hidealllines=true, backgroundcolor=black!10]
\paragraph{Spielleiterhinweis}

Einige Attribute können problematisch werden. Je nach Setting und Art der weiteren Regeln kann z.B ein
Schwertkämpfer mit absurd hoher Geschicklichkeit (Steigerung durch Erfahrungspunkt zuzüglich leicht verfügbarer
magischen Verbesserung) die Balance im Kampf stören (er trifft durch hohe Geschicklichkeit gut und ist nicht zu
treffen da er eine hohe Basisverteidigung hat). Die Spieler sollten zwar ihre Charaktere optimieren können aber
dabei sollten Grenzen eingehalten werden (z.B. wenn der Spielercharakter die anderen Spielercharakter abhängt und
dadurch entweder nicht herausgefordert ist oder Kämpfe für andere zu schwer werden). Viele Attribut können auf
hohen Werten problematisch werden. Die Spieler sollten die Attribute langsam steigern damit der Spielleiter die
Veränderung in seiner Kampagne berücksichtigen kann

\end{mdframed}
\begin{mdframed}[hidealllines=true, backgroundcolor=black!10]
\paragraph{Spielleiterhinweis}

Alternative zum Steigern von Attributen. Attribute verbessern sich nicht.
Die Attribute sind nach Charaktererschaffung festgeschrieben und bleiben in Summe für alle Charaktere gleich.
Nach einem längeren Handlungsabschnitt kann in Absprache mit dem Spielleiter aber ein Attributspunkt von einem
Attribut zum nächsten geschoben werden. Z.B. der Charakter trainiert etwas wie Körperkraft und vernachlässigt dafür
seine Bildung oder zieht sich aus sozialen Interaktionen zurück. Er würde dann z.B. einen Punkt mehr Stärke bekommen
und dafür einen Punkt Intelligenz verlieren.

\end{mdframed}
\begin{center}
\subsection{Beschreibung der einzelnen Attribute}
\end{center}

Im Folgenden werden die Attribute und ihr Bedeutung vorgestellt. 

\subsubsection{Stärke}

Die körperliche Kraft und in gewissen Massen auch der kontrollierte Einsatz dieser Kraft wird durch dieses Attribut
beschrieben. Eine hohe Stärke lässt den Charakter besser schwere Sachen stemmen oder tragen. Die Wucht von
Nahkampfangriffen profitiert direkt von der Stärke des Charakters. Kräftige Charaktere neigen zu größerem und vor
allem stabilen Körperbau während schwache Charaktere eher schmächtig sind. Zusammen mit Geschicklichkeit beschreibt
die Stärke Sachen wie Schnellkraft. Zusammen mit Konstitution das körperliche Durchhaltevermögen bei extremen
Belastungen. Sinkt die Stärke eines Charakters auf null so ist er gelähmt. Das schliesst die Fähigkeit zu atmen mit
ein! Der Tod tritt innerhalb weniger Minuten ein wenn dem Charakter nicht geholfen wird. Solange ist der Charakter
bei Bewusstsein und hat vielleicht sogar Optionen sich selbst zu helfen.

\subsubsection{Geschicklichkeit}

Die Fähigkeit des Charakters seinen Körper zu kontrollieren. Balance, Gewandheit und Fingerfertigkeit werden durch
dieses Attribut dargestellt. Geschickte Charaktere neigen eher zu sehnigen und schmalen Körperbau, während
ungeschickte eher plumb und dicker sind. Zusammen mit der Wahrnehmung beschreibt es die Fähigkeit des Charakters
schnell zu reagieren. Fällt die Geschicklichkeit auf 0 so ist der Charakter gelähmt, allerdings bleiben seine
rudimentären Lebensfunktion (Atmung und Herzschlag) erhalten es droht keine unmittelbare Lebensgefahr.

\subsubsection{Konstitution}

Beschreibt das körperliche Durchhaltevermögen des Charakters. Ausdauer und Gesundheit sowie die Widerstandskraft gegen
(körperliche) Verwundungen, Krankheit und Giften wird mit diesem Attribut abgedeckt. Ein Charakter mit hoher
Konstitution wirkt gesünder während einer mit niedriger eher kränklich wirkt. Die Konstitution zusammen mit
Willenskraft gibt an wie gut ein Charakter Erschöpfung oder körperliche Traumata widerstehen kann. Sinkt die
Konstitution auf 0 so erleidet der Charakter einen Herzstillstand, unverzügliche Hilfe ist nötig (beenden des Zustands
das die Konstitution auf 0 setzt und Reanimation).

\subsubsection{Wahrnehmung}

Die Fähigkeit des Charakters Details seiner Umgebung aufzunehmen und schnell zu verarbeiten. Die Genauigkeit der
Sinne des Charakters werden mit diesem Attribut beschrieben. Charaktere mit hoher Wahrnehmung zeichnen sich durch
ihre Aufmerksamkeit aus. Zusammen mit der Geschicklichkeit wird die Reakionsgeschwindigkeit des Charakters bestimmt,
zusammen mit Glück kann der Charakter prüfen ob er möglicherweise was wertvolles entdeckt. Sinkt die Wahrnehmung des
Charakters auf null so kann er nicht auf seine Aussenwelt reagieren (er ist blind und taub, Angriffe sieht er nicht
kommen) er bleibt aber handlungsfähig (wenn auch äußerst beschränkt). Charaktere mit Beeinträchtigung ihrer
Sinnesorgane können ihre Wahrnehmungsattribut für jedes Sinnesattribut führen (sofern das nötig ist). Ein blinder
Charakter wird den schleichenden Dieb nicht sehen aber vielleicht hören. Meist reicht aber ein Vermerk wie blind
oder taub um die Beeinträchtigung festzuhalten.

\subsubsection{Intelligenz}

Dieses Attribut gibt an wie klug der Charakter ist. Es behandelt analytische und sprachlich Fähigkeiten sowie das
Gedächtnis des Charakters. Kluge Charaktere sind in der Lage schnell Informationen zu verarbeiten und intuitiv
Berechnungen anzustellen oder gute Schätzungen abzugeben. Sie neigen dazu sich umfangreiches Wissen (über entsprechende
Fertigkeiten) anzueignen. Sinkt die Intelligenz eines Charakters auf 0 so ist er handlungsunfähig, defensive und
instinktive/unterbewusste Handlungen werden weiter durchgeführt. Der Charakter gilt als benommen und führt keine
Handlungen mehr aus.

\subsubsection{Willenskraft}

Die geistige Stabilität des Charakters, sein Antrieb und mentaler Durchhaltewille werden durch dieses Attribut
abgebildet. Willenstarke Charaktere lassen sich meist nicht so einfach überzeugen und beeindrucken. Willensschwache
erliegen eher Versuchungen. In stressigen Situation behalten willenstarke Charakter länger den Überblick. Sie
beschreibt ausserdem die Wucht mentaler Attacken, sofern zu treffend. Sinkt die Willenskraft eines Charakters auf 0
so verliert er jeden Antrieb, inklusive seines Überlebenswillens. Er wird sich hinsetzen und die Umgebung Umgebung
sein lassen (mentaler Zusammenbruch)

\subsubsection{Persönlichkeit}

Die Persönlichkeit beschreibt die soziale Intelligenz des Charakters, seine Fähigkeit mit Anderen zu interagieren,
auzutreten und Eindruck zu schinden. Charaktere mit hoher Persönlichkeit neigen zu extrovierten Auftreten und stehen
meist im Mittelpunkt, Charaktere mit niedriger Persönlichkeit sind eher Aussenseiter. Auch Führungsqualitäten werden
durch dieses Attribut beschrieben. Charaktere mit hoher Persönlichkeit werden meist von Aussenseitern für
vertrauenswürdig gehalten (egal ob sie es sind oder nicht).

\subparagraph{Hinweis}

Körperliche Attraktivität ist mit Absicht nicht explizit erwähnt. Sie geht meist Hand in Hand mit der Persönlichkeit
(hübsche Charaktere wissen um ihren Vorteil und nutzen ihn instinktiv, hässliche Charaktere halten sich aus Erfahrung
eher im Hintergrund). Die Persönlichkeit kann also zu Rate gezogen werden. Bei der Anziehung zwischen Charakteren die
sich länger kennen sollten aber auch andere Faktoren in Betracht gezogen werden. (z.B. hohe körperliche Attribute
deuten auf einen gesunden sportlichen Körper hin der allgemein als attraktiv empfunden wird). Charaktere haben
durchaus unterschiedliche Vorlieben (möglicherweise fühlt sich ja jemand von Intelligenz angezogen) und das sollte
bei Charakteren in der Geschichte, sofern wichtig, ausgearbeit werden. Wenn nötig sollte die Spielrunde hier
Absprachen treffen (z.B. könnte Persönlichkeit immer mit Attraktivität gleichgesetzt werden, das könnte wichtig sein
wenn das Verführen anderer Charaktere für die Geschichte wichtig ist, sei es für Informationsbeschaffung oder Zugang
zu Orten).

\subsubsection{Glück}

Als übernatürliches Attribut beschreibt es ob und wieviel die kleinen Dinge im Leben des Charakters für oder gegen ihn
sind. Charaktere mit hohem Glück können schwierige Situation manchmal durch hilfreiche Zufälle überwinden während
Charaktere mit wenig Glück sich unverhofft in eben diesen Situationen wiederfinden. Unmittelbar hat das Glück Einfluss
wieviele Wiederholungswürfe zur Verfügung stehen. Sinkt das Glück auf null so passiert dem Charakter erstmal nichts
weiter (aber er sollte nichts mehr dem Zufall überlassen)

\begin{mdframed}[hidealllines=true, backgroundcolor=black!10]
\paragraph{Spielleiterhinweis}

Das Glücksattribut ist ein Attribut das der Erzählweise im Weg stehen kann. Ein variables Glücksattribut macht Sinn
wenn Attribute wie Schicksal und Vorherbestimmung in der Erzählung vorkommen. Wenn die Erzählweise eher direkte
Kausalketten und festgelegte Handlungsweisen bevorzugt empfiehlt es sich das Glück für jeden Charakter auf 3 zu
setzen und den Spielern bei der Charaktererschaffung nur die Gestaltung der restlichen Attribute zu erlauben. Eine
Runde die nicht an Glück glaubt wird dann sicherlich auch bereitwillig ihre Würfel von jeden benutzen lassen...

\end{mdframed}
\begin{center}
\section{Fertigkeiten}
\end{center}

Fertigkeiten beschreiben erworbenes Wissen, handwerkliches oder körperliches Können. Sie beschreiben seine Berufung
und werden durch Zeit und Übung erlernt und erworben. Charaktere bekommen zu Start ein paar Fertigkeiten.

Eine Fertigkeit auf 1 zu erwerben bedeutet das der Charakter praktische Erfahrung mit dieser Fertigkeit hat. Je nach
Art der Fertigkeit kann das einige Zeit dauern diese Fertigkeit zu erwerben. (Man erlernt Autofahren nicht über
Nacht, aber möglicherweise kann man sich Geschichtswissen für eine überschaubare Epoche anlesen. Ein Schläger wird
die ersten Fertigkeitspunkte einfach so erwerben, usw.) Der Spielleiter entscheidet ob der Erwerb einer Fertigkeit
zulässig ist, auch bei der Charaktererschaffung!

Eine Fertigkeitsstufe von 3 zeugt von Erfahrung und Routine. Eine Fertigkeitsstufe von 5 eine Meisterschaft.

Fertigkeitsstufen sollten nicht weit über 5 hinaus gehen. Sie sollten auch nur für des Charakters ikonische
Fertigkeiten vergeben werden.

Fertigkeiten sind in Gruppen unterteilt. Innerhalb einer Gruppe sind Fertigkeiten die thematisch zusammen gehören,
sie sollten nicht als ''Klassenfertigkeiten'' missverstanden werden! So wird ein Schwertkämpfer sicherlich
kämpferische als auch körperliche sowie technische Fertigkeiten erwerben!

\begin{mdframed}[hidealllines=true, backgroundcolor=black!10]
\paragraph{Spielleiterhinweis}

Der Umgang mit nicht erworbenen Fertigkeiten ist schwierig. Ein Charakter dem eine Wissensfertigkeit fehlt sollte
möglicherweise keine Probe gestattet werden. Während eine Probe auf Aufmerksamkeit oder die meisten Proben auf
körperlichen Fertigkeiten dann nur mit dem Attribut geworfen werden können.

Regelalternative, die Fertigkeiten werden aufgeteilt nach:
\begin{itemize}
\item natürliche Fertigkeiten: Probe nur mit Attribut + 0 Fertigkeit möglich. Erlernen und Steigern der Fertigkeiten ist einfach oder passiert nebenbei.
\item professionelle Fertigkeiten: Probe nur mit Attribut + 0 Fertigkeit möglich, aber zusätzlich 1W Malus. Das Erlernen und Verbessern der Fertigkeit sollte hin und wieder durch Anleitung unterstützt und durch intensives Training begleitet werden.
\item spezielle Fertigkeiten: Probe nur mit mindestens einem Fertigkeitspunkt möglich. Das Erlernen geht nur mit Anleitung, das Verbessern wie bei einer professionellen Fertigkeit.
\end{itemize}

\end{mdframed}
\begin{center}
\subsection{Ausweichen auf andere Fertigkeit}
\end{center}

Hin und wieder kann es sein das ein Charakter eine Probe auf eine Fertigkeit ablegen muss die er nicht hat oder nur
auf Anfängerniveau beherscht. Falls er zeitgleich eine ähnliche Fertigkeit sehr gut beherscht so kann der
Spielleiter erlauben auf die besser Fertigkeit auszuweichen. Er verhängt passend dazu einen Malus von ein oder zwei
Würfeln. Sollte der Charakter öfter von einer auf eine andere Fertigkeit ausweichen so ist er angehalten die andere
Fertigkeit zu erlernen oder zu verbessern.

\begin{center}
\subsection{Verbessern}
\end{center}

Charaktere können entweder den Basiswert einer Fertigkeit verbessern oder sie spezialisieren.

\subsubsection{Steigerung von Fertigkeiten}

Fertigkeiten werden Stufe für Stufe gesteigert. Das Erlernen einer neuen Stufe kostet Erfahrungspunkte in Höhe der
neuen Stufe. Der Spielleiter hat das Recht eine Steigerung einer Fertigkeit zu untersagen. Insbesondere das Erlernen
einer neuen Fertigkeit sowie das Steigern über eine Fertigkeitstufe von 3 oder 4 sollte nur mit Zustimmung des
Spielleiters passieren.

\begin{mdframed}[hidealllines=true, backgroundcolor=black!10]
\paragraph{Spielleiterhinweis}

Um nicht im Detail jeden Charakter zu überwachen und den Spieler die Möglichkeit zu geben den Charakter nach ihren
Vorstellungen zu formen kann man auch eine generelle Erlaubnis aussprechen. z.B.:
\begin{itemize}
\item Die Reise durch die Wildnis war hart und anstrengend. Alle körperliche Fertigkeiten können auf bis zu 3 gesteigert werden.
\item Obergrenze für Fertigkeiten ist 3, darüber hinaus kann ein Charakter zwei Fertigkeiten auf 4 erwerben (es werden von den Spielern dann sicher passende Fertigkeiten gewählt).
\item Stark progressive Preise von Trainern die Fertigkeiten bereitstellen (oder sich hinter auswählbaren Abenteuern verbergen).
\end{itemize}

\end{mdframed}
\subsubsection{Spezialisierung}

Charaktere können pro Fertigkeiten eine Spezialisierung erwerben. Diese kostet 2 Erfahrungspunkte. Bei einem zu der
Spezialisierung passenden Einatz der Fertigkeit bekommt der Charakter einen Bonus von 1W. Spezialisierungen sollten
''schmal'' gefasst werden. Spezialisierungen für Waffenfertigkeiten sind z.B. Langschwert oder Kurzschwert (Schwerter
wäre zu allgemein). Der Spielleiter hat hier wie immer das letzte Wort.

\begin{center}
\subsection{Fertigkeiten nach Gruppe}
\end{center}

\subsubsection{Akademisch}

In diesem Bereich werden alle Fertigkeiten zusammengefasst die Wissen und Bildung des Charakters wiederspiegeln.
Die Fähigkeiten können gebraucht werden um den Charakteren Interpretationen dessen zu geben was sie gesehen, gefunden
oder erlebt haben. Viele Fertigkeiten können die Bildung des Charakters widerspiegeln, allerdings hängen sie stark
von der Kampagne ab und sollten mit dem Spielleiter abgesprochen sein. Hier einige Vorschläge.

\subparagraph{Astrologie}

Bestimmen der Position verschiedenster Himmelskörper und Schlussfolgern der mystischen Bedeutung. Je nach Setting
kann das Einfluss auf die Magie haben oder eventuell sogar für prophetische Vorraussagen genutzt werden.

\subparagraph{Mathematik}

Ein Fertigkeit die dem Charakter ermöglicht mathematische Sachverhalte (z.B.Geldfluss und Besitz) zu verstehen und
zu berechnen. Auch der Umgang mit mathematischen Hilfsmitteln ist in der Fertigkeit enthalten.

\subparagraph{Pflanzenkunde}

Das Wissen welche Pflanze mit welchen Eigenschaften wie, wo und wann wächst. Kann hilfreich für den Alchemisten oder
Waldläufer sein. Das jeweilige Wissen ist aber auch z.B. über Wildniskunde (allerdings fehlen hier dann die
alchemistschen Informationen) zu bekommen

\subparagraph{Geologie}

Das Wissen über Steine und Erze und wie und wo sie zu finden sind. Auch hier kann die alchemistische Bedeutung der
verschiedenen Metalle und Edelsteine Teil der Fertigkeit sein.

\subparagraph{Kultur}

Das Wissen über eine spezielle Kultur. Diese Fertigkeit ermöglicht auch das Identifizieren eines Kulturguts. Die
Fertigkeit sollte eher für untergegange Kulturen verwendet werden und überschneidet sich dann mit Geschichte. Kultur
beinhaltet auch das Wissen über Rituale und Gebräuche

\subparagraph{Berühmte Persönlichkeiten (Gruppe)}

Hier könnte je nach Setting Wissen über eine Gruppe von Künstlern, Kulturschaffenden oder Wissenschaftlern einer
Epoche gemeint sein. Z.B. Komponisten im Barock oder antike griechische Philosophen. Umfang ist mit dem Spielleiter
abzusprechen

\subparagraph{Organisationen}

Hier könnte Wissen über eine spezielle Organisation, ihre Rituale und Gebräuche sowie ihre Vorgehensweise gemeint sein.

\paragraph{Geschichte}

Wissen über historische Abläufe. Detailwissen kann über eine eigene Fertigkeit oder Spezialisierung abgebildet
werden.

\paragraph{Magiekunde}

Kenntnis der magischen Effekte und Bedeutung der magischen Runen sind Teil dieser Fertigkeit. Sie ermöglicht einem
nicht das Wirken solcher aber das möglicherweise richtige Erkennen dieser.

\paragraph{Navigation}

Die Fertigkeit sich auf der Welt und am Himmel zurechtzufinden. Diese Fertigkeit beinhaltet das Wissen über
Geographie, Astrologie sowie den Umgang mit Karten und Werkzeugen zur Positionsbestimmung. Sie hilft zwar die Richtung
zu halten, aber um sich in der Wildnis zurechtzufinden sind andere Fertigkeiten zu verwenden.

\subsubsection{Diebesfertigkeiten}

Fertigkeiten die dem Charakter helfen sich Zutritt zu verschaffen oder Gegenstände zu entwenden.

\paragraph{Feinmechanik}

Umgang mit Uhren, Schlössern, mechanischen Automaten und Fallen. Diese Fertigkeit bildet viel ab wird aber meist
verwendet um sich ungewollten Zutritt zu verschaffen.

\paragraph{Taschendiebstahl}

Fingerfertigkeit und die Fähigkeit einem anderen Charakter Gegenstände zu entwenden. Auch Taschenspielertricks wie
das Hütchenspiel oder Zaubertricks (ohne echte Magie) können mit dieser Fertigkeit vollbracht werden.

\paragraph{Verkleiden}

Sein Aussehen so zu verändern das man nicht erkannt oder sogar für jemand anderes gehalten zu wird. Auch
Schauspielen fällt hier mit rein.

Ein Charakter der eine Berfusfertigkeit Schauspiel o.ä. beherscht kann diese adequat verwenden.

\subsubsection{Fernkampf}

Hier befinden sich die Fertigkeiten um Gegenstände oder Geschosse zielsicher auf entfernte Feinde zu bringen.

\paragraph{Armbrust}

Alle Bogen oder bogenähnliche Waffen die mittels spezieller Vorrichtung die Energie eines Schusses speichern und
somit nicht direkt durch die Körperkraft betrieben werden. Armbrüste können dadurch direkter gefeuert werden,
benötigen aber auch mehr Vorbereitung zum schiessen. Alle Armbrüste beliebiger Größe fallen unter diese Kategorie.
in Absprache mit dem Spielleiter könnten auch Ballisten unter diese Fertigkeit fallen.

\paragraph{Bögen}

Fernkampfwaffen die Pfeile mittels Körperkraft durch besagte Waffen abfeuern. Zu den Waffen gehören alle ''Bögen'' sei
es ein einfacher Kurz oder ein hochkomplexer Kompositbogen, egal ob zu Fuß oder vom Reittier.

\paragraph{Werfen}

Alle Waffen die der Charakter relativ direkt ohne Umwege auf den Feind schleudert. Unter diese Kategorie fallen
Waffen wie Wurfspeere, Wurfäxten oder Wurfmesser. Auch das effektive Werfen von Steinen oder exoktischen Waffen wie
Wurfsternen u.ä. Falls mit dem Spielleiter abgesprochen können hier auch Waffen wie Schleudern abgebildet werden
(oder sie benötigen eine eigene Fertigkeit).

\subsubsection{Handwerk}

Her sind die Fertigkeiten aufgelistet die den Charakteren ermöglichen einem Beruf nachzugehen oder Sachen zu
erschaffen oder Dinge zu warten oder repariern (usw.). Einige der Fertigkeiten sind für einen Abenteuerer nützlich
anderere sind eher für NSCs gedacht.

\paragraph{Beruf}

Ein Charaker kann einen Beruf erlernt haben. Da Berufsfertigkeiten erfahrungsgemäß wenig im Spiel verwendet werden,
wird hier mit einer Fertigkeit sowohl der praktische als auch theoretische Teil abgehandelt. Berufsfertigkeiten kann
es viele geben und hängen auch von der Kampagne ab. Hier sollte Rücksprache mit dem Spielleiter gehalten werden.
Einige Berufsfertigkeiten können wenn es passt auch ersatzweise für andere Fertigkeiten verwendet werden. So könnte
ein Feinmechaniker/Uhrmacher/Schlosser bestimmt auch ein Schloss öffnen während er wahrscheinlich keine Ahnung vom
Fallen entschärfen hat. Hier ein paar Beispiele für Berufe:

\subparagraph{Architekt}

Fähigkeit Gebäude zu planen (Statik, Material, etc) und den Bau dieser zu beaufsichtigen. Diese Fertigkeit ist recht
umfangreich und sollte durch weiteres Wissens oder Handwerksfertigkeiten ergänzt werden.

\subparagraph{Feldarbeiter}

Kenntnis, Aufzucht und Ernten von Nutzpflanzen. Auch der erste Verarbeitungsschritt wie das Dreschen oder Trocknen
sind Teil dieser Fertigkeit. Alchemistische Pflanzen sind Nutzplanzen.

\subparagraph{Juwelier}

Umgang mit Edelsteinen und Edelmetallen zum Zwecke der Verschönerung/Verzierung oder Herstellung von
Schmuckgegenständen.

\subparagraph{Metallarbeiter (Schmied)}

Umgang mit Metallen zum Herstellen und Warten von Werkzeugen, Rüstungen, Waffen, etc. Beinhaltet auch das Schmieden
und damit die Kenntnis von Metallurgie und Erzverarbeitung. Um Feinmechaniken oder Verzierungsarbeiten oder Schmuck
herzustellen sind die entsprechenden Fertigkeiten nötig.

\subparagraph{Musiker}

Darbietungen mit Instrument und begleitendem Gesang. Ein guter Charakter kann vor Publikum spielen und es
begeistern. Möglicherweise kann ein guter Darsteller das Publikum emotional beeinflussen (z.B. aufstacheln oder
beruhigen). Falls der Spielleiter es festlegt muss diese Fertigkeit für jedes Instrument und Gesang einzeln erlernt
werden (das sollte z.B. der Fall sein wenn Musik ein zentrales Element der Kampagne ist), ansonsten sind das
passende Spezialisierungen.

\subparagraph{Schauspieler}

Die Fähigkeit des Charakters eine fremde Rolle überzeugend darzustellen. Kann verwendet werden wenn sich Charaktere
verkleiden wollen um sich ungesehen zu bewgen oder wenn sie vorgaukeln eine andere Person zu sein. Natürlich kann
ein Charakter diese Fertigkeit auch einfach dafür verwenden in einem Bühnenstück mitzuwirken.

\subparagraph{Schneider}

Der Schneider kann mit Stoffen, Fellen und Leder umgehen. Er kann aus diesen Kleidungen, Vorhänge, Zelte, etc.
schneidern und solche Gegenstände warten.

\subparagraph{Steinmetz}

Fähigkeit Gestein zu bearbeiten und Werkstoffe daraus zu Erstellen. Der Steinmetz ist in der Lage im Steinbruch z.B.
Quader für Gebäude zu ernten. Beeinhaltet Kenntnis über den Bau einfacher Bauten wie Wälle oder gepflasterte
Straßen.

\subparagraph{Uhrmacher}

Feinmechaniker der mit kleinen Federn, Zahnrädern, etc. Maschinen für einen Zweck plant, baut und wartet. Diese
Fertigkeit kann beliebig durch die Diebesfertigkeit Feinmechanik ersetzt werden oder diese ersetzen.

\subparagraph{Waldarbeiter}

Aufzucht, Pflege und Ernten von Bäumen. Der Waldarbeiter weiss um die Eigenschaften der verschiedenen Holzarten und
wie sie zu verarbeiten sind. Beeinhalten auch die ersten Arbeitsschritte wie das Erstellen von Brettern aus
Baumstämmen.

\subparagraph{Zimmermann}

Kenntnis über das Bauen und Reparieren von Holzgebäuden und Schiffen. Durch die Kenntnisse der nötigen
Mechanikgesetze ist auch der Bau von einfachem Belagerungsgerät möglich.

\begin{mdframed}[hidealllines=true, backgroundcolor=black!10]
\subparagraph{Spielleiterhinweis}

Die Idee ist hier nicht generische Superfertigkeiten zu erstellen. Sie sollten eher der Charakterbeschreibung dienen
als zu Alleskönnerfertigkeiten zu werden. Die obigen Beispiele sind Spezialfälle (also mögliche Spezialisierung) von
anderen Fertigkeiten. Ein Beruf Jäger sollte also nicht die ganze Sparte Wildniskunde und Überleben abarbeiten.

Alternativ könnte der gegenteilige Ansatz verwendet werden. Der Großteil der Fertigkeiten wird durch Berufe ersetzt.
So könnte es auch den Beruf Dieb/Magier/Krieger geben und die Charaktere werden durch ihre sehr wenigen und
übersichtlichen Berufsfertigkeiten definiert.

\end{mdframed}
\paragraph{Heilkunde}

Medizinische Fertigkeit um Verletzungen und Krankheiten zu heilen. Sie kann auch als forensische Fertigkeit
verwendet werden. Insbesondere Wundversorgung können im Laufe der Kampagne wichtig werden (näheres dazu in den
Kapiteln zum Kampf).

\subsubsection{Körperlich}

Fertigkeiten die es dem Charakter ermöglichen sich zu bewegen und Körperkontrolle zu behalten.

\paragraph{Aufmerksamkeit}

Die Verwendung der eigenen Sinne und Interpretation der erhaltenen Informationen wird mit dieser Fertigkeit
ermittelt. Spuren und Geheimnisse finden können mit dieser Fertigkeit gefunden werden (evtl. ist aber eine
handwerkliche Fähigkeit angebracht). Ob ein Charakter z.B. durch Gehör oder Geruch gewarnt zu wird, wird mit dieser
Fertigkeit ermittelt.

\paragraph{Gleiterflug}

Der Umgang mit den diversen Gleitern. Gleiter ermöglichen es Charakteren eingeschränkt zu fliegen. Für den Start, die
Landung und allem Dazwischen wird diese Fertigkeit benötigt. Der Charakter lernt ausserdem die verschieden Gleiter
einzuschätzen und zu warten.

\paragraph{Heimlichkeit}

Sich lautlos und ungesehen fortzubewgen wird auf dieser Fertigkeit gewürfelt. Meist in direktem Vergleich zu
Wahrnehmung und Aufmerksamkeit.

\paragraph{Sportlichkeit}

Alle athletischen Herausforderungen vor den ein Charakter stehen könnte, werden mit dieser Fertigkeit dargestellt.
Laufen und Springen sei es um schnell zu sein, Hindernisse zu überqueren oder andere herausfordernde Situationen zu
meistern. Zur Fertigkeit gehört auch in gewissen Rahmen Körperkontrolle und Ausdauer (wie atme ich richtig welche
Geschwindigkeit kann ich länger laufen usw.)

\subsubsection{Magie}

Fertigkeiten des Übernatürlichen. Ihre genaue Verwendung wird meist in einem eigenen Kapitel geklärt. Die
Fertigkeiten stellen Würfel für die entsprechenden Proben.

\paragraph{Alchemie}

Die Befähigung Tränke Gifte und Essenzen aus Zutaten herzustellen. Genaueres im entsprechenden Kapitel. Die Fertigkeit
umfasst auch die Kenntnis über alchemistische Zutaten und ihre Gewinnung. Ausserdem ermöglicht die Fertigkeit
alchemistische Erzeugnisse wie Tränke zu identifizieren.

\paragraph{Kristallisieren}

Zucht von Splittern im Nebel und Kondensieren bzw Herstellen von magischen Kristallen werden mit dieser Fertigkeit
abgebildet. Charaktere die hohe Werte in dieser Fertigkeit haben, bekommen ein intuitives Verständnis des Nebels.

\paragraph{Runenmagie}

Das Wirken von aus Runenworten gebildeten Zaubersprüchen. Genaueres im Kapitel Magie.

\paragraph{Verzaubern}

Herstellen von magischen Gegenständen. Siehe entsprechendes Kapitel.

\subsubsection{Nahkampf}

Alle Fertigkeiten die darauf Abzielen mit oder ohne Hilfe von Schlagverstärkern einen Gegner zu überwältigen.

\paragraph{Handgemenge}
Die Fähigkeit sich unbewaffnet oder fast unbewaffnet zur Wehr zu setzen. Vom Ringkampf übers Schlagen und Treten
wird hier alles abgebildet. Auch Waffen mit kurzer Reichweite, wie Dolche und Messer oder exotischere Waffen wie
Palmsticks und Schlagringe oder Krallen werden mit dieser Fertigkeit erfasst.

\paragraph{Hiebwaffen}

Waffen die einen erheblichen Reichweitenvorteil gegenüber einer unbewaffnete Person geben und gleichzeitig relativ
''wuchtig'' sind werden mit dieser Fertigkeit abgebildet. Knüppel, Einhandschwerter, Äxte und ähnliche fallen
eindeutig in die Kategorie Waffen die mit dieser Fertigkeit abgebildet werden. Einige Zweihandwaffen wie
Zweihänder oder zweihändige Äxte können entweder mit Hiebwaffen oder mit Stangenwaffen geführt werden.

\paragraph{Spezialwaffen}

Sofern in der Kampagne vorhanden kann hier der Umgang mit besondere Waffen abgebildet werden. Eventuell ist es nötig
diese Fertigkeit noch mehrmals zu unterteilen, bzw auf dem Charakterblatt festzuhalten welche Waffengattung gemeint
ist. Vorschläge für Waffen die Spezialwaffen sind:
\begin{itemize}
\item Flegel in seinen Varianten
\item Peitschenwaffen in ihren verschiedenen Ausführungen
\item ''Fernöstliche'' Waffen wie Nunchaku oder Kusarigama
\item Doppelwaffen
\end{itemize}

\paragraph{Stangenwaffen}

Umgang mit langen Waffen (Größe ähnlich eines Menschen) werden mit dieser Fertigkeit abgebildet. Speere, Lanzen,
Hellebarden, Gleven und ähnliche. Lange Hiebwaffen wie Zweihänder oder lange Äxte können auch mit dieser Fertigkeit
geführt werden.

\subsubsection{Sozial}

Fertigkeiten die den Umgang des Charakters mit anderen Charakteren abbilden.

\paragraph{Anführen}

Die Befähigung schnell die richtigen Kommandos zu geben und andere dazu zu bringen ihnen Folge zu leisten.

\paragraph{Einschüchtern}

Die Befähigung durch drohendes Gebärden oder glaubhafter Wortwahl und Auftreten eine gegnerische Partei von der
eigenen Üerlegenheit zu überzeugen. Es empfiehlt sich je nach Art der Einschüchterung Intelligenz, Persönlichkeit
oder Stärke als Attribut zu verwenden. (Entscheidung des Spielleiters!). Mittels Einschüchterung können auch Verhöre
durchgeführt werden die mittels Druck zum Informationsgewwinn führen sollen.

\paragraph{Umgangsformen}

Bescrheibt die Befähigkung eines Charakters sich angemessen zu verhalten. Benehmen und Protokoll (sofern vorhanden)
werden damit ebenfalls dargestellt.

\begin{mdframed}[hidealllines=true, backgroundcolor=black!10]
\paragraph{Spielleiterhinweis}

Die Fähigkeit ist recht allgemein gehalten und kann bei Bedarf gesplittet werden (z.B. Umgangsformen Adel und
Umgangsformen kriminelle Unterwelt). Alternativ kann ähnlich wie bei Wildniskunde mit Spezialisierungen gearbeitet
werden. Es sollte aber eine einheitliche Regelung für die Kampagne festgelegt werden.

\end{mdframed}
\paragraph{Verhandeln / Überzeugen}

Die Kunst jemand anderes höflich und sozial zu beeinflussen und vom eigenen Standpunkt zu überzeugen. Das kann beim
Handel wie auch in sozialen Situation verwendet werden. Ein wesentlicher Aspekt dieser Fertigkeit ist das Lesen von
Menschen. Proben zum Erkennen von Motivation und emotionalen Zustand sowie das Einschätzen der Glaubwürdigkeit des
Gegenübers werden ebenfalls mit dieser Fertigkeit abgegolten (Menschenkenntnis).

\subsubsection{Wildnis}

Fertigkeiten die dem Charakter das Überleben und Bewegen ausserhalb der Zivilisation ermöglichen.

\paragraph{Jagdkunst}

Die Fähigkeit erlegte Wildtiere zu verwerten. Das beinhaltet sichere und gute Nahrung oder aber möglichst viel
Nahrung zu bekommen. Ebenso das Verwerten weiterer Überbleibsel (Fell, Haut, Sehnen, Knochen) wird mit dieser
Fertigkeit abgebildet. Die Fähigkeit zu Gerben oder Sehnen und Knochen zu extrahieren und haltbar zu machen ist
in Absprache mit dem Spielleiter möglich bzw. sollte mit dieser Fertigkeit gehandhabt werden wenn keine andere
Handwerksfertigkeit zur Verfügung steht.

\paragraph{Nebelkreaturen}

Wissen und Erfahrung über Wesen im Nebel. Diese Fertigkeit kann verwendet werden um feindliche Begegnungen im Nebel
zu umgehen (Proben z.B. mit Glück). Proben mit Intelligenz können Verhaltensweisen und Schwachstellen zu einer
identifizierten Kreatur offenbaren (Spielleiterentscheidung). Proben mit Wahrnehmung könnten z.B. gemacht werden um
Nebelkreaturen zu identifizieren (z.B. anhand gefundener Spuren, etc)

\paragraph{Nebelkunde}

Wissenfertigkeit um im Nebel zu navigieren, seinen Einflüsterung zu entgehen, seine Strömungen zu verstehen. Diese
Fertigkeit wird intuitiv von Nebelläufern und Kristallisierern erlernt. Die Fähigkeit kann verwendet werden um
Gefahren im Nebel zu umgehen oder gute Plätze für die Kristallzucht zu finden.

\paragraph{Führen von Tieren}

Dressur, Ausbildung und generelles Beherschen von Tieren. Ein vertrautes Tier wird einfache Befehle immer befolgen,
ein unbekanntes Tier muss aber erst dazu gebracht werden. Diese Fertigkeit zeigt die Fähigkeit des Charakters Tiere
zu beherschen (wütende Meute Hunde davon abzuhalten anzugreifen) oder sich in diese einzufühlen. Auch der Umgang mit
Wildtieren ist in dieser Fähigkeit dabei. (Wie erkenne ich Drohgebärden, etc.)

\paragraph{Wildniskunde}

Die Fähigkeit in der Wildnis zu überleben. Beinhaltet die Fähigkeit einfach Nahrung und Wasser zu finden, sofern
möglich. Lokale Wettervorhersagen und das Finden eines sicheren Unterschlupf gehört ebenso dazu. Um Spuren von
Tieren oder Menschen zu finden kann in Absprache mit dem Spielleiter ebenso mit dieser Fertigkeit gewürfelt werden.
Unterscheidung von essbarer und giftiger Nahrung ist mit dieser Fertigkeit in gewissen Umfang ebenso möglich.

\begin{mdframed}[hidealllines=true, backgroundcolor=black!10]
\paragraph{Spielleiterhinweis}

Wildniskunde kann, wenn für die Kampagne gewollt mehrmals gelernt werden. Es wird dann immer eine ''Region'' gemeint.
Verwandte Region können dann z.B. über die Ausweichen Fertigkeit abgebildet werden.

Alternativ um nicht alle klimatischen Regionen mit einer Fertigkeit darzustellen, kann auch erlaubt werden mehrere
Spezialisierungen zu erlauben. Dann darf die Probe nur mit passender Spezialisierung gemacht werden, oder bei
fehlender mit Würfelabzug. Zu Beginn der Kampagne sollte aber feststehen ob es hier eine Hausregel gibt.

\end{mdframed}
\begin{center}
\section{Wege}
\end{center}

Wege ermöglichen es den Charakter sich jenseits von Attributen und Fertigkeiten weiterzuentwickeln. Wege erweitern
die Fähigkeiten des Charakters um Talente.
Sobald ein Charakter einen Weg erlernt hat kann er nacheinander die drei dazugehörigen Talente erwerben. Diese
sind aufeinander aufbauend also muss für die dritte Technik die Zweite beherscht werden die als Vorraussetzung die
erste Technik hat.

\begin{center}
\subsection{Erlernen eines Weges}
\end{center}

Das Erlernen eines Weges erfordert Training und möglicherweise einen Trainer (Spielleiterentscheidung!). Der
Charakter muss die Vorraussetzung erfüllen und sollte in der passenden Fertigkeiten eine Stufe von 3 oder höher
vorweisen.

Das Erlernen eines neuen Wege kostet 1 Erfahrungspunkt plus einen weiteren für jeden bereits vom Charakter erlernten
Weg.

Das erste Talent eines Weges kostet 2, das Zweite 3 und das Dritte 4 Erfahrungspunkte.

\begin{center}
\subsection{Wege nach Kategorie}
\end{center}

\subsubsection{Alchemie}

Wege um die Nutzung oder Herstellung alchemistischer Substanzen zu verbessern. Alchemy ist ein Handwerk und kann
durch Wege des Handwerks verbessert werden.

\paragraph{Apparatmeister}

\begin{itemize}
\item gefährliche Prozedur: Der Alchemist kann die Kritikalität um 3 erhöhen, er halbiert dabei den Herstellungszeitraum.
\item sichere Prozedur: Der Alchemist kann bis zu 3 Punkte vom Mindestwurf der Kritikalität zum Mindestwurf der Erstellung verschieben. Das geht in beide Richtungen.
\item unterstützende Apparatur: Die Höhe des Würfelbonus der alchemistischen Ausrüstung zählt gleichzeitig als Erleichterung der Mindestwürfe für Herstellungsschwierigkeit und Kritikalität.
\end{itemize}

\paragraph{Forscher}

\begin{itemize}
\item geübte Experimente: Kritikalität um 2 Punkte (-4 Schwierigkeit) gesenkt.
\item Analyse: Der Alchemist kann eine Ja/Nein Frage zu einer Zutat stellen. Er experimentiert 8h mit der Substanz und verbraucht dabei 5 mal diese Zutat. Schwierigkeit und Kritikalität des Prozess sind 10 + 3 * Zutatenschwierigkeit
\item Farbverbindung: Ein alchemistischer Prozess geht nicht automatisch schief wenn es zwei oder mehr gleich starke Farbworte gibt. Sie werden mit den CONDUC Regeln verbunden (und nur wenn das Resultat ungültig ist ist der Prozess ein Fehlschlag). Die beteiligten Farbworte sind dadurch alle primäre Farbworte und fügen keine Kritikalität hinzu. Außerdem neutralisieren sich gegensätzlich Farben nicht länger in Ihrer Stärke!
\end{itemize}

\paragraph{Giftmeister}

\begin{itemize}
\item Praxis: Keine Selbstverletzung bei Patzern im Umgang mit Gift (die Herstellung ist hiervon nicht betroffen!)
\item Giftkenntnis: Waffen können 50\% mehr Gift speichern. Der Alchemist kann mit Alchemie oder einer passenden Wissensfertigkeit ein Gift bestimmen (Heilkunde, etc. Schwierigkeit sollte sich an der Herstellschwierigkeit orientieren)
\item optimiertes Gift: Der Alchemist fügt 1 Punkt mehr Gift bei Angriffen mit selbsthergestellten Gift zu.
\end{itemize}

\paragraph{Kesselmeister}

\begin{itemize}
\item Massenproduktion: Es wird die zehnfache Menge in der dreifachen Zeit hergestellt. Es werden dabei aber auch das zehnfache an Zutaten verbraucht und die Schwierigkeit steigt um 2.
\item Einkochen: Es wird die dreifache Menge an Zutaten und Herstellungszeit benötigt. Die Schwierigkeit ist um 2 erhöht aber dafür ist die Effektstärke um 1 höher
\item Überdosieren: Es kann ein Zutat dreimal verwendet werden. Ihre Farbeffektstärken steigen um 1.
\end{itemize}

\paragraph{Kristallisierer}

\begin{itemize}
\item Geheimnis des Nebels: Splitter können als MAGI oder AN MAGI der Effektstärke 1 (1Splitter) oder 2 (3Splitter) verwendet werden
\item Verbrauchsfoki: Verbrauchsfoki können hergestellt werden. Der Vorgang darf keine gültige Form muss aber eine gültige Farbe haben. Es werden 3 * Effektstärke plus Willenskraft Kraftpunkte hergestellt. Verbrauchsfoki können Kosten für Zaubersprüche übernehmen sofern ihre Farbe übereinstimmt (Die Farbe MAGI kann für alle Zauber verwendet werden). Die Anzahl die pro Runde verwendet werden kann ist limitiert auf die Halbe Summe aus Intelligenz und Willenskraft.
\item Runenalchemist: Verbrauchsfoki können als Zutaten verwendet werden. Sie haben eine Schwierigkeit von 3. Sie haben nur ihre Farbe mit einem Plus (Effektstärke 1).
\end{itemize}

\subsubsection{Allgemein/Handwerk}

Unspezialisierte Wege und Wege die Berufsfertigkeiten verbessern.

\paragraph{Fokus}

\begin{itemize}
\item Charakters Fokusmaximum ist um 1 erhöht
\item Charakters Fokusmaximum ist um 1 erhöht
\item Der Charakter kann zwei Fokuspunkte gleichzeitig (in einer Runde) einsetzen
\end{itemize}

\paragraph{Handwerk}

\begin{itemize}
\item Amateuer: 1W Bonus auf Fertigkeitsproben.
\item Fortgeschritten: Improvisierte Werkzeuge und Umstände. Es können bis zu 2W Mali durch Umstände oder fehlende Werkzeuge neutralisiert werden.
\item Experte: Schwierigkeit sinkt um 1
\end{itemize}

\paragraph{Juggernaut}

\begin{itemize}
\item Widerstandsfähig: Ein zusätzliches Kästchen pro Reihe auf dem körperlichen Monitor.
\item Kerngesund: Konstitution ist um 1 erhöht bei Widerstand gegen Gift und Krankheit. Gift heilt zusätzlich um einen Punkt die Runde
\item Unsterblich: Ein weiteres Kästchen pro Reihe auf dem körperlichen Monitor.
\end{itemize}

\paragraph{Schicksal}

\begin{itemize}
\item Glück: Die Anzahl der Wiederholungswürfe ist um 1 erhöht.
\item Schicksalsnetz: Wiederholungswürfe dürfen im Verhältnis 2:1 an andere (wichtige) Charaktere weitergegeben werden.
\item Heldenglück: Die Anzahl der Wiederholungswürfe steigt erneut um 1. Mittels Einsatz eines Wiederholungswurf kann der Spieler des Charakters einen gegen ihn gerichteten Wurf wiederholen lassen.
\end{itemize}

\paragraph{Wissen}

\begin{itemize}
\item Amateur: Beim Recherchieren wird der Zeitraum halbiert.
\item Fortgeschritten: Bei Würfen auf Vorhersagen/Erinnern oder allgemeine Proben bei denen der Spieler von der Spielleitung in Kenntnis gesetzt wird gibt es 2 Bonuswürfel.
\item Experte: Der Bonus ist von der Fertigkeit abhängig und sollte mit dem SL passend zur Fertigkeit geklärt werden. Der Bonus sollte aber nur in speziellen Situationen zum tragen kommen. Git es einen anderen Weg der Boni gibt so sollte die Art des Bonus eine Andere sein.
\end{itemize}

\begin{mdframed}[hidealllines=true, backgroundcolor=black!10]
\paragraph{Spielleiterhinweis}

Hier ein paar generische Vorschläge für den Bonus in der dritten Stufe

\begin{itemize}
\item Schadensbonus gegen passende Kreaturen. (Lichtpaladin könnte hier Bonus gegen Untote bekommen)
\item Widerstandsbonus. entweder 2W oder eintreffender Schaden wird um 1 reduziert.
\item Einstellungsbonus. Betroffene sind dem Charakter positiver gesonnen. Z.B Wildniskunde → Tiere sind ruhiger und greifen vl nicht an.
\end{itemize}

\end{mdframed}
\subsubsection{Chi}

Wege der östlichen Reiche. Sie fokussieren die innere Lebensenergie und erzeugen so quasi übernatürliche Effekte.

\paragraph{Weg des Chi}

\begin{itemize}
\item Meditation: Der Charakter hat zwei zusätzliche Fokuspunkte. Diese regenerieren nur wenn zusätzlich pro Punkt 1h meditiert wird. Das gilt für alle Fokuspunkte die mit diesem oder dem Meisterweg des Chi erhalten werden.
\item Chakrablockade (1FP, Maböver): Der Charakter muss das Ziel berühren (unbewaffneter Nahkampf). Das Ziel und der Charakter machen eine vergleichende Willenskraftprobe. Gewinnt der Charakter so kann das Ziel den restlichen Kampf keine Fokuspunkte verwenden. Das Ziel kann als Aktionhandlung jede Runde versuchen die Blockade durch Wiederholen der Willenskraftprobe (gegen den ursprünglichen Wert) zu beenden. Der Charakter hat einen zusätzlichen Fokuspunkt.
\item Chi-Heilung (1 FP, freie Handlung): Der Charakter kann Schadenspunkte von seinem körperlichen auf den geistigen Monitor schieben. Pro Einsatz der Fertigkeit darf er maximal die Summe aud Willenskraft und Konstitution verschieben. Der Charakter hat einen zusätzlichen Fokuspunkt.
\end{itemize}

\paragraph{DrunkenMaster}

\begin{itemize}
\item Fröhlich: Der Charakter ist automatisch in diesem Kampfstil sofern er schwer betrunken ist. Abzüge für Trunkenheit entfallen. Der Charakter hat eine um 2 erhöhte Verteidigung.
\item Albern (1FP, Manöver): Wird der Gegner getroffen so wird er so aus dem Konzept gebracht das er die nächste Runde keinen Zugang zu irgendeiner Kampfkunst hat.
\item Da war noch was (Manöver): Der Charakter verursacht keinen Schaden aber bekommt für diesen Kampf einen Fokuspunkt. Kann mit Manövern aus dem Weg der Verwirrung, Lähmung oder FEchter kombiniert werden.
\end{itemize}

\paragraph{Erde}

\begin{itemize}
\item Einstieg (1 FP): Der Charakter verliert seine Bewegungs und Aktionshandlung. Für den restlichen Kampf kann er Fertigkeiten dieses Kampfstils einsetzen. Der Schaden egal aus welcher Quelle ist um 1 reduziert. Das gilt auch für z.B. Giftschaden.
\item Steinfaust (Manöver): Der unbewaffnete Angriff des Charakters verursacht zusätzlich Schaden in Höhe der Konstitution. Kann mit mächtiger Hieb oder Techniken aus dem Weg der Stärke kombiniert werden.
\item Statue (Manöver): Die Verteidigung des Charakters ist 16. Die ersten drei Punkte Bonusschaden durch Erfolgsgrade entfallen. Der Charakter bekommt zusätzlich zwei Punkte Rüstung.
\end{itemize}

\paragraph{Feuer}

\begin{itemize}
\item Einstieg (1 FP): Der Charakter verliert seine Bewegungs und Aktionshandlung. Für den restlichen Kampf kann er Fertigkeiten dieses Kampfstils einsetzen. Dieser Kampfstil kann mit allen Nah und Fernkampfwaffen verwendet werden. Der Schaden aller Waffen ist um 1 erhöht.
\item Flächenbrand: Für jeden diese Runde eingesetzten Fokuspunkt ist der Schaden um 2 erhöht.
\item Fokussierte Flamme: Für jeden diese Runde eingesetzten Fokuspunkt ist das Ergebniss von Angriffsproben um 1 erhöht.
\end{itemize}

\paragraph{Holz}

\begin{itemize}
\item Einstieg (1 FP): Der Charakter verliert seine Bewegungs und Aktionshandlung. Für den restlichen Kampf kann er Fertigkeiten dieses Kampfstils einsetzen. Der Charakter ignoriert für den Kampf eine Reihe körperlichen Schaden. Es müssen sechs Reihen Schaden verursacht werden bevor der Charakter Handlungsunfähig ist.
\item Wildschwein: Mali für körperlicher Schaden erhöhen den Würfelpool für Angriff statt ihn zu reduzieren.
\item Bär (1 FP, als Reaktion auf das Erhalten von Schaden): Der Schaden wird um eine Kategorie niedriger makiert. Sternschaden wird zu Kreuz und Kreuz zu Strichschaden. Strichschaden der so behandelt wird verfällt.
\end{itemize}

\paragraph{Metall (Fliegende Schwerter)}

\begin{itemize}
\item Einstieg (1 FP): Der Charakter verliert seine Bewegungs und Aktionshandlung. Für den restlichen Kampf kann er Fertigkeiten dieses Kampfstils einsetzen. Das Bereitmachen einer Klingenwaffe kostet keine Aktion mehr. Waffen können vom Boden bereit gemacht werden, diese Waffen können auch in Nachbarfeldern liegen (kostet keine Aktion).
\item Wurf: Das Werfen von Klingen provoziert keinen Gelegenheitsangriff. Schwerter haben die Reichweite von Wurfmessern und diese Reichweite ist uzsätzlich verdoppelt. Fernkampfwege können mit geworfenen Klingen verwendet werden.
\item Unmögliche Bahn: Geworfene Klingen prallen einmal ab. Sie können so um Ecken fliegen oder ein zweites Ziel treffen. Beim Abprallen verliert die Klingen aber 3W für die Trefferprobe und die Stärke für den Schaden ist um 3 Punkte reduziert.
\end{itemize}

\paragraph{Schatten}

\begin{itemize}
\item Huscher: Der Charakter zählt als eine Kategorie kleiner bzw leichter wenn es darum geht zu bestimmen wie er sich verstecken oder ungesehen bewegen kann. Bei Gleichstand einer vergleichenden Probe zwischen Heimlichkeit und Wahrnehmng gewinnt der Charakter (aber normale Regeln werden bei zwei Nutzern dieser Technik angewand).
\item Mimikri (1FP): Für die Szene kann der Charakter Körperhaltung und Bewegungsmuster eines Anderen imitieren. Er kann so bei schlechter Sicht täuschen. Seine Verkleidung wirkt echter (+3W für alle Proben auf Schauspiel/Verkleiden oder ähnliches). Der Charakter muss sein Ziel ausreichend studieren um es zu imitieren.
\item Schattenschritt (1FP): Der Charakter kann von einem Schatten zum nächsten teleportieren sofern der zu erreichende Schatten gesehen werden kann und sich innerhalb der Bewegungsreichweite des Charakters befindet.
\end{itemize}

\paragraph{Weg des Strohhuts}

\begin{itemize}
\item Strohhutverteidigung: Der Charakter erhält 2W für alle Versuche die sich gegen eine Entwendung seines Strohhut richten. Das gilt auch für Wahrnehmungsproben! (Auch wenn der Charakter sonst nichts mitbekommt.)
\item Reserve: Der Charakter hat immer genug Strohhüte dabei. Wieviele? Genug! Er kann ausserdem den vollen Weg des Handwerks für die Herstellung von Strohhüten verwenden. Ja er hat auch immer genug Stroh dabei. Nein nicht nur im Kopf.
\item typisch Strohhut: Das tragen eines Strohhut gibt dem Charakter 2W für alle Proben in denen er mysteriös oder einschüchternd wirken soll. Dafür muss er aber auch eine gradlinige und schweigsame Persönlichkeit haben. Er selbst ist immun gegen Einschüchterung (er bekommt ja nichts mit). Der Charakter kann ausserdem zwei Kampftechniken kombinieren die nicht kombinierbar sind (Spielleiterentscheidung). Er hat sie schliesslich tausendmal geübt.
\end{itemize}

\paragraph{Wasser (Wu Shu)}

\begin{itemize}
\item Einstieg (1 FP): Der Charakter verliert seine Bewegungs und Aktionshandlung. Für den restlichen Kampf kann er Fertigkeiten dieses Kampfstils einsetzen. Die Verteidigung des Charakters ist für diesen Kampf um 1 erhöht.
\item Parade (reaktive Handlung): Der Charakter kann nach einem erfolgreichen Angriff diesem entgehen in dem er eine passende Probe (Kampffertigkeit oder Sportlichkeit) würfelt. Seine Verteidigung ist das Ergebnis der Probe. Bei Gleichstand wird der Schaden halbiert und abgerundet bevor er gegen die Rüstung/Schilde/Lebenspunkte angerechnet wird.
\item Fliessende Bewegung: Der Charakter hat eine zusätzliche reaktive Handlung pro Runde. Er kann 1 FP einsetzen um eine weitere reaktive Handlung zu generieren. Die Parade kann auch für Geschosse durchgeführt werden die durch die eigene Kontrollzone fliegen.
\end{itemize}

\paragraph{Wolken}

\begin{itemize}
\item Großer Sprung (1FP): Für die Szene ist die Sprungreichweite verdreifacht. Der Charakter nimmt nur halben Schaden durch Stürze. Fallhöhen ohne Schaden zu erleiden sind verdoppelt. Sprünge können Teil eines Sturmangriffs sein, der Schaden wird zusätzlich um 1 erhöht, der Mindestwurf für die Stolpernprobe steigt auf 20.
\item Doppelsprung (1 FP oder aktive Handlung): Der Charakter kann seinen Sprung fortsetzen sofern er sich abstossen kann. Der Doppelsprung kann jede Handlung fortgesetzt werden. Er kann so zwischen zwei Wänden in beliebige Höhe kommen.
\item Baumläufer (1FP): Das Gewicht des Charakters belastet den Untergrund nur mit einem Viertel seines normalen Gewichts. Er bekommt -4 auf alle Proben das Gleichgewicht zu halten. Proben fürs Balancieren zählen nicht für die Bestimmung eines etwaigen Malus für Multitasking (mehrere Proben gleichzeitig würfeln) und sind auch von solchen Mali nicht betroffen.
\end{itemize}

\subsubsection{Fernkampf}

Umgang mit Waffen die entfernten Gegnern schaden.

\paragraph{Fernkampfwaffenexperte}

\begin{itemize}
\item Defensiv: 10\% mehr Reichweite.
\item Offensiv: Kernschussreichweite wird verdoppelt.
\item Überwältigen: Fernkampfangriffe in Kernschussreichweite fügen zusätzlich 2 Punkte Schaden zu.
\end{itemize}

\paragraph{Scharfschütze}

\begin{itemize}
\item Weitschuss: Reichweite ist um 30\% erhöht.
\item Geübtes Auge: Mali zum Schiessen werden so berechnet als wenn der Schütze eine Kategorie dichter steht.
\item Gezielter Schuss: Eine Runde zielen (Verbraucht die komplette Handlung). Das Ziel verliert 1 Punkt Deckung. Der Schütze bekommt +2W und ignoriert bis zu 3Punkte Rüstung und macht 1 Punkt Extraschaden.
\end{itemize}

\paragraph{Schnellschütze}

\begin{itemize}
\item Schnellladen: Beim Nachladen ist eine zusätzliche Bewegungsaktion möglich.
\item Pfeilregen (Manöver):Angriff auf drei unterschiedliche Ziele mit je -2W.
\item Konterschuss: Herausgezögerter Schuss gegen anderes Geschoss möglich (TEL Geschosse so Verteidigung 21, Pfeile 22, Bolzen 23). Nützlich um Magiegeschoss zu zerstören.
\end{itemize}

\paragraph{Trickschütze}

\begin{itemize}
\item Entwaffnen: Der Getroffenen erleidet nur halben Schaden, muss aber mit Geschick gegen den Angriffswurf bestehen oder er lässt seine Waffen fallen (oder einen anderen gehaltenen Gegenstand)
\item Schwachstelle: -2W dafür 3Punkte Rüstung ignorieren
\item Fallschuss: Der Angegriffene stürzt zu Boden ausser ihm gelingt eine Geschickprobe gegen den Agriffswurf.
\end{itemize}

\subsubsection{Führung}

Wege die den Umgang mit kleinen Gruppen von Kämpfern beschreibt. Es kann sofern nicht anders angegeben immer nur eine
Formation ausgeführt werden. Das Ausführen einer Formation verhindert meist auch das der Anführer ein Manöver
durchführt (Spielleiterentscheidung)

\paragraph{Anführen}

\begin{itemize}
\item Kampfbefehl: Der Charakter kann für Proben von Untergebenen seinen Fokus einsetzen. Er bekommt ausserdem einen Fokuspunkt den er nur für Untergebene verwenden darf.
\item Zusammenhalt: Der Charakter kann Proben von Untergebenen mit seinem Glück unterstützen. Er bekommt ausserdem einen zusätzlichen Glückspunkt den er nur für Proben von Untergebenen verwenden kann.
\item Anführer (kostet 1FP): Anstatt das jeder Untergebener die Probe würfelt wird für die gesamte Truppe das Ergebnis des anführenden Charakters verwendet. Dieser würfelt seine Probe aber mit dem Attribut Persönlichkeit. Insbesondere ist hier auch die Zustimmung des Spielleiters Vorraussetzung! Der Charakter bekommt einen zusätzlichen Fokuspunkt der nur für Untergebene verwendet werden kann (dazu gehört auch der Einsatz dieser Technik)
\end{itemize}

\paragraph{Angriffsformation}

\begin{itemize}
\item gemeinsamer Sturm (Formation): Waffenvorteile (z.B. durch Waffenvergleichswerten) werden für jedes Mitglied der Truppe so gewertet als wenn die anderen Mitglieder der Truppe bereits ihren Angriff durchgeführt haben.
\item Durchbrechen (Formation): Der Anführer würfelt eine Probe auf Persönlichkeit und Anführen/Führung. Der Mindestwurf ist 18. Gewinnt der Anführer haben die Untergebenen eine Mindestwurferleichterung von 1 bei Angriffen. Sollte die gegnerische Formation in Auffangformation sein, so ist der Mindestwurf die Anführenprobe des feindlichen Anführers. Kann mit gemeinsamer Sturm kombiniert werden.
\item Schwachstelle: Bei Angriffen auf dasselbe Ziel bekommt jeder Charakter der Truppe plus 1 Schaden für jeden weiteren Angreifer aus der Truppe (5 greifen dasselbe Ziel an, jeder bekommt +4 Schaden).
\end{itemize}

\paragraph{Fliegerformation}

\begin{itemize}
\item Sturmlandung (Formation/Landung): Mitglieder der Formation können nur durch den Gelegenheitsangriff des Angegriffenen getroffen werden. Ihre Angriffe machen automatisch Niederschlagsproben wie im Weg der Stärke beschrieben.
\item Wolkenformation (Formation): Im Luftkampf weisst der Anführer jeweils einen Verteidiger den Angriffen von aussen zu. Insbesondere kann er so Mitglieder im inneren der Wolke schützen. Er kann für jeden so Angegriffenen Unterstützer mitschicken (die er dann nicht für weitere Angriffe zur Verfügung hat) Diese erhöhen die Verteidigung des Angegriffenen um 1. Kann mit Techniken und Formationen aus dem Weg der Verteidigungsformation kombiniert werden.
\item Schwarmangriff (Formation 1FP): Der Anführer verteilt die Angreifer auf ein oder beliebig viele Ziele. Nach den Angriffen fliegen die Charaktere weiterhin geordnet. Kann mit Techniken und Formationen aus dem Weg der Angriffsformation kombiniert werden.
\end{itemize}

\paragraph{Schützenformation}

\begin{itemize}
\item Abfangfeuer (Formation/Herauszögern): Die Charaktere der Truppe feuern ihre bereit gemachten Fernkampfwaffen auf angreifende Feinde bevor diese den Nahkampf erreichen. Sie fügen pro Treffer so +2 Schaden zu. Sie wechseln danach automatisch auf ihre Nahkampfwaffen
\item Indirektes Feuer (Formation): Wenn es die Bewaffnung oder Situation zulässt (Schleuder, Bögen, oder nur vollständige Sichtdeckung) reicht es wenn ein Mitglied der Truppe die zu bekämpfenden Feinde sieht umd diese anzugreifen. Alle indirekt Feuernden erleiden einen Malus von 2 auf ihr Ergebnis.
\item Geschosshagel (Formation/1FP): Der Anführer macht eine Persönlichkeit plus Anführen/Führungsprobe) und legt ein grob kreisförmige Gebiet fest das eine Anzahl von Feldern in der Grösse seiner Truppe entspricht. Jeder innerhalb des Zielbereichs wird von zwei Geschossen getroffen mit der Anführenprobe als Ergebnis.
\end{itemize}

\paragraph{Taktik}

\begin{itemize}
\item eingespieltes Team: Einzelne in der Truppe können nicht mehr überrascht werden. Eine Bedrohung die ein Mitglied der Truppe erfasst ist sofort allen klar. Wahrnehmungsproben in Gruppen von 3 oder mehr bekommen einen Bonus von 1W.
\item Spezialisten: Spezialisten im Trupp sind bekannt und es wird vernünftige Zuarbeit geleistet. Falls ein Spezialist vorhanden bekommt er einen Bonus von 1W. Würfelt stattdessen der Anführer eine Probe und ein entsprechender Spezialist ist im Team so kann er mit Zustimmung des Spielleiters einen Bonus von 2W bekommen.
\item Formieren (Formation): Mitglieder der Truppe haben eine um 2 erhöhte Bewegung diese Runde. Gegen Gelegenheitsangriffe diese Runde steigt die Verteidigung um 2. Kann mit anderen Formationen kombiniert werden.
\end{itemize}

\paragraph{Verteidigungsformation}

\begin{itemize}
\item Standhaft (Formation): Die Formation kann Angriffe herauszögern und der Waffenvorteil wird so berechnet als wenn jedes Mitglied der Formation eine Waffe der nächstbesten Kategorie führen würde (+1 Waffenwert in der Verteidigung)
\item Auffangen (Formation): Der Anfühere der Formation würfelt eine Probe auf Persönlichkeit und Anführen, der Mindestwurf ist 18. Die verteidigende Formation bekommt +1 Verteidigung. Kann mit Standhaft kombiniert werden. Sollte eine gegnerische Formation mit dem Manöver durchbrechen angreifen ist der Mindestwurf die Anführenprobe der feindlichen Formation.
\item Keiner bleibt zurück: Jedes Mitglied der Truppe kann pro Runde einen Agriff den ein benachbartes Mitglied der Truppe erleiden würde auf sich lenken. (NSCs machen das automatisch wenn das Leben eines anderen in Gefahr ist).
\end{itemize}

\subsubsection{Gleiterflug}

Wege um das Bewegen und Kämpfen in der Luft zu verbessern.

\paragraph{Bomber}

\begin{itemize}
\item Zielwurf: Das Werfen von Wurfwaffen auf Ziele unter dem Gleiter ist nicht länger mit einem Malus belegt
\item Fernkämpfer: Der Malus des Nutzen von Fernkampfwaffen im Gleiter ist halbiert (abgerundet).
\item Magier: Zaubersprüche können ohne Malus gesprochen werden.
\end{itemize}

\paragraph{Kampf}

\begin{itemize}
\item Rolle: Der Charakter ist in der Lage sich in Rückenlage zu begeben und zu handeln. Er verliert danach aber 2 Höhenstufen.
\item Sturzflug: Für jeden Punk Höhenverlust bekommt der Charakter +1 Stärke (für Angriffe und ähnliches). Es kann so höchsten 4 Höhenstufen überbrückt werden. Ausserdem muss eine Fliegenprobe gegen 12 + doppelten Verlust an Höhenstufen bestanden werden.
\item Flügelschlag: Der Kämpfer irritiert mit seinen Flügel den Gegner und bekommt dafür 1W für Angriffsproben solange er seinen Gleiter trägt (auch am Boden mit offenen Flügeln).
\end{itemize}

\paragraph{Kontrolle}

\begin{itemize}
\item Geschwindigkeit: Aufprallwucht kann um 2 zusätzliche Stufen geändert werden (nach Wahl des Charakters).
\item Ausweichen (1FP): Kann reaktiv verwendet werden, die Verteidigung ist gegen einen Angriff um 5 erhöht oder Flächenschaden wird gedrittelt.
\item Kurven (1FP): Die Verteidigung ist für diesen Luftkampf um 1 erhöht.
\end{itemize}

\paragraph{Kunstflug}

\begin{itemize}
\item Geschickt: Irgendwie kommt der Charakter durch/um/über Hindernisse die ihn aufhalten sollten. Bis zu 4 Punkte Malus durch Hindernisse werden neutralisiert.
\item Leicht: Höhenverlust durch Manöver ist halbiert.
\item Wendig: Manöverklasse ist um 1 besser, Minimalgeschwindigkeit ist um 1 reduziert.
\end{itemize}

\paragraph{Langstreckenflug}

\begin{itemize}
\item Thermik: Der Charakter kann thermische Strömungen erkennen und nutzen.
\item Flügelschlag: Der Charakter kann aktiv mit den Gleiterflügeln schlagen um Höhe zu gewinnen. Dazu macht er eine Stärke plus Konstitutionsprobe gegen 16 ( +1 für jede vorausgegangene Probe). Seine Höhe steigt um 1 plus 1 pro Erfolgsgrad.
\item Ausdauer: Der Charakter erschöpft nur halb so schnell im Gleiter. Bei geraden Flug ist die Sinkrate halbiert.
\end{itemize}

\subsubsection{Kristallisierer}

Wege die denn Kristallisierer den Nebel und die aus ihm gewonnenen Splitter und Kristalle besser verstehen lässt.

\paragraph{Kristallbeherschung}

\begin{itemize}
\item mächtiger Kristall: Verzauberungen die einen Kristall verwenden bekommen +1 Willenskraft
\item Zwilllingskristall: Es können zwei Kristalle beim Erschaffen verwendet werden. Ihre Stufen addieren sich.
\item eingestimmter Kristall: Eine Verzauberung deren Auslösekosten nur durch Kristall bezahlt wird reduziert die Kosten um 1 (also 6 Auslösekosten können durch 5er Kristall getragen werden). Alternativ können genauso die kompletten Kosten der aufrechterhaltung getragen werden. (Auslösekosten 9 → 3/Runde, mit einem 2er Kristall Auslösen für 7 Energie, aber die 3 pro Runde werden vom 2er Kristall übernommen).
\end{itemize}

\paragraph{Magiefluß}

\begin{itemize}
\item Umlenken: Energiespeicher aus einer Verzauberung können im Verhältnis 2:1 in einen anderen umgelenkt werden.
\item alternative Energie: Energiespeicher einer Verzauberung können ein Auflademechanismus haben. Sie laden 1-3 Punkte pro Stunde auf dem sie einem Element ausgesetzt werden (z.B. in einem Schmiedefeuer oder im Sonnenlicht). Weitere Aufladungsregeln können mit dem Spielleiter abgesprochen werden (z.B 10Pkte beim Blitztreffer, OBSEK BET CENSA <>). Schwierigkeit der Verzauberung steigt um 3.
\item Energieakkumulation: Energiespeicher einer Verzauberung laden um 1Pkt pro Stunde wieder auf (kann mit alternativer Energie kombiniert werden oder mehrfach verwendet werden). Diese Technik lässt sich nur bei Gegenständen mit Kristallen einsetzen, die eine positive Stufe haben dabei steigt Schwierigkeit um 1 und die effektive Kristallstufe sinkt um 1.
\end{itemize}

\paragraph{Nebeladept (Selbst)}

\begin{itemize}
\item Erschaffen: Für 1Pkt geistigen Sternschaden kann 2W4 Splitter erschaffen werden.
\item Splitterabsorbtion: Der Adept zählt als 10Pkt Energiespeicher (mit Splittern aufladen) und kann mit dieser Energie die Aktivierungskosten eines berührten Artefakts zahlen (oder Teile davon).
\item Kristalltransfer: Energie eines ungebundenen Kristalls kann im Verhältnis 2:1 zur Aktivierung eines Artefakts verwendet werden.
\end{itemize}

\paragraph{Nebelläufer}

\begin{itemize}
\item Resistent: 2 Extrawürfel für alle Widerstandsproben gegen Effekte des Nebels oder Nebelgeister
\item Schwachstelle: Kreaturen des Nebels erleiden 1Pkt Extraschaden durch Angriffe des Nebelläufers
\item Kreaturenjäger: Kreaturen des Nebels verlieren 2Pkte Verteidigung gegen Angriffe des Jägers
\end{itemize}

\paragraph{Sammler}

\begin{itemize}
\item Manawirbel: Der Kristallisierer findet bessere Orte im Nebel. Die effektive Nebelstärke steigt um 1.
\item cleveres Sammeln: Die Ausbeute ist die ersten 10 Tage nicht reduziert (bessere Formgebung und Aufhängung)
\item Nebelsinn: Kann grob die Stärke des Nebels erkennen, Mit Wa+Kristallisieren auch genauere Informationen möglich. (z.B entdecken von Türmen in der Nähe, wirkende Verzauberungen oder verborgene Nebelkreaturen). Der Sammler erhält 2W beim Orientieren/Schleichen/Wahrnehmen im Nebel. Größere Wahrnehmungsreichweite.
\end{itemize}

\subsubsection{Magie}

Wege die die freie Runenmagie verbessern.

\paragraph{Bannmagiers}

\begin{itemize}
\item Flussumkehr: Die Komplexität der AN Rune sinkt um 1
\item Meister der Schutzkreise: Schildzauber und Kreise halten 20\% mehr Schaden ab.
\item Energiekontrolle: Gegenzauber, die einen existierenden Zauber auflösen, kosten nur die Hälfte der Energie.
\end{itemize}

\paragraph{Heiler}

\begin{itemize}
\item Heiler: Heilzauber heilen 1 Punkt mehr. +1W bei Krankheit oder Gift heilen.
\item Starke Stärkung: Stärkungszauber erhöhen das betroffene Attribut um 1 Punkt zusätzlich.
\item Regeneration: Mittels Ritualmagie können zerstörte Körperteile wieder hergestellt werden (CENSA MAGIE o.ä.).
\end{itemize}

\paragraph{Körperadept}

\begin{itemize}
\item Übertriebene Gesten: Weit ausholende Bewegungen +1W
\item Ritualer Tanz: Zaubernder bewegt sich und setzt seinen ganzen Körper ein -2 Schwierigkeit
\item Selbstfokus: Ein Punkt Magiekosten können vom Körperadept bezahlt werden ohne das er einen Punkt Schaden auf dem Monitor bekommt. (Die magischen Energiekosten pro Runde sind um 1 gesenkt und können trotz aktiver oder gesprochener Zauber auf Null fallen)
\end{itemize}

\paragraph{Ritual}

Rituale werden in magischen Kreisen gewirkt. Ob diese Kreise besondere Beschaffenheit haben regelt das Setting.

\begin{itemize}
\item Ritualzauber: Zauber der als Ritual gesprochen wird (1 Minute pro Komplexität des Zaubers) bekommt -2 Schwierigkeit
\item Ritualvorbereitung: Verdreifacht die Zeit fürs Ritual -2 Schwierigkeit.
\item Eingeschwungener Kreis: Magischer Kreis wird beim Zaubern verbraucht. Er muss speziell für ein Ritual vorbereitet werden. Vorbereitungszeit entspricht Ritualzeit -2 Schwierigkeit
\end{itemize}

\paragraph{Wort}

\begin{itemize}
\item Lautes Zaubern: Laut gesprochene Runenwörter geben dem Zaubernden +1W.
\item Schnelles Sprechen: Die Anzahl der gesprochenen Worte pro Runde steigt auf 8.
\item Deutliches Sprechen: Worte pro Runde sind halbiert -2 Schwierigkeit für Zauber.
\end{itemize}

\paragraph{Zauberspezialisierung}

Der Weg kann mehrmals erlernt werden. Er bezieht sich auf einen festgelegten Zauber. Dabei ist nur die Form und
Farbe festgelegt, die Effektstärke kann spontan und beliebig gewählt werden.

\begin{itemize}
\item Spezialist: Die Komplexität des Zaubers sinkt um 1.
\item Meister der Form: Es dürfen Worte die die Größe oder Energiekosten veränder in den Zauber eingefügt werden (z.B BET oder AROL).
\item Perfektion: Die Komlexität des Zaubers sinkt ein weiteres Mal um 1. Zauberbeginn und Zauberendworte dürfen hinzugegfügt oder entfernt werden.
\end{itemize}

\paragraph{verdeckte Zauber}

\begin{itemize}
\item Gestenloses Zaubern: Kein Abzug statt 1W Abzug
\item Stilles Zaubern: Statt 2W nur ein 1W Abzug
\item Meister der Illusion: Illusionen wirken als wenn die Farbbeschreibung eine Stufe besser ist.
\end{itemize}

\subsubsection{Nahkampf}

Wege für den Umgang mit den unterschiedlichen Waffen. Sofern nicht anders angegeben gelten die Boni und Mali nur für
eine Runde. Der Spielleiter kann sein Veto einlegen falls Techniken eingesetzt werden die nicht zusammen oder in der
aktuellen Situation eingesetzt werden können.

\paragraph{Fechter}

\begin{itemize}
\item Beinarbeit: Die Geschwindigkeit des Charakters steigt um 1.
\item Finte (Manöver): Es wird kein Angriff ausgeführt. +2 Verteidigung. Nächste Runde hat der Fintierte -2 Verteidigung gegen Angriffe des Kämpfers.
\item Schneller Konter (1FP): Einsatz möglich gegen einen Gegner der den Fechter gerade im Nahkampf verfehlt hat. Es ist sofort ein Nahkampfangriff gegen diesen Gegner möglich. Das verbraucht keine Handlung.
\end{itemize}

\paragraph{Festung}

\begin{itemize}
\item Rüstungsgewöhnt (passiv 1FP): Malus durch Behinderung ist für den Kampf oder die Szene um ein reduziert. Kann mehrmals verwendet werden um mehr als einen Punkt Behinderung zu reduzieren.
\item Unbeweglich (passiv): Rüstungsschutz kann als Bonus für Attributsproben verwendet werden wenn festgestellt werden muss ob er niedergeworfen wird (oder weggestoßen oder ähnliches).
\item Festungswall (1FP passiv), Der Kämpfer bekommt einen Schild der 3 mal Rüstung Schadenspunkte abhält. Dieser Schild ist durch die Rüstung geschützt. Nur einmal pro Kampf möglich.
\end{itemize}

\paragraph{Geschick}

\begin{itemize}
\item Klingenwirbel (Manöver): -2 Schaden dafür +1Verteidigung. Gelegenheitsangriffe machen aber vollen Schaden.
\item Überraschendes Manöver (Manöver): Einmal pro Gegner und Kampf möglich. Der Angriff wird mit Sportlichkeit gewürfelt. Verteidigung ist um 2 erhöht und der Schaden für den Angriff um 1.
\item Lücke suchen (Manöver): Ignoriert Kontrollzonen. Kann mit anderen Manövern kombiniert werden. Darf nicht in zwei aufeinanderfolgenden Runden verwendet werden.
\end{itemize}

\paragraph{Geschwindigkeit}

\begin{itemize}
\item Meidbewegungen 1FP: Der Kämpfer erhöht seine Verteidigung um 2 für diese Runde.
\item Blitzgeschwind (Passiv): Bewegung kann auch nach dem Angriff genutzt werden.
\item Katzengleich 1FP: Verteidigung ist für den gesamten Kampf um 1 erhöht.
\end{itemize}

\paragraph{Kämpfer}

\begin{itemize}
\item Waffenkenntnis: Der Kämpfer darf beliebige viele Waffenspezialisierungen lernen.Waffenspezialiserungen kosten die Hälfte an Erfahrungspunken.
\item Kampfübersicht: Der Kämpfer hat 1 zusätzliche reakitve Handlung pro Runde.
\item Kampfkontrolle: Die Reichweite der Kontrollzone erweitert sich um 1 Feld bis zum ersten Gelegenheitsangriff. Ist für diesen der Schritt ins Nachbarfeld nötig so wird er durchgeführt. Für den Rest der Runde ist die Kontrollzone normal. (Anders ausgedrück der Charakter kann ein Feld gehen um einen Gelegenheitsangriff durchzuführen oder dazu provoziert zu werden).
\end{itemize}

\paragraph{Lähmung }
kann auch im Fernkampf in Kernschußreichweite verwendet werden

\begin{itemize}
\item Schmerzhafter Treffer (Manöver): Der Angriff richtet nur halben Schaden an. Allerdings halbiert sich die Laufgeschwindigkeit des Getroffenen für 3 Runden.
\item Beeinträchtigender Schlag (Manöver): Der Angriff richtet keinen Schaden an senkt aber Geschick, Stärke und Wahnehmung um 2 Punkte (regeneriert sich pro Runde um je einen Punkt).
\item Betäuben (Manöver): wie Beeinträchtigender Schlag reduziert aber Will und Int. Auserdem muss der Getroffene ein Geschickprobe gegen das Angriffsergebnis erzielen oder er stürzt.
\end{itemize}

\paragraph{Nahkampfwaffenexperte}

\begin{itemize}
\item Defensiv: Waffenwert in der Verteidigung steigt um 1 bei Nahkampfwaffen.
\item Offensiv: Waffenwert steigt um 1 im Angriff.
\item Überwältigen: Nahkampfangriffe fügen zusätzlich 2 Punkte Schaden zu.
\end{itemize}

\paragraph{Schild}

\begin{itemize}
\item Schildnutzung: Das Nutzen eines Schildes bringt 1 Punkt Verteidigung.
\item Schildmanöver: Verwenden eines Schildes bringt 1 Punkt Waffenwert.
\item Schildexperte: Benutzen der Charakter ein ausreichend großen Schildes so bringt dieses zusätzlich 1 Punkte Verteidigung.
\end{itemize}

\paragraph{Stärke}


\begin{itemize}
\item Rüstung brechen: Nahkampfangriffe ignoriert ein Punkt Rüstung. Statt das Ziel anzugreifen kann versucht werden das Schild zu zerschlagen. (Schaden gegen Schildlebenspunkte ein schlechtes Holzschild  hat z.B nur 5 ein gutes Metallschild hat 20)
\item Überrennen (Manöver): Gelingt der Angriff so kann der Angreifende den Gegner 1 Feld nach hinten stoßen (der Gestossene provoziert keine Gelegenheitsangriffe) oder umstoßen. Dazu legt er eine vergleichend Probe mit Stärke gegen seinen Gegner ab (der hat die Wahl ob er sich mit Stärke oder Geschick verteidigt).
\item Zerschmettern (Manöver): Ignoriert 3 Punkte Rüstung. Kann mit mächtigen Angriff kombiniert werden und macht dann zusätzlich 2 Punkte Schaden.
\end{itemize}

\paragraph{Verteidiger}

\begin{itemize}
\item Fordern: Der angegriffene Gegner provoziert einen Gegenschlag wenn er seinen Angriff nicht gegen den Kämpfer richtet.
\item Unterdrücken (Manöver): Der Angegriffene hat diese Runde keine Kontrollzone.
\item Beschützen (Manöver): Alle Nahkampfangriffe gegen das zu beschützende Ziel richten sich stattdessen gegen den Kämpfer. -1 Verteidigung für den Kämpfer. Der Kämpfer und das Ziel das er beschützt müssen beieinander stehen.
\end{itemize}

\paragraph{Verwirrung}

\begin{itemize}
\item Ablenken (Manöver): Der Angriff richtet nur halben Schaden an allerdings verliert der Gegner für alle Kampf und Magieproben 3W im Falle eines Treffers.
\item Verwirren: Einmal pro Gegner und Kampf möglich. Es wird kein Angriff durchgeführt aber der Gegner verliert seine nächste Handlung.
\item Exponieren (Manöver): Trifft der Angriff so provoziert der Getroffene einen Gelegenheitsschlag bei Verbündeten des ausführenden Charakters (einmal pro Gegner und Kampf).
\end{itemize}

\paragraph{verdeckte Klinge}

\begin{itemize}
\item Verdeckter Schlag (Manöver): Der Angriff wird mit -2W gemacht. Dafür wird das Ergebniss um 1 punkt verbessert. Kann mit anderen Manövern kombiniert werden.
\item Schwachstelle (Manöver): -2Würfel. Ignoriert 3 Punkte Rüstung.
\item Meister des Hinterhalts: Hinterhältige Angriffe richten zusätzlich 5 Punkte Schaden an (vor der Verdopplung).
\end{itemize}

\subsubsection{Reiterei}

Wege die den Umgang mit und das Verhältnis zu Reittieren verbessert. Diese Weg können sofern passend auch zu anderen
vertrauten Tieren erworben werden.

\paragraph{Berittener Bogenschütze}

\begin{itemize}
\item Übung: Malus für Fernkampfangriffe vom Rücken eines Reittieres sind halbiert (abgerundet).
\item Ausweichen: Falls sowohl der Charakter als auch sein Reittier sich nicht im Nahkampf befinden bekommen beide +1 Verteidigung. Reiter und Reittier müssen in unmittelbaren Kontakt miteinander sein.
\item Meisterschütze: Fernkampfangriffe können irgendwann auf der Bewegungsstrecke des Reittiers durchgeführt werden. Der Malus für das Feuern vom Reittier entfällt.
\end{itemize}

\paragraph{Nahkampf}

\begin{itemize}
\item Sturmangriff: Geschick und Tiere führen(18), Der Reiter bekommt den Vorteil des Sturmangriffs, seine Stärke wird um die Hälfte des Reittiers erhöht.
\item Niederreiten: Das Reittier kann den gesamten Weg der Stärke im Nahkampf verwenden.
\item Durchbrechen: Ist der Nahkampfangriff des Reiters erfolgreich so entfällt eine mögliche reaktive oder aufgeschobene aktive Handlung des Getroffenen.
\end{itemize}

\paragraph{Verbundenheit}

\begin{itemize}
\item gemeinsames Ziel: Der Charakter kann mit Fokus die Proben des Reittiers unterstützen.
\item gemeinsames Schicksal: Der Charakter kann seine Glückspunkte auch für das Reittier verwenden.
\item eigene Geschichte: Das Reittier hat 2 Fokuspunkte und einen Glückspunkt zur Verwendung.
\end{itemize}

\paragraph{Vertrautheit}

\begin{itemize}
\item Tricks: Das Reittier versteht besser Befehle, für einfache Aufgaben entfällt die Probe auf Tiere führen. Komplizierte Befehle werden besser verstanden (-2 auf den Mindestwurf)
\item Moral: Bei Proben ob das Reittier sich widersetzt (Furcht, Schmerz, etc) bekommt das Reittier 2W auf seine Moralprobe.
\item Empathie: Reittier und Charakter verstehen sich intuitiv. Befehle können nonverbal und fast ohne Gesten gegeben werden. Gemütszustand wird dem Charakter klar und das Tier beteiligt sich aktiv an Wahrnehmungsproben.
\end{itemize}

\subsubsection{Steckenmagier}

Wege um die mit Zauberstecken und Runenringen hervorgerufene Magie zu verbessern. Diese Wege übernehmen einige
Techniken aus anderen arkanen Wegen. Gleiche oder ähnliche Techniken sollte nicht kombiniert werden.

\paragraph{Kampfmagier}

\begin{itemize}
\item Routine: Die Energiekosten aller schadenverursachenden Zauber sind um 1 reduziert.
\item Nachdruck: Die Willenskraft kommt auch bei Flächenschaden als Bonus dazu.
\item Vielfältigkeit: Fernkampfwege können mit Geschoßzauber verwendet werden.
\end{itemize}

\paragraph{Kristallmagier}

\begin{itemize}
\item Gesprochen: Zauber werden laut und mit Nachdruck gesprochen, dafür gibt es einen zusätzlichen +1W.
\item Fokus: In der anderen Hand kann ein Kristall gehalten werden. Dieser reduziert die die Energiekosten in Höhe seiner Stufe.
\item Ursprung: Nebel kann als Verbrauchsfokus verwendet werden. Effektiv übernimmt der Nebel Kosten in Höhe seiner Stufe. Dabei ist aber zu beachten das ein Charakter ein Fokuslimit in Höhe der Hälfte seine Willenskraft plus Intelligenz pro Runde hat.
\end{itemize}

\paragraph{Runenmagier}

\begin{itemize}
\item Zeichen: Runen werden mit Stecken in die Luft gemalt, das gibt einen Bonus von +1W.
\item Ritual: Die Zauberdauer ist jetzt Komplexität Minuten, allerdings sinkt die Schwierigkeit um 2.
\item Runenwand: Der Zaubernde bewegt sich 3 Felder weit (das ist seine Bewegungshandlung), dafür fällt die Komplexität des Zaubers um 2.
\end{itemize}

\paragraph{Schnellmagier}

\begin{itemize}
\item Positionieren: Der Magier kann bis zu 6 Runen tauschen und trotzdem seine halbe Bewegung nutzen.
\item Flexibiltät: Der Magier kann Auswahlrunen nutzen und bis zu zwei in einer freien Handlung ändern.
\item Nahkämpfer: Schild und Nahkampfkampfzauber können verwendet werden ohne einen Gelegenheitsangriff zu provozieren.
\end{itemize}

\paragraph{Verzauberer}

\begin{itemize}
\item Routine: Die Komplexität der KONFAR - Rune ist um 1 reduziert.
\item Starke Waffen: Schadensverzauberungen von Waffen bekommen die Willenskraft als Bonus, analog zu Schadenszaubern.
\item Stärke der Elemente: Attributstverzauberung erhalten eine zusätzlich Effektstärke.
\end{itemize}

\subsubsection{Verzauberung}

Wege die genutzt werden können um verzauberte Gegenstände besser zu ''designen'', ''herzustellen'' oder zu ''nutzen''.

\paragraph{Artefaktschmied (Erschaffen)}

\begin{itemize}
\item langsames Arbeiten: verdoppelt die Zeit beim Erschaffen, senkt die Schwierigkeit beim Erschaffen um 2.
\item kostbare Materialien: seltene und teure Materialien können folgende Effekte habe (nur ein Material gilt). Silber eröht die Willenskraft des Gegenstands ist um 2. Gold vergrößert Energiespeicher um 40\%. Platin erhöht die Effektstärke ist um 1.
\item Sicheres Arbeiten: Kristalle und zu verzaubernder Gegenstand werden nicht mehr beschädigt beim Fehlschlag (Zeit und Testsplitter sind trotzdem weg).
\end{itemize}

\paragraph{Elementarverzauberer}

\begin{itemize}
\item Routine: Die Komplexität der KONFAR - Rune ist um 1 reduziert.
\item Starke Waffen: Schadensverzauberungen von Waffen bekommen die Willenskraft als Bonus, analog zu Schadenszaubern.
\item Stärke der Elemente: Attributstverzauberung erhalten eine zusätzlich Effektstärke
\end{itemize}

\paragraph{Energie (Nutzen und Erschaffung)}

\begin{itemize}
\item Splitternutzer: Es kann pro Runde 3 zusätzliche Splitter in einen magischen Gegenstand geladen werden.
\item Speichernutzer: Erstellte Verzauberungen haben 20\% größere Speicher
\item effiziente Magie: Erstellte Verzauberung haben die Kosten um 1 reduziert
\end{itemize}

\paragraph{Geschosse (Nutzen)}

\begin{itemize}
\item Geschossexperte: +1W für magische Geschosse
\item Overcharge (einmal pro Kampf oder 1FP): Aktivierungskosten des Artefakts sind verdoppelt allerdings auch die Effektstärke und Willenskraft des Gegenstands.
\item Magic is my Weapon: erlernte Fernkampfwege können auch mit Geschossen aus Artefakten verwendet werden
\end{itemize}

\paragraph{Innere Macht (Selbst)}

\begin{itemize}
\item gesplittertes Selbst: Statt mit Splittern kann ein Gegenstand mit geistigen Kästchen aufgeladen werden (Limitierung auf 2 pro Runde bleibt).
\item Absorption (passiv, reaktiv): Wird der Charakter von einem für ihn sichtbaren magischen Effekt getroffen so kann er ihn für 1 FP neutralisieren im Moment des Eintreffens). Zusätzlich bekommt er eine Menge geistigen Schaden die den Kosten des Effekts entsprechen.
\item Down the Drain (einmal pro Kampf oder 1FP): Ein Gegenstand wird nur mit geistigen Kästchen aktiviert (kein Limit). Laden und Aktivieren geht in einer Handlung. Die Willenskraft des Nutzers wird zu der des Gegenstands für den Effekt addiert.
\end{itemize}

\paragraph{Krämer (Erschaffen)}

\begin{itemize}
\item Fliessbandarbeit: Schwierigkeit ist um 2 bei der Erschaffung erhöht. Dafür wird nur die Hälfte der Zeit gebraucht.
\item Verbrauchsgegenstände: Auslöser darf Beschädigung/Zerstören/Aufschlagen des Gegenstands sein. Solche Gegenstände ignorieren den größen Multiplikator wenn die Verzauberung exakt für diesen Gegenstand ist (z.B Explosionspfeil, Heiltrank, Lichtmünze, etc.)
\item günstige Verzauberung: Verzauberungen werden geladen erschaffen.
\end{itemize}

\paragraph{Macht (Erschaffen)}

\begin{itemize}
\item Wille: Schwierigkeit sinkt um 1.
\item Stärke: maximal mögliche Effektstärke um 1 erhöht.
\item Erkenntnis: Komplexität der Effekte sinkt beim Erstellen um 1
\end{itemize}

\paragraph{magischer Schild (Nutzen)}

\begin{itemize}
\item Schildnutzung: Ein aktiviertes magisches Schild gibt einen Waffenwert von +1 in der Verteidigung.
\item schnelle Aktivierung: Aktivieren eines getragenen Schildes kann als freie Handlung durchgeführt werden (kein Gelegenheitsangriffi mehr).
\item Aufflackern: Der Treffer der ein Schild zerstört verursacht keinen weiteren Schaden am Schildnutzer.
\end{itemize}

\paragraph{gebundene Waffe (Nutzen)}

\begin{itemize}
\item gebundene Waffe: Die ausgewählte Waffe (Waffenbindung dauert ca 1Tag) kann den sekundären Effekt ihre Verzauberung direkt und verstärkt nutzen (z.B. als helle Fackel). Die Verzauberung kann wenn nötig unterdrückt/inaktiv gemacht werden (und ohne weitere Kosten weiterlaufen).
\item Überladen: Die Waffe kann für eine Runde zusätzliche Effektstärke (Bonus W4 Schaden) bekommen (freie Handlung): Für 1 Kästchen geistigen Schaden gibt es einen Bonus von 1 auf die Effektstärke, für 3 Kästchen einen Bonus von 2 und für 6 Kästchen einen Bonus von 3.
\item Rufen: Für 10 Kästchen geistigen Schaden (aktive Handlung) erscheint die Waffe in der Hand des Nutzers.
\end{itemize}

\paragraph{Mechanik (Erschaffen)}

\begin{itemize}
\item mechanischer Auslöser: Magische Effekte können durch einfache mechanische Auslöser getriggert werden (z.B. Schalter oder Trittplatten).
\item Sensor: Wahrnehmende Effekte können Lichteffekte oder Nadeln beeinflussen. Das kann als mechanischer Auslöser verwendet werden (z.B. Thaumameter SCIEN CONDUC AUM). Nur Komplexität für die Wahrnehmung.
\item Wahlschalter: Durch mechanisches Auswählen kann das Farbwort verändert werden. Folgende Regeln gelten: 1.) Nur ähnliche Farben (Bogen der Feuerpfeile oder Eispfeile verschiesst) 2.) Komplexität muss nur für die größte Farbe bezahlt werden. Zusätzlich steigt die Komplexität aber um 1. 3.) Schwierigkeit steigt um die Anzahl der Auswahlmöglichkeiten (Im Beispiel sind das Feuer und Eis, d.h. +2 Schwierigkeit)
\end{itemize}

\paragraph{Multiverzauberung}

\begin{itemize}
\item zusätzliche Verzauberung: Ein Gegenstand kann mit weiteren Verzauberungen versehen werden. Die Schwierigkeit beim Erschaffen steigt um 1 Punkte pro vorhandener Verzauberung.
\item geteilter Speicher: Die Verzauberung nutzt den Energiespeicher einer vorhanden Verzauberung. Die Schwierigkeit beim Erschaffen steigt um 1 (Ja man kann einfach nur einen Energiespeicher der Größe 50 als erste Verzauberung erstellen…). Ein zusätzlicher eigener Speicher ist ebenfalls möglich.
\item geteilter Kristall: Die Verzauberung kann den Kristall einer anderen Verzauberung nutzen (oder falls diese mehrere hat dann alle). Ein Kristall liefert aber nur Energie in Höhe seiner Stufe pro Runde, falls mehrere Verzauberungen aktiv sind kann die Energie aller anderen Verzauberungen angepzapft werden. Die neue Verzauberung kann keinen Kristall enthalten! Schwierigkeit beim Erschaffen steigt um 1.
\end{itemize}

\paragraph{Schildverzauberer (Erschaffung)}

\begin{itemize}
\item starke Schilde: Schildverzauberungen haben 20\% mehr Schildpunkte
\item selbst heilende Schilde: Schild heilt Effektstärke Punkte pro Runde (AUM UR)
\item gepanzerte Schilde: Schaden der in das Schild geht wird um die Willenskraft reduziert (+3Komplexität, UR KONFAR)
\end{itemize}

\paragraph{Seelenschmied (Erschaffen, verborgener Weg)}

\begin{itemize}
\item dunkle Ahnung: Für eine um +2 erhöhte Schwierigkeit, kann der Gegenstand sehr einfache eigene Entscheidungen treffen (Aktivieren durch magische Wort, Aktivieren wenn Kreaturentyp in der Nähe ist).
\item dunkler Traum: Wahrnehmungszauber die am Gegenstand sind können von der dunkeln Ahnung interpretiert werden. Einfache Bedingungen können beschrieben werden. (Löse aus wenn du Menschen mit aktiver Magie wahrnimmnst, etc.)
\item dunkler Geist: Der Gegenstand hat ein sehr einfaches Bewusstsein (Wa3, Int2) und kann alle am Gegenstand angeschlossenen Kontroll und Wahrnehmungseffekte nutzen (so kann z.B. ein rudimentärer Golem erschaffen werden). Die Komplexität steigt um 5.
\end{itemize}

\paragraph{Signatur (Design)}

\begin{itemize}
\item eigener Stil: Für 1 Pkt erhöhte Schwierigkeit kann der Verzauberer seinen persönlichen Stil einfliessen lassen (Flammen oder Licht einfärben, zusätzliche Geräusche). Die Signatur wird damit festgelegt (also einmal „Wush“ beim grünen Feuerball dann ist der Stil damit festgelegt)
\item starker Stil: Das Nutzen eigener Verzauberungen mit eigenem Stil erhöht die Willenskraft des magischen Effekts um 1.
\item gemeisterter Stil: Die Energiekosten einer Verzauberung mit eigenem Stil ist um 1 Punkt günstiger. Zusätzliche Beschreibungseffekte sind möglich. (Der Feuerball hinterlässt schwarzen Rauch der z.B. Sichtbehinderung bietet)
\end{itemize}

\paragraph{Tätowierung (Selbst)}

\begin{itemize}
\item erste Tätowierungen: Der Körper des Charkter kann Verzauberung für 10 Punkte Komplexität aufnehmen. (können gut versteckt werden). Verzauberungen die tätowiert werden sind um 2Pkte schwieriger. 
\item komplexe Tätowierung: 10 weitere Punkte Komplexität möglich (bedecken fast den gesamten Körper)
\item Abschliessende Tätowierung (nur eins der Folgenden möglich): 1.) vollständig Tätowiert (auch Gesicht und Hände) +10 Komplexität möglich 2.) kunstvoll, Die Runen sind künstlerisch in Motiven verborgen 3.) Seelenbindung, die (natürliche) Willenskraft bietet permanente Energie wie ein Kristall für die tätowierten Verzauberungen.
\end{itemize}

\begin{center}
\section{Wege der Meister}
\end{center}

Die hier vorgestellten Wege erhöhen die Stärke der Charaktere massiv. Sie sollten nur mit Zustimmung des Spielleiters
erlernt werden und passend zum Charakter.

\begin{center}
\subsection{Erlernen eines Meisterweges}
\end{center}

\begin{itemize}
\item Das Erlernen der Meisterwege hat einen eigenen Pool von gezählten Wegen. Das heisst die initialen Kosten sind unabhängig von der Zahl der normalen Wege
\item Es gelten die selben Regeln für das Lernen und Steigern der Meisterwege wie für die normalen Wege. Allerdings sind die Kosten 5 mal so hoch
\end{itemize}

\begin{mdframed}[hidealllines=true, backgroundcolor=black!10]
\paragraph{Spielleiterhinweis}

Anstatt die Meisterwege über Erfahrungspunkte zugänglich zu machen empfiehlt sich diese als Belohnung für bestandene
Abenteuer zu geben. so hat der Spielleiter mehr Einfluss darauf wie mächtig die Charaktere werden.

\end{mdframed}
\begin{center}
\subsection{Meisterwege nach Kategorie}
\end{center}

\subsubsection{Alchemie}

\paragraph{Meister der Alchemie}

\begin{itemize}
\item Schlüssel der Welt: Kenntnis der Attribute aller Zutäte mit Ausnahme der Seltensten
\item Distille: Die Effektstärke hergestellter Substanzen kann um 1 erhöht werden, die Kritikalität steigt um 4
\item Stein der Weisen: Eine Zutat kann in ein Andere der gleichen Gruppe verwandelt werden (z.B. Eisen in Gold). Das dauert pro Zutat eine Stunde und erfordert einen Alchemietisch.
\end{itemize}

\subsubsection{Allgemein und Handwerk}

\paragraph{Meister der Konzentration}

\begin{itemize}
\item Fokus: Der Fokus des Charakters steigt um 2
\item Multifokus: Der Charakter kann einen weiteren Punkt Fokus pro Runde einsetzen. Der Fokus des Charakters ist um 1 erhöht.
\item Meditation: Der Charakter kann einen Punkt Fokus pro Stunde Ruhe regenerieren. Das Fokuslimit steigt ein weiters Mal um 1.
\end{itemize}

\paragraph{Meister des Handwerks}

Wird für jedes Handwerk einzeln gelernt. Muss für das Handwerk festgelegt werden

\begin{itemize}
\item Routine: 2W Bonus für alle Proben
\item Übung macht den Meister: Schwierigkeit aller Proben um 2 reduziert
\item Aussergewöhnlich: Bei der Herstellung kann der Meister ein aussergewöhnliches Werk erschaffen. Dieses zeichnet sich durch Einzigartigkeit aus. In Absprache mit dem Spielleiter können besondere Effekte oder andere Regeln gebrochen werden (Z.B.: Ein Schwert aus Gold das hart und scharf wie aus Stahl ist. Ein Garten in dem zu jeden wichtigen Feiertag, und auch nur dann, Blumen blühen.)
\end{itemize}

\paragraph{Meister des Überlebens}

\begin{itemize}
\item Durchhalten: Der Charakter benötigt in Extremsituation nur die Hälfte an Schlaf, Wasser und Nahrung. (Er holt das später nach). Erschöpfungsproben sind nur halb so oft fällig (und Schwierigkeiten steigen entsprechend nur halb so schnell)
\item Totgesagte leben länger: Die Wunden des Charakters zählen immer als behandelt. Er bekommt 3W für Widerstand gegen Krankheit und Gift. Statuseffekte wie Gift und Benommenheit zählen so als wären sie 3 Stufen weniger, wenn es um die Bestimmung der Auswirkung geht
\item Unsterblich: Der Charakter bekommt eine zusätzlich Schadenswiderstandsreihe vor der ersten. Das bedeutet es müssen erst 2 Reihen voll sein bevor es Mali gibt und er kann 6 Reihen Schaden aushalten. (Nur körperlicher Monitor)
\end{itemize}

\subsubsection{Chi}

Wege der östlichen Reiche. Sie fokussieren die innere Lebensenergie und erzeugen so quasi übernatürliche Effekte.

\paragraph{Meisterweg des Chi}

\begin{itemize}
\item Atemtechniken: Zum Meditieren wird nur noch eine Zeitspanne von 10min pro Fokuspunkt gebraucht. Das gilt auch für den Meisterweg des Fokus. Der Charakter hat einen zusätzlichen Fokuspunkt.
\item Meister der Heilung. Die Menge der Punkte der Chi-Heilung ist verdoppelt. Ausserdem kann körperlicher Schaden eines berührten Ziels auf den eigenen körperlichen Monitor transferiert werden. Das funktioniert wie die Chi-Heilung, es wird also ein 1FP ausgegben und ist auf Willenskraft plus Konstitution limitiert (Den Multiplikator von 2 aus dieser Technik nicht vergessen). Der Charakter hat einen zusätzlichen Fokuspunkt.
\item Das siebte Chakra: Der Charakter kann Lebensenergie um sich herum spüren. Er kann ebenfalls Emotionen erfassen. Für passive soziale Proben oder Wahrnehmungsproben erhält er 3W. Einige dieser Proben scheitern automatisch wenn sie gegen den Charakter gerichtet sind (z.B. Anschleichen wenn die Lebenskraft nicht irgendwie verborgen ist). Der Charakter hat einen zusätzlichen Fokuspunkt.
\end{itemize}

\paragraph{Erde}

\begin{itemize}
\item Golem (1FP): Der Charakter verwendet das Manöver Statue, allerdings entfallen die ersten 5 Punkte Bonusschaden und die Bonusrüstung beträgt 3. Die Beeinschränkug Manöver entfällt.
\item Marmorseele (1FP): Der Bonus des Einstiegs beträgt jetzt 3 Punkte für den restlichen Kampf.
\item Stampftritt (1FP, Bewegungshandlung): Feindliche Charaktere in bis zu 3 Feldern Entfernung müssen eine Stolpernprobe gegen die Konstitution plus Stärke des Stampfenden machen. Misslingt diese so fallen sie zu Boden.
\end{itemize}

\paragraph{Feuer}

\begin{itemize}
\item Inneres Feuer: Der Charakter ist in der Lage im Dunkelen zu sehen. Selbst mit verbundenen Augen zählt er nicht als blind, sein maximaler Malus durch Blendung und ähnliche Effekte ist 2W.
\item Chi-Feuer (1FP, Manöver): Trifft der Angriff so richtet er 6 Punkte Gift an. Kann mit Manövern die Statuseffekte oder Attributmali verursachen kombiniert werden.
\item Drache (1FP, Manöver): Verursacht einen Flächenangriff Feuer in einem Kegel mit 60 Grad Öffnungswinkel und 5 Feldern Reichweite. Die Effektstärke ist 3 (3W4 Schaden), zusätzlich wird der Schaden um die Willenskraft oder Konstitution erhöht. Für je 1 FP kann der Öffnungswinkel oder die Reichweite verdoppelt werden.
\end{itemize}

\paragraph{Holz}

\begin{itemize}
\item Wolf: Der Charakter bekommt einen Bonus in Höhe seines körperlichen Malus auf seine Stärke.
\item Tiger: Der Charakter bekommt einen Bonus in Höhe seines halben abgerundeten Malus auf seine Geschicklichkeit.
\item Körperkontrolle: Der Charakter regeniert 1 Punkt Kreuzschaden und 5 Strichschaden pro Runde. Der Charakter kann zu jeder Zeit beliebig viele Sschadenskästchen auf seinen körperlichen Monitor hinzufügen. Der Charakter bekommt für seinen körperlichen Monitor eine zusätzliche Reihe mit demMalus -5. Der Meister kann zusätzlich und gleichzeitig in den Kampfstil des Feuers wechseln.
\end{itemize}

\paragraph{Metall (Fliegende Schwerter)}

\begin{itemize}
\item Tanzende Schwerter: Für Angriffshandlung kann der Charakter eine auf den Boden liegende Klinge in Sichtweite verwenden. Diese Klinge blockiert das Feld bis zur nächsten Runde des Charakters und hat eine Kontrollzone.
\item Metallsturm (1 FP): Der Charakter macht Flächenschaden in einem Kegel mit 60 Grad Öffnungswinkel und 7 Feldern Reichweite. Der Charakter benötigt sieben Klingen für diese Technik die am Ende der Technik zurückkehren. Jedes Ziel wird möglicherweise zweimal ohne Erfolgsgrade getroffen (2 Würfe auf W8, 1-7 für jede verwendete Klinge, bei einer 8 entgeht das Ziel dem Angriff)
\item Klingenvorhang: Der Charakter hat zwei zusätzliche Angriffshandlung für die verwendung der tanzenden Schwerter. Unter Einsatz eines Fokuspunktes können diese Runde zwei weitere Schwerter bewegt werden oder alle bewegten Schwerter können eine Bewegungshandlung durchführen. Der Meister kann zusätzlich und gleichzeitig in den Kampfstil des Wassers wechseln.
\end{itemize}

\paragraph{Schatten}

\begin{itemize}
\item Schattensinn: Der Charakter kann im Dunkeln sehen. Allerdings sieht er eher Umrisse und keine Farben. Lesen ist so auch nicht möglich. Allerdings entfallen für Angriffe sämtliche Einschränkungen durch Sicht. Ist der Charakter blind so kann er sich wie ein Sehender orientieren, sein Wahrnehmungsattribut bleibt für sichbasierte Proben (z.B. Fernkampf) trotzdem reduziert.
\item Doppelgänger: Die Mimikry Technik beinhaltet nun auch das Nachahmen von Sprache und Mimik. Die Gesichtszüge passen sich denen des Ziels an. Mit der richtigen Kleidung ist er vom Ziel nicht unterscheidbar. Für 1 FP kann auch eine Fertigkeit oder Handlung imitiert werden, hier kommt aber ein Malus von 3W zum tragen.
\item Ninjutsu: Die Reichweite des Schattenschritts ist verdoppelt. Der Charakter kann auch ausserhalb eines Schattens einen Schattenschritt durchführen (ein Schatten der groß genug und in Reichweite ist vorrausgesetzt). Bei einem Schattenschritt kann der Charakter sofort eine Heimlichkeitsprobe ablegen um sich zu verbergen, das get auch im Kampf.
\end{itemize}

\paragraph{Wasser}

\begin{itemize}
\item leere Hand: Der Charakter verliert nie Verteidigung durch Waffenvorteil. Der Charakter kann sofern er unbewaffnet ist den Waffenschaden des getroffenen Ziels verwenden. Seine Verteidigung ist um 1 erhöht.
\item Weg der Faust und Tritt: Kämpft der Charakter mit maximal leichtem Schuhwerk so kann er zu jeder Angriffshandlung einen zusätzlichen Tritt durchführen. Tritte des Charakters machen Schaden wie Keulen und können mit Manövern des Wegs der Stärke kombiniert werden.
\item Jeet Kune Do: Der Charakter bekommt eine weitere reaktive Handlung. Gegner die den Charakter verfehlen provozieren einen Gelegenheitsangriff. Der Meister kann zusätzlich und gleichzeitig in den Kampfstil des Holz wechseln.
\end{itemize}

\paragraph{Wolken}

\begin{itemize}
\item In Bewegung: Der Charakter kann mit einer freien Handlung aufstehen. Alle Probe für das Bewegen (Sportlichkeit, etc.) zählen nicht für die Bestimmung eines etwaigen Malus für Multitasking (mehrere Proben gleichzeitig würfeln) und sind auch von solchen Mali nicht betroffen.
\item Windläufer: Handlungen die fürs Bewegen genutzt werden sind nicht verbraucht und stehen für andere Handlungen zur Verfügung.
\item Wolkenläufer: Sprünge nutzen die volle Bewegungsreichweite. Für 1FP ist der Doppelsprung ist auch ohne Untergrund möglich (gilt für den Rest der Szene). 
\end{itemize}

\subsubsection{Fernkampf}

\paragraph{Artillerist}

\begin{itemize}
\item Indirektes Feuer: Der Charakter kann indirekt auf bekannte Ziele ausserhalb seiner Sicht feuern. Das Geschoss muss den Weg aber auch zurücklegen können. Je nach dem wie die Flugbahn ist gibt es einen Malus von 1 bis 4W. Ausserdem gibt es kein Bonusschaden durch Erfolgsgrade oder hinterhältigen Angriff
\item Waffenkombination: Alchemistische Substanzen / Zaubersprüche / etc. werden an das Geschoss gebunden. Solche ausserhalb des Kampfes vorbereitete Geschosse transportieren ihre Ladung ins Ziel und lösen sie dort aus.
\item Durchschuss (1FP): Der Schaden wird um 5 erhöht. Ausserdem können bis zu 3 Ziele in gerader Linie mit dem Schützen getroffen werden. Ob und wie gut getroffen wird, wird mit normal mit der Verteidigung und dem Angriffswurf berechnet. Das zweite Ziel bekommt die Rüstung des Ersten als Bonus, das dritte Ziel die rüstung des Ersten und Zweiten.
\end{itemize}

\paragraph{Meisterschütze}

\begin{itemize}
\item kritische Treffer: Erfolgsgrade werden pro Punkt über dem Mindestwurf berechnet (statt 2)
\item Zielanalyse: Ergebnis des Angriffswurfs ist um 1 erhöht. Für 1FP kann das Ergebnis um weitere 2 Punkte erhöht werden. Fokuspunkt müssen vor dem Wurf ausgegeben werden.
\item Lähmender Schuss: Das Ziel bekommt den Statuseffekt ''gelähmt'' in Höhe des halben abgerundenten Schaden. Der Spielleiter kann bei großen und sehr großen Kreaturen die Effektstärke weiter reduzieren. Diese Fähigkeit macht mit einer stumpfen Waffe stattdessen den Effekt ''Benommen''.
\end{itemize}

\paragraph{Meister des Schnellfeuers}

\begin{itemize}
\item fliessende Bewegung: Eine Bewegungshandlung zum Nachladen entfällt.
\item Pfeilhagel (Manöver): Ein Gebiet mit dem Radius von 2 wird angegriffen. Pro Feld mit Ziel wird eine Angriffsprobe mit -3W durchgeführt.
\item Sperrfeuer (Manöver): Zwei Angriffe sind möglich wenn das Nachladen klappt. Für 1 FP ist ein weiterer Angriff möglich.
\end{itemize}

\subsubsection{Gleiterflug}

\paragraph{Flugmeister}

\begin{itemize}
\item Beweglichkeit: Manöverklasse ist um 1 besser. Höchstgeschwindigkeit steigt um 1.
\item Himmelsstürmer: Kein Höhenverlust ausserhalb von Manövern. Der Charakter kann sich bei sehr ruhiger Fluglage ausruhen. Ausruhzeiten sind aber verdoppelt.
\item Spezialkonstruktion: Der Charakter hat immer freie Hände im Gleiter, der Gleiter stört am Boden nicht und kann mit einer freien Handlung bereit gemacht werden. Der Charakter benötigt keinen Anlauf um zu starten. Gelegenheitsangriffe für Gleiterbewegung entfallen.
\end{itemize}

\paragraph{Gleiterkampfmeister}

\begin{itemize}
\item Durchsacken (reaktive Handlung): Der Charakter kann bis zu 4 Höhenstufen einbüßen. Ein erlittener Nahkampftreffer macht 3 Punkte weniger Schaden pro so verlorener Höhenstufe. Die so verlorenen Hohenstufen geben keine Geschwindigkeit.
\item Geschwindikeitskontrolle: Der Charakter macht 1 Punkt mehr Nahkampfschaden pro Geschwindigkeitsstufe zu seinen Gunsten. Gegner machen 1 Punkt weniger Schaden pro Punkt Geschwindigkeitsvorteil gegen diesen Charakter. Für 1FP kann die Geschwindigkeit angepasst werden (schneler oder langsamer, das ist auch reaktiv möglich und kostet keine Handlung).
\item Todesengel: Nahkampf und Fernkampfmanöver sind in Kombination mit Gleitermanöver möglich. Die Verteidigung des Charakter ist am Gleiter um 1 erhöht. Angriffe gegen nicht fliegende Ziele sind um 1 leichter.
\end{itemize}

\subsubsection{Kristallisierer}

\paragraph{Kristallmeister}

\begin{itemize}
\item Kristallschwingung: Verzauberungen die der Kristallmeister erstellt hat oder von ihm verwendet werden haben +1 Willenskraft. (Nicht kumulativ)
\item Kristallnutzung: Alle Kristalle die der Kristallmeister erstellt oder verwendet haben +1 Stufe. (Nicht kumulativ)
\item Kristallenergie: Erstellte Verzauberungen mit Speicher und Kristall können ihren Speicher jede Runde in Höhe der Kristallstufe aufladen. Ein Kristall der den Speicher laden soll kann nicht für die Kosten der Verzauberung verwendet werden.
\end{itemize}

\paragraph{Nebelmeister}

\begin{itemize}
\item Nebelabsortion, Der Nebelmeister hat 10 zusätzliche Punkte Energiespeicher (siehe Nebeladept). Sein Energiespeicher lädt sich automatisch nach einer Stunde im Nebel auf.
\item Nebelkontrolle, Der Adept kann um sich herum die Stärke des Nebels um bis zu drei Punkte anpassen (verstärken oder schwächen). Das reicht meist aus um kleine Zonen ohne Nebel zu erschaffen.
\item Nebelverschmelzung. Für alle Auseinandersetzung mit dem Nebel oder seinen Kreaturen und Effekten bekommt der Charakter zwei Extrapunkte Willenskraft. Das hilft bei allen Widerstandsproben, der Bonus gilt aber auch falls z.B. eine Nebelkreatur kontrolliert werden soll oder als Bonus für magischen Schaden.
\end{itemize}

\subsubsection{Magie}

\paragraph{Erzmagier }

\begin{itemize}
\item Meister der Namen: Der Erzmagier kann Farbworte mit Zustimmung des Spielleiters erschaffen. Als Richtlinie sollte hier Formworte mit CONDUC oder absteigender Beschreibung genutzt werden. Solche selbst erschaffende Runenworte sind in ihrer Komplexität um 1 niedriger als die entsprechende ursprüngliche Kombination.
\item Runenmeister: Komplexität aller Zauber um 2 reduziert
\item altes Wissen: Die Komplexität von Verstärkungsrunen wird halbiert (aufgerundet).
\end{itemize}

\paragraph{Festungsmagier}

\begin{itemize}
\item Schildecho: Schildezauber sind nicht beendet wenn ihre Schildpunkte auf 0 gefallen sind. Für 1 FP regeneriert ein Schild des Charakters 15 Schildpunkte (freie Handlung).
\item Farbe Schild: Schilde erleiden nur halben Schaden (abgerundet). Komplexität der UR Rune ist 0
\item Festung: Es können Effektstärke Personen von einem Schild beschützt werden (Dieses hat enur einen Pool voni Schildpunkten egal wer von den so Geschützen Schaden erleidet es wird von diesem einen abgezogen). Die Effektstärke von Schildzaubern ist um 1 erhöht.
\end{itemize}

\paragraph{Fokimagier}

\begin{itemize}
\item Quelle der Macht: Der Magier kann aus arkanen Machtquellen Verbrauchsfoki herstellen. Verbrauchsfoki können von jederman verwendet werden um magische Effekte zu bezahlen. Etwaige andere Limitierungen (z.B Aufladungsrate von verzauberten Gegenständen entfallen bei der Nutzung). Die Nutzung ist auf halbe Intelligentz plus halbe Willenskraft Punkte pro Runde beschränkt. Welche Machtquelle verwendet werden kann regelt das Setting (ebenso Gefahr und etwaige Kosten).  * gebundene Macht: Der Magier kann spezielle Foki herstellen die jede Runde Energiekosten bereitstellen. Die obige Limitierung bezieht sich auf jede Kombination von permanenten und verbrauchbaren Foki.
\item gebunden Seele: Der Magier kann einen Gegenstand herstellen der einen Teil seines Schadens aufnimmt. Der Gegenstand hat zwei Monitore deren größe von der Größe und Materialbeschaffenheit des Gegenstands abhängt. Beide Monitore sind gleich groß und können mindestens 10 (Amulett) und höchsten 50 Kästchen (Edelmetallstatue) aufnehmen. Genaueres bestimmt der Spielleiter nach Vorschlag des Spielers. Der Schaden wird 1:1 zwischen Charakter und Gegenstand aufgeteilt (Wahl des Spielers wo der 1 mögliche übrige Punkt hingeht).
\end{itemize}

\paragraph{Meister der Zauberspezialisierung}

Benötigt den normalen Weg der Zauberspezialisierung und führt diesen für den dort festgelegten Zauber fort. Dieser
Weg ist auch mehrmals lernbar (aber immer nur für einen bereits spezialisierten Zauber).

\begin{itemize}
\item Die 7 Winde: Die primären Farbwort in dem Zauber können gegen andere primäre Farbworte getauscht werden. Mit Zustimmung der Spielleitung können auch primäre Mischrunen wie POR getauscht oder eingesetzt werden.
\item Seelenecho: Die Kosten des Zaubers sind um 2 gesenkt. Sie können so auf Null fallen.
\item Geheimnis der Wiederholung: Effektstärke des Zaubers ist um 1 erhöht.
\end{itemize}

\paragraph{Meister der 7 Winde}

\begin{itemize}
\item Windwechsel: MAGI Zauber können einmal pro Runde als freie Handlung in eine Basisfarbe verwandelt werden. Ebenso können Zauber die nur eine Basisfarbe haben zurückverwandelt werden.
\item arkaner Wind: Die ersten 3 Punkte Zauberkosten die Runde entfallen (Z.B Aufrechterhaltungskosten).
\item arkaner Sturm: Zauber mit nur Basiselementen haben ihre Effektstärke um 1 erhöht. Für 1 FP kann zusätzlich ihre Effektstärke um 1 erhöht werden. (Das geht auch nachträglich bei aufrechterhaltenen Zauber)
\end{itemize}

\paragraph{Meister der Veränderung}

\begin{itemize}
\item Geheimnis des Lebens: (1FP) Der Heilungseffekt wird um 3 Effektstufen gesteigert.
\item Geheimnis der Elemente: Stärkungseffekte sind um 1 Effektstufe gesteigert
\item Geheimnis der Materie: Der Charakter kann ausserhalb von Kämpfen Material (auch arkanes/alchemistisches) permanent verwandeln. Das Material muss innerhalb seiner übergeordneten Gruppe bleiben (Blei in Gold verwandeln). Das geht mit der Rate von einer Zutat pro Stunde. Die Form bleibt dabei nicht erhalten. Um ein Stahlschwert in Silber zu verwandeln sind ein paar Tage nötig, das Schwert müsste danach neu geschmiedet werden).
\end{itemize}

\paragraph{Schicksalsschmied}

\begin{itemize}
\item der Erste der Götter: Göttliche Magie hat ihre Komplexität um 2 reduziert.
\item Geheimnis der Gedanken: Widerstandsproben mit Willenskraft gegen Zauber des Charakters bekommen einen Malus von 2W.
\item Schicksal: Wenn der Zauberer eine Person an ein Verhalten/Aufgabe bindet (CENSA AUM HUMI Effekte) dann ist die Komplexität um 2 reduziert und der Zeitraum der Wirkung verdreifacht
\end{itemize}

\paragraph{Schlachtmagier}

\begin{itemize}
\item Geheimniss der Entfesselung: LEKO Effekte sind doppelt so groß.
\item Geheimniss der Zerstörung (1FP): Magische Schilde schützen nur mit halben Stärke gegen Zaubers des Charakters. Wenn passend werden Rüstungen u.ä. nach dem erleiden von Zauber des Charakters um die Effektstärke reduziert (geschmolzen, zerttrümmert, etc)
\item Runenecho: Einfache Zauber können aber immer wieder gesprochen werden und dann zusammen ausgelöst werden. Solche Zauber kumulieren ihre Freisetzungseffekte oder(!) Schadenseffekte. Schildeffekt schützen ausserdem einen proportional größeren Bereich. (Hier hat der Spielleiter das letzt Wort, eine gute Faustregel ist die Quadratwurzel aus der Anzahl der Zauber zu ziehen und als Faktor zu verwenden. Beispiel nach 100 mal ORT TEL FLAM LEKO macht der Zauber 10 mal mehr Schaden oder der LEKO Effekt ist verzehnfacht. Die Wiederholungen können auch auf die unterschiedlichen Effekte akkumuliert werden (In dem Beispiel könnte 100 mal Sprechen einen Faktor von 10 ergeben der den Radius um den Faktor 5 und den Schaden um den Faktor 5 erhöht). Der Spielleiter sollte hier kommunizieren ob er solche Landschaftsvernichter möchte und wie er sie handhabt).
\end{itemize}

\subsubsection{Nahkampf}

\paragraph{Axtmeister}

\begin{itemize}
\item Hacken: Waffenwert und Schaden sind um 1 erhöht
\item Rüstung zerschlagen (Manöver): Bei einem erfolgreichen Treffer wird der Rüstungsschutz (bis zur Reperatur/Heilung) um 1 reduziert. (Bevor der Schaden berechnet wird).
\item Klaffende Wunden (Manöver): Der Charakter kann eine beliebige Zahl Würfel aus seinem Würfelpool der Angriffsprobe nehmen. Jeder so reduzierte Würfel erhöht den Schaden um 2.
\end{itemize}

\paragraph{Dolchmeister}

\begin{itemize}
\item schnelle Klinge: Das Bereitmachen eines Dolches kostet keine Handlung. Das gilt auch für geworfene Messer. Waffenwert ist um 1 im Angriff erhöht.
\item Schwachstellen finden (Manöver): Es wird ein normaler Angriff durchgeführt mit -2W. Es wird ein Punkt Rüstung des Ziels für den Rest des Kampfes ignoriert falls der Angriff gelingt. Das kann wiederholt werden und ist kumulativ.
\item Meister des Dolches: Behinderung von Rüstungen ist gleich ihrem Schutzwert. Verteidigung und Bewegung sind um 2 erhöht. Dolche verursachen 3 Punkte mehr Schaden.
\end{itemize}

\paragraph{Meister der Beinarbeit}

\begin{itemize}
\item Beweglichkeit: Der Charakter kann zwei Felder als freie Handlung bewegt werden ohne Gelegenheitsangriffe zu provozieren. Dieser Schritt kann auch Teil einer rausgezögerten Hanldung sein.
\item Durchschlängeln: Der Charakter kann durch feindliche Felder durchziehen (sofern sinnvoll). Jedes so durchquerrte Feld kostet aber einen zusätzlichen Bewegungspunkt.
\item Manöver stören (Manöver): Der Charakter kann diese Runde nicht von Effekten anderer Manöver betroffen sein die sich gegen seine Verteidigung, Bewegung oder Würfelpool richten. Für Manöver die ihn zu Fall bringen oder seine Position ändern sollen bekommt er 4 Bonuswürfel. Dieses Manöver ist mit defensiven Manövern wie volle Verteidigung kombinierbar.
\end{itemize}

\paragraph{Meister der Rüstungen}

\begin{itemize}
\item Tragegewohnheit: Der Behinderungsmalus von Rüstungen ist um 1 reduziert.
\item Rüstungsveteran: Die Rüstung schützt auch gegen elementaren Schaden wie Feuer oder Blitz. Schaden der direkt die Lebenskraft angreift wie MORT Zauber sind davon aber nicht betroffen.
\item angepasste Rüstung: Sofern der Charakter mindestens 3 Punkte Rüstungsschutz hat steigt dieser um 1. Der Rüstungsschutz kann so den Wert 7 erreichen.
\end{itemize}

\paragraph{Schildmeister}

\begin{itemize}
\item Schildhand: Waffenwert mit Schild ist um 1 erhöht. Die Verteidigung gegen Geschosse ist um 1 erhöht.
\item Schildschlag (Manöver freie Handlung): Das Ziel des Manövers verliert 1 Punkt Verteidigung oder weicht einen Schritt mittels seiner reaktiven Handlung zurück.
\item Parade: Ein erfolgreicher Angriff kann als reaktive Handlung oder für 1FP stattdessen Schaden gegen das Schild machen. Schilde haben je nach Material und Größe unterschiedliche Lebenspunkte (zwischen 5 schlechtes Holzschild und 20 für ein gutes Metalschild). Wird ein Schild zerstört so verfällt der übrige Schaden.
\end{itemize}

\paragraph{Schwertmeister (einhändig)}

\begin{itemize}
\item Fechter: Waffenwert ist um 1 erhöht. Wird nur mit einem einhändigen Schwert (kann zweihändig geführt sein) steigt der Schaden um 2
\item halbe Verteidigung (Manöver): -3W die Verteidigung des Charakters ist um 2 erhöht. Kombinierbar mit voller Verteidigung.
\item Niederstrecken (Manöver): Das Angriffergebniss ist um 2 erhöht. Es werden ausserdem 2 Punkte Extraschaden gemacht. Diese Boni können für 1 FP verdoppelt werden.
\end{itemize}

\paragraph{Speermeister}

\begin{itemize}
\item Aufhalten: Waffenwert ist um 1 erhöht. Gelegenheitsangriffe machen 3 Punkte mehr Schaden
\item Reichweite: Die Kontrollzone ist um 1 Feld erhöht. Angriffe können ebenfalls über zwei Felder Reichweite gemacht werden.
\item Aufspiessen(Manöver): Der Angriff ignoriert bis zu 5 Punkte Rüstung. Falls der Angriff trifft halbiert sich die Bewegung des Ziels für eine Runde. Die Verteidigung des Ziels sinkt ebenfalls für eine Runde gegen alle weiteren Angriffe um 1.
\end{itemize}

\paragraph{Stabmeister}

\begin{itemize}
\item Wanderer: Verteidigung ist wie der Waffenwert um 1 erhöht
\item zwei Enden (Manöver): Der Angriff verursacht zusätzlich 2 Punkte Benommenheit falls er trifft. Erfolgsgrade erhöhen die Benommenheit anstatt des Schadens.
\item peitschende Hiebe (Manöver): Trifft der Angriff so steigt die Behinderung des Getroffenen für diese Runde um 1. Kann mit zwei Enden kombiniert werden.
\end{itemize}

\paragraph{Waffenmeister}

\begin{itemize}
\item kritische Treffer: Erfolgsgrade werden pro Punkt über dem Mindestwurf berechnet (statt 2)
\item Bedrängen (Manöver): Verteidigung des Ziels sink um 1. Diese Verteidigungssenkung gilt auch für alle anderen Angriffe!
\item Überlegenheit: Statt einer Bewegungshandlung kann ein zweiter Angriff durchgeführt werden. (Die Aktionshandlung muss hierfür aber ebenfalls Angriff sein!)
\end{itemize}

\paragraph{Zweihandmeister (Schwert oder Axt)}

\begin{itemize}
\item Ansturm: Waffenwert ist um 1 erhöht. Schaden ist um 2 erhöht
\item Spalten: Schaden ist um 3 erhöht.
\item Durchschwingen: Falls der erste(!) Angriff gegen das Ziel fehlschlägt kann mit der reaktiven Handlung oder für 1FP sofort ein weiterer Angriff gemacht werden.
\end{itemize}

\paragraph{Meister mit zwei Waffen}

\begin{itemize}
\item Klingenwirbel: Waffenwert ist um 1 erhöht. Verteidigung ist um 1 erhöht.
\item zweiter Hieb: Mit der Zweithandwaffe kann ebenfalls ein Angriff gewürfelt werden. Sollte der Waffenmeister gelernt sein, so kann mit der Bewegungshandlung ein dritter Angriff durchgeführt werden 
\item 3er Kontakt: Trifft der erste Angriff und der zweite aus ''zweiter Hieb'' so kann einer der folgenden Effekte gewählt werden: 1.) Niederwerfen vergleichende Stärke/Geschickproben (jeweils Wahl durch Kontrohenten). Misslingt die Probe des Verteidigers so fällt er zu Boden. 2.) Die Verteidigung des Getroffenen ist für den Rest der Runde gegen alle weiteren Angriffe um 2 reduziert. 3.) Es werden für den Schaden des zweiten Angriffs 4 Punkte addiert.
\end{itemize}

\subsubsection{Reiterei}

\paragraph{Dressurmeister}

\begin{itemize}
\item es versteht jedes Wort: Für einen Glückspunkt oder einen Fokuspunkt des Tiers wird auch ein komplexer Befehl verstanden. Aber es ist immer noch ein Tier (z.B. bringe den Verwundeten zu einem Heiler -> Tier kehr erst zurück wenn sich jemand um den Verwundeten kümmert, wird aber etwaige Approbationsurkunden nicht genauer in Augenschein nehmen).
\item Lernfähig: Der Charakter kann für seine Erfahrungspunkte Attribute, Fertigkeiten und Wege des Tiers verbessern. Hier hat wie immer wenn es ums Verbessern geht der Spielleiter das letzte Wort. Selbiger ist aber auch angehalten dem Tier für herausragende Leistung eigene Erfahrungspunkte zu geben.
\item Legendär: Alle körperlichen Attribute des Tieres verbessern sich um 1. Ausserdem wird ein herausragendes Talent des Tier wird verbessert. Passende Beispiele wären: Zweiter Angriff für einen Kampfgefährten. 50\% mehr Geschwindigkeit für ein Reittier. Besondere Sinne für ein Wachtier oder Jagdtier.
\end{itemize}

\subsubsection{Steckenmagier}

\paragraph{Energiemeister}

\begin{itemize}
\item Umgebungsmagie: Die Energiekosten für Zauber sind um 1 reduziert. Solche reduzierten Zauber können auch ohne Kosten gesprochen werden.
\item Steckenbindung: Solange der Stecken nicht genutzt wird kann er einen mit ihm gesprochenen Zauber aufrecht erhalten. Für den Energiemeister fallen keine Kosten an.
\item Nachglühen: Wenn der Energiemeister exakt den gleichen Zauber spricht den er in der Runde zuvor gesprochen hat so halbieren sich die Kosten (aufrunden!).
\end{itemize}

\paragraph{Steckenmeister}

\begin{itemize}
\item Speicher: Die Kapazität des Stecken ist verdoppelt sofern vorhanden.
\item Effektstärke: Die Effektstärke des Steckens kann um 1 Punkt erhöht oder gesenkt werden.
\item Kristall: Falls der Stecken die Komplexität reduziert so reduziert er sie um 2 weitere Punkte.
\end{itemize}

\subsubsection{Verzauberung}

\paragraph{Artefaktmeister}

\begin{itemize}
\item Ausbrennen: Der Speicher des Artefakt wird permanent um 10\% reduziert und dabei vollständig geladen. Eine Reperatur für 10\% der Artefaktkosten ist möglich um den Speicher wieder herzustellen. Mehrmaliges Ausbrennen ist möglich erhöht aber auch entsprechend die Reparaturkosten. Ausbrennen erfordert eine aktive Handlung.
\item Überladen: Für doppelte Kosten oder Aufladedauer kann der Speicher eines Artefakts um bis zu 50\% zusätzliche Ladungen enthalten.
\item Entladen: Sofern das Artefakt es möglich macht kann es bis zu dreimal in einer Runde genutzt werden. Aber hier gilt das Prinzip der Magieabstossung, d.h. es kann ein Ziel immmer nur einmal in dieser Runde betroffen sein.
\end{itemize}

\paragraph{Interdisziplinärer Meister}

Spielleitung und Spieler sollten hier absprechen was im einzelnen möglich und sinnvoll ist. Sollte dieser Meisterweg
bei der Verzauberung eingesetzt werden so kann, falls passend, die Spielleitung auf spezielle Recherche, seltene
Materialen, astronomische Konstellation oder Ähnliches bestehen. (Die Anzahl der so erstellten Artefakte sollte
limitiert sein)

\begin{itemize}
\item arkane Erkenntnis: Wege aus anderen arkanen Bereichen können bei der Nutzung eines verzauberten Gegenstands eingesetzt werden (der Verzauberer muss sie natürlich beherschen).
\item arkanes Wissen: Wege aus anderen arkanen Bereichen können bei der Verzauberung eingesetzt werden sofern sinnvoll.
\item arkane Erleuchtung: Meisterwege aus anderen arkanen Wegen können sofern sinnvoll und vom Verzauberer erlernt bei der Artefaktherstellung eingesetzt werden.
\end{itemize}

\paragraph{Runenschmied (Erschaffung)}

\begin{itemize}
\item Gravur: Schwierigkeit sinkt um 2. Der Zeitraum zur Erschaffung wird verdoppelt (kumulativ mit Artefaktschmied möglich). Der Gegenstand ist äusserst robust und eigentlich nur mit Magie zerstörbar.
\item Runenwort: Willenskraft des Gegenstands steigt um 2.
\item Runenmacht: Effektstärke steigt um 1.
\end{itemize}

\begin{center}
\section{Magie}
\end{center}

\begin{center}
\subsection{Regeln der Magie}
\end{center}

Im folgenden Kapitel geht es um Magie. Magische Effekte die z.B. durch Zauber verursacht werden werden durch die
Runenworte beschrieben die ihn formen. Mittels genannter Runenworte wird in einer Art magischen Sprache der Effekt
beschrieben. Eine Rune ist dabei ein einzelnes Wort innerhalb der Beschreibung.

Magische Effekte werden durch drei Eigenschaften beschrieben:

\begin{itemize}
\item Form: Der Komplizierte Teil. Was ein Zauber bewirkt. Hier sind primär die Formworte entscheident. Allerdings kann die Form um Zauberbeginn und Zauberendworte modifiziert werden.
\item Farbe: Das Ziel des Zaubers. Es spzezifiziert die Formworte um das ''Was''. Zauberfarben beschreiben den Zauber meist ''zu Ende''. Ein Kontrollzauber über die Farbe ''Gegenstände'' wird im Gegensatz zu ''Menschen'' als Telekinese anstatt Bezauberung verstanden.
\item Effektstärke: Sie beschreibt um einen die Mächtigkeit als auch die ''Tiefgründigkeit'' des Zaubers. Angriffszauber leiten aus ihrer Stärke zum Beispiel den Schaden ab.
\end{itemize}

Im folgenden wird anhand einiger Beispiele von einfachen Zaubern zu komplizierten Zaubern gegangen. Bei den
einzelnen Runenwörtern können noch weitere Informationen und Hinweise gefunden werden.

Wenn ein Zauber allgemein beschrieben ist, darf der Zaubernde Kleinigkeiten oder weitergehende Details frei beschreiben.
Er darf bei einer Lichtbeschwörung z.B. die Farbe modifizieren (ORT PRIX als rötliches Licht) oder bei einer Illusion
(in Grenzen) Farben und Form beschreiben.

\begin{mdframed}[hidealllines=true, backgroundcolor=black!10]
\subparagraph{Spielleiterhinweis}

Da es sich hier um ein freies Magiesystem handelt, sollte der Spielleiter Buch über alle getroffenen Entscheidungen
zu ''komplizierten'' Zaubern führen die die Charaktere entwickeln. So hat er Präzedenzfälle parat um zukünftige
Zauber bewerten zu können. Am Ende des Kapitels gibt es (nochmals) eine Zusammenfassung aller Regeln (die natürlich
durch Hausregeln ergänzt werden können).

\end{mdframed}
\paragraph{Beispiel}

Beispielhaft für die Verwendug der Magie (Effektgebunde Magie durch Runenworte) sind hier drei Charaktere:
\begin{itemize}
\item Frau Wentir,  eine allwissende Lehrerin
\item Fuliis ein aufmerksamer Schüler
\item Lanu eine etwas unbeholfene Schülerin
\end{itemize}

\subsubsection{einfache Zauber}

Ein einfacher magischer Effekt setzt sich Zusammen aus Effektform und Farbe.

\begin{itemize}
\item Form ist das was der Zauber grundsätzlich macht, Z.B Geschoß oder Kontrolle.
\item Farbe beschreibt das Ziel oder die Wirkung genauer (Feuer, Gegenstände).
\item Die simplesten Hervorrufung brauchen nur ein Elementar bzw Farbwort.
\item Formrune kommt vor der Farbrune.
\end{itemize}

\paragraph{Beispiel}

\begin{itemize}
\item Lanu: ''Wie kann ich ein Lagerfeur entzünden?''
\item Fuliis: ''Mit FLAM rufst du ein kurzfristigen Feuerblitz hervor''
\item Lanu: ''Aber da ist ja garkeine Form dabei!''
\item Fuliis: ''Einfache Hervorrufungen benötigen nur die Farbe...''
\item Frau Wentir: ''Wenn man das Lagerfeuer kontrollieren möchte was wäre der richtige Spruch?''
\item Lanu: ''Ähmm...''
\item Fuliis: ''AUM FLAM. Die Zauberform ist ''Kontrolle'' durch die AUM Rune und da wir Feuer kontrollieren wollen, beschreiben wir den Effekt durch FLAM. Auch wenn es etwas unangenehm ist, da der Zauber auf Berührungsreichweite gesprochen wird.''
\item Frau Wentir: ''Sehr gut, Lanu hast du alles verstanden? Was ist die Basis eines Zaubers?''
\item Lanu: ''Farbe und Form!''
\item Frau Wentir: ''Form und Farbe!''
\item Lanu: ''Sag ich doch.''
\item Frau Wentir: ''Die Reihenfolge ist äusserst wichtig. Erst die Form festlegen und dann mit der Farbe ergänzen.''
\end{itemize}

\subsubsection{Zauberbeginnwörter}

Einfache Zauber wirken auf Berührungsreichweite. Da das meist nicht praktikabel ist können sie mit Zauberbeginn
modifiziert werden.

\begin{itemize}
\item Zauberbeginn sind folglich die ersten Runen
\item Diese können ohne Verbundrune kombiniert werden. (Reihenfolge ist Multiplikationswörter, Zielrune, sonstige)
\end{itemize}

\paragraph{Beispiel}

\begin{itemize}
\item Frau Wentir: ''Zeigt mal euer Kampfrepertoire''
\item Lanu: ''TEL FLAM ORT. Huch warum brennt denn jetzt mein Haustier?''
\item Fuliis: ''ORT AUM FLAM KONIU. Das sollte zum löschen reichen. So Frau Wentir ...''
\item Frau Wentir: ''Gleich Fuliis. Was hast du falsch gemacht Lanu?''
\item Lanu: ''Ich hab wieder keine Aufzeichnung gemacht oder mir die Reihenfolge gemerkt...''
\item Frau Wentir: ''Genau, wie muss der Spruch lauten?''
\item Lanu: ''TEL FLAM ist doch richtig, oder? Und ORT ist ein Zauberbeginnwort, wo kam das nochmal hin?''
\end{itemize}

\subsubsection{Zauberendwörter}

Zauberformen können am Ende noch kontrolliert werden. Meist ist da ein zusätzlicher Aspekt oder eine zeitliche
Kontrolle gemeint.

\begin{itemize}
\item Zauberendworte schliessen den Zauber ab und kommen nach der Farbe
\item Sie sind nicht(!) beliebig kombinierbar (KONFAR und KONIU schliessen sich z.B. aus, KONIU und AUM wären möglich)
\item Längerfristige und ''nachgesteuerte'' Zauber sollten AUM KONIU als Schlussworte haben
\end{itemize}

\paragraph{Beispiel}

\begin{itemize}
\item Lanu: ''Wenn mir mal ein Licht aufgehen soll sprech ich PRIX KONIU''
\item Frau Wentir ''Und wenn jemand anderem eins aufgehen soll''
\item Lanu: ''Hmm mit ORT TEL PRIX sollte das gehen, aber was passiert eigentlich wenn ich da KONIU ranhänge?''
\item Frau Wentir: ''Gute Frage Lanu. Leider ist der Effekt ein Geschoss und damit mit Auftreffen beendet. Allerdings kannst du die Effektdauer des Geschoss erhöhen und so, wenn du genug Zeit hast, es viel weiter schmeissen''
\item Fuliis: ''Wo wir bei Fragen sind warum ist der Stärkungszauber für Intelligenz PRIX KONFAR und nicht etwa KONFAR PRIX?''
\item Frau Wentir: ''Das hängt damit zusammen das du elementares Licht erschaffst und es mit KONFAR in dein Ziel bannst. Bei Menschen führt das dazu das ihnen im wahrsten Sinne ein Licht aufgeht. Die KONFAR Rune bestimmt hier auch den zeitlichen Aspekt, wie du ja weisst benötigen Stärkungszauber keine KONIU Rune. Auch solltet ihr wissen KONFAR und KONIU beissen sich immer''
\end{itemize}

\subsubsection{komplexere Magie}

Einige Formworte können auch als Farbworte verwendet werden. Bestes Beispiel ist hier die Rune TEL die auch für das
Beschreiben von Geschossen wie z.B. Pfeilen verwendet werden kann.

Magie kann neben der Effektstärke auch ''Deutlichkeit'' gewinnen indem die einzelnen Runenworte mit VAS oder INGVA
verstärkt werden. So könnte ein ORT AUM EBOR KONIU zwar Erde bewegen (und mit Hilfe der Effektstärke wird bestimmt
wieviel Erde bewegt wird), für grazile Kontrolle um z.B. Figuren zu formen sollte dann aber besser 
ORT VAS AUM EBOR KONIU gesprochen werden.

Sowohl Farb als auch Formworte können mit CONDUC verbunden werden. Dabei wird ihre Form oder Farbe gemischt. So kann
die Rune POR (Blitz) aus den Runen FLAM (Feuer) und LITAX (Luft) gebildet werden. Ein kegelförmigen Angriffszauber
kann man mit LEP CONDUC LEKO beschreiben (ein sich freisetzender Strahl)

Viele Zauber leben von ihrer deutlichen Beschreibung in der Farbe. So sollte eine Illusion von einem Tier verbessert
werden in dem man nach der Tierrune weiter beschrieben wird. Dabei gilt die Farbworte werden nicht gemischt sondern
jedes weiter ergänzt die vorherigen. So ist ein FERA NOX (Tier Dunkelheit) eine gute Beschreibung für ein Raubtier.
Ob eine Beschreibung passt sollte mit der Spielrunde abgesprochen sein.

\paragraph{Beispiel}

\begin{itemize}
\item Fuliis: ''Mit TEL FLAM KONFAR kann ich Feuerpfeile herstellen?''
\item Frau Wentir: ''Ja. TEL ist hier kein Formwort sondern schon Teil der Farbe. Besser verständlich wird das bein einem 
\item ORT VAS AUM TEL KONIU mit dem man nervige Geschosse zurückschleuder kann''
\item Lanu: ''Hmm PRIX CONDUC FLAM beschreibt die Sonne, dann macht es Sinn das Kreaturen des Nebels diese nicht mögen, schliesslich kommt dort die Beschreibung WAKU CONDUC NOX am besten hin.''
\item Frau Wentir: ''Hmm das klingt vernünftig jedes ELement hat jeweils seinen Konterpart. Wie würdest du eine Illusion einer Eiche erschaffen''
\item Lanu: ''Das ist einfach ORT SYMA INGVA FLORA EBOR KONIU. Die Eiche ist ja ein recht großes Gestrüpp und am besten passt bei ihr das Attribut Erde für Ausdauer. Die Frage ist doch eher ist eine Weide besser mit Wasser wegen der Biegsamkeit beschrieben oder mit Luft?''
\item Frau Wentir: ''Wie würdest du das herausfinden?''
\item Lanu: ''Ähmm...''
\item Frau Wentir: ''Wir sollten uns bei Gelegenheit mal über die Farben bei Wahrnehmungszaubern unterhalten''
\end{itemize}

\subsubsection{Prinzipien der Magie}

Es gibt noch ein paar Prinzipien zu beachten:

\begin{itemize}
\item Bereitwilligkeit von unbelebten Zielen. Objekte lassen Zauber zu, ob diese stark genug sind etwas zu bewirken ist etwas ganz anderes.
\item Seelenwiderstand: Belebte Ziele können einen Zauber zulassen oder versuchen sich gegen ihn zu wehren. Es steht dem Opfer immer ein vergleichende Willenskraftprobe zu um sich einem Zauber zu widersetzen. Kampfzauber sind meist davon ausgenommen, da sie z.B. ein Geschoss erschaffen.
\item Freiwilligkeit: Willenskraftproben gegen Zauber können freiwillig versagt werden!
\item Prinzip der Abstossung: Ein Ziel kann immer von einem Zauber nur einmal betroffen sein. So bekommt ein Ziel das in der Überlappungszone von Explosionen desselben Zaubers steht trotzdem nur einfachen Schaden.
\item Prinzip der Abstossung: Gleichartige Zauber stossen sich ab. Dabei ist auch die Farbe zu beachten. So kann ein Ziel nur mit einem primären Element verstärkt werden oder nur eine zusätzlich Sicht mit SCIEN haben. (Bedingt hilft hier die CONDUC Rune weiter).
\item Prinzip der Abstossung: Gleichartige Zauber stossen sich mit echten physikalischen Kräften ab (kann für einfache Mechaniken verwendet werden). Soll ein Ziel mit einem neuen Zauber überschrieben werden so ist eine vergleichende Willenskraftprobe zwischen dem Zauberer des Originals und des neuen Zaubers fällig (hier gilt auch Freiwilligkeit).
\item Prinizp der Abstossung: Kampfzauber/Schildzauber Schildzauber erleiden nur halben Schaden wenn sie von Effekten der gleichen Farbe getroffen werden.
\item Prinzip der Auslöschung: Kampfzauber/Schildzauber erleiden doppelte Schaden wenn sie von Elementen der gegensätzlichen Farbe getroffen werden. (z.B: Feuerschild das von Wasser oder Eisgeschossen getroffen wird, oder von einem ORT TEL AN FLAM)
\item Prinzipien der Auslöschung, Gegenzauber: Genaueres wird im Kapitel Gegenzauber beschrieben. Exakte Gegenzauber ( entweder gleiche Form und exakte Gegenfarbe oder exakte Gegenform und gleiche Farbe) löschen sich bei gleicher Effektstärke aus (kein Willenskraftwettstreit!)
\end{itemize}

\paragraph{Beispiel}

\begin{itemize}
\item Lanu: ''Erst Zauber ich NOX KONFAR um meine geistige Ausdauer zu verbessern, dann PRIX KONFAR um klüger zu werden. Dann bin ich aber trotzdem müde. Allerdings verstehe ich nach PRIX KONFAR das ich den NOX KONFAR überschrieben habe.''
\item Frau Wentir: ''...''
\item Fuliis: ''Gegen Willenskraftsschwache Gegner wäre das auch eine Möglichkeit ihre Starken Verzauberungen durch unnütze oder Schwächere zu ersetzen...''
\item Lanu: ''Wäre da nicht das Prinzip der Auslöschung besser? Ich wirke einfach denselben Zauber nur AN - ders.''
\item Fuliis: ''Einerseits ja, aber möglicherweise ist der unnütze Verstärkungszauber einfacher zu wirken.''
\item Lanu: ''Wie löse ich eigentlich das Problem mit meinen Verstärkungszauber?''
\item Frau Wentir: ''CONDUC Rune. Sie kann auch Gegensätze wie Licht und Dunkelheit verbinden, zumindest primär elementar.''
\end{itemize}

\subsubsection{Effektstärke}

Nach dem Form und Farbe festegelegt sind und damit der Effekt des Zaubers feststeht muss die Effektstärke festgelegt
werden. Diese Effektstärke bestimmt bei Kampfzaubern wieviel Schaden mit einem Zauber gemacht oder verhindert wird
oder bei einem Kontrollzauber wie gut und wie viel kontrolliert wird. Effektstärke bestimmt z.B.:

\begin{itemize}
\item Kampfzauber, Schaden verursachen, verhindern oder heilen. Genaueres im Kapitel Kampf.
\item Illusionszauber, die Größe der Illusion, oder ihre Deutlichkeit. Mögliche Boni auf Würfelpools werden durch die Effektstärke beschrieben.
\item Kontrollzauber, Die Menge, Geschwindigkeit oder Größe des zu Kontrolierenden.
\item Wahrnehmungszauber, Sichtweite oder Detailgrad. Hohe Effektstärke kann Bonuswürfel zu Wahrnehmungs- oder Analyseproben geben.
\item Veränderungszauber, Hier bestimmt die Effektstärke zusammen mit den Farben wie weit verändert werden darf.
\item Stärkungszauber, Die Effektstärke bestimmt den Bonus der Stärkung (z.B. um wieviel ein Attribut gesteigert wird).
\item Beschwörungszauber, Wesenheiten. Die Größe des zu rufenden Wesens.
\item Beschwörungszauber, Gegenstände. Größe und etwaige Boni werden durch die Effektstärke bestimmt.
\end{itemize}

\subsubsection{Wirkungsdauer von Magie}

Zaubereffekte sind sofern sie nicht anders beschrieben sind kurzlebig. Ohne Ergänzung durch Runen wie KONIU
wirken sie in der Runde (Kampfrunde oder eine Handlung in einer Szene) in der sie gezaubert werden. Das erlaubt
Schadenswirkung oder ein schnelles Umschauen z.B. mit einem SCIEN Zauber. Soll der Zauber weitergehen so kann er
mit dem Zauberendwort KONIU verlängert werden. Solange der Zauber wirkt muss der Zaubernde aber Energiekosten zahlen
um den Effekt stabil zu halten (Die Kosten sind in der Regel 1/3 der Originalkosten, es wird aufgerundet).

Zaubereffekt können ebenfalls mittels KONFAR an eine Person oder Gegenstand gebunden werden. Dabei ist zu beachten
das KONFAR noch Einfluss auf die Wirkung hat. So ist ein ''FLAM'' Zauber nur ein kurzer Feuerblitz während FLAM KONFAR
die Stärke verbessert.

Aufrechterhatungskosten müssen im Kampf oder für kampfrelevante Zauber (Schaden, Heilung, Schilde,...) jede Kampfrunde
bezahlt werden. Zauber mit weniger direkter Wirkung werden pro Probe/Szene bezahlt. Hier sollte aber spätestens alle
5min Aufrechterhaltungskosten bezahlt werden (z.B. AUM EBOR KONIU um sich durch Gestein zu bewegen, der Spielleiter
bestimmt wie lange das dauert und legt die Aufrechterhaltungskosten fest).

Regeln für die Wirkdauer:

\begin{itemize}
\item KONIU, wirkt solange der Zauberer den Zauber aufrechterhält. Dafür ist keine Handlung nötig.
\item KONIU kostet ein Drittel der Originalkosten (aufgerundet) an Energiekosten pro Runde oder Handlung.
\item KONFAR Zauber wirken für die Szene oder Willenskraft in Minuten.
\end{itemize}

\paragraph{KONFAR oder KONIU}

KONFAR hat seine eigene Wirkdauer und kann so nie mit KONIU im selben Zauber stehen. KONFAR verändert den Zauber und
macht ihn so ''passiv''. KONFAR ist die Rune für Verzauberungs (nicht mit Bezauberung verwechseln) oder Stärkungsmagie.
KONIU lässt den Zauber so wie er ist fortbestehen.

\subsubsection{Komplexität}

Nachdem man den Zauberspruch und den zugehörigen magischen Effekt beschrieben hat, muss man sich Gedanken über
''Kosten'' des Zaubers sowie die Schwierigkeit ihn zu wirken machen. Dafür wird zuerst die Komplexität des Zaubers
bestimmt. Aus dieser kann man die Basiskosten und die Schwierigkeit ableiten. Bei der Bestimmung geht man folgenden
Schritte nacheinander durch:

\begin{itemize}
\item Komplexität ist Summe der Komplexität der Runenwörter. Dabei ist zu beachten das einige Runenwörter andere in ihrer Komplexität modifizieren, sowie das evtl die Komplexität der Runenworte durch Charakterboni oder -mali modifiziert sind.
\item Komplexität durch die Effektstärke wird addiert (siehe Tabelle unten).
\item Wege und Talente werden angewand (sofern vorhanden) um die Komplexität weiter zu modifizieren.
\item Schwierigkeit ist 10 + Komplexität.
\item Kosten sind gleich der Komplexität.
\item Wege, Talente und Sonderregeln die Schwierigkeit und/oder Kosten modifizieren werden angewandt.
\end{itemize}

Nachdem die Schwierigkeit errechnet wurde muss der wirkende Charakter eine entsprechende Probe würfeln um den Effekt
auszulösen. Sollten andere Schwierigkeiten erreicht werden müssen (trifft ein Feuergeschoss einen Gegner) so kann das
mit derselben Probe abgegolten werden.

Das Bezahlen der Kosten hängt von der Kampagne und dem Magiestil ab. Hier sollte der Spielleiter zu Beginn der
Kampagne Regeln festlegen und kommunizieren.

Nachfolgende Tabelle gibt an wie die Effektstärke die Komplexität des Zaubers erhöht.


\begin{small}
\begin{tabular}{|m{3cm}|m{3cm}|}
\hline
\textbf{Effektstärke}&\textbf{Komplexität}\\
\hline
\hline
0&0\\
\hline
1&1\\
\hline
2&3\\
\hline
3&6\\
\hline
4&10\\
\hline
5&15\\
\hline
\end{tabular}
\end{small}

\subsubsection{Deutlichkeit}

Wie bei komplexen Zaubern erwähnt können Farben an Deutlichkeit gewinnen. Diese Deutlichkeit kann aus zwei verschiedenen
Quellen kommen. Zum einen sollte die Farbe ausreichend beschrieben werden durch hinzufügen weiterer Farbworte.
So kann der Magier z.B bei einer Tierillusion mit SYMA FERA ein beliebiges Tier formen. Glaubwürdiger wird es aber
wenn weitere Farbworte dazukommen um das Tier genauer zu beschreiben. Die andere Methode ist es das primäre Farbwort
durch Verstärkungsrunen wie VAS oder BET zu modifizieren. Sollten aus Sicht des Spielleiters eine ungenügende
Deutlichkeit vorliegen so kann z.B. ein Bonus auf Widerstandsproben gegeben werden oder der Zauber funktioniert
eingeschränkt. Auch sollte die Effektstärke angemessen sein (Größe von Illusionen sollte über die Effektstärke
abgebildet). Bei besonders deutlichen Zaubern kann auch ein Bonus für den Zauber gegeben werden. Hier sollte, wie
eingangs erwähnt wurde, Buch geführt werden damit gleichartige Zauber auch gleichartig wirken.

\begin{mdframed}[hidealllines=true, backgroundcolor=black!10]
\subparagraph{Spielleiterhinweis}

Ursprünglich entstand das Magiesystem als System bei dem die Charakter Runenwort und damit verbundenen Gesten frei
Nutzen konnten, sofern sie Kenntnis über die Runen hatten. Hier bietet sich an sich relativ schnell regenerierenden
geistige Energie (z.B. geistigen Kreuzschaden, einfache Erschöpfung o.ä.) zu verwenden. Das führt zu einer sehr
stark magischen Welt in der Runenkenntnis viel wert ist. Andere Alternativen magische Effekte zu verwenden sind:

\begin{itemize}
\item Steckenmagie. Magie wird über Runenringe die auf einen Stecken gesteckt werden gewirkt. Stecken stellen die Effektstärke bereit. Besitz und Herstellung sind limitiert oder ein untergegangenes Geheimnis.
\item Verzauberungen. Werden initial erstellt und funktionieren dann immer. Haben aber Schwierigkeiten und Sonderregeln um die magischen Kosten zu bezahlen.
\item Alchemie. Effekt und Effektstärken werden durch die Zutaten bestimmt. Diese übernehmen auch die Kosten, allerdings erfordert das Herstellen (meist einfacher) Zauber viel Zeit und Vorbereitung.
\item Limitierung der Magie durch andere Ressourcen (Runenstein die den Zauber formen und dann verbraucht sind, stellare Konstellationen die magische Farben oder Formen erlauben oder verbieten).
\end{itemize}

\end{mdframed}
\subsubsection{Antimagie und Magiewiderstand}

In einer stark magischen Welt muss es Möglichkeiten geben sich gegen Magie zu schützen oder diese zu unterbinden und
aufzulösen. Wie bereits in den Magieregeln erklärt kann ein Lebewesen Magie widerstehen (oder die Magie zulassen).
Hier ein paar Regeln wie dieser Widerstand aussehen kann:

\begin{itemize}
\item Kontrollzaube: Nachdem der Zauber erfolgreich gewirkt wurde, treten Zauberer und Ziel in einen Willenskraftwettstreit ein. Verliert das Ziel ist es unter Kontrolle des Zaubers. Gewinnt das Ziel wird der Zauber automatisch beendet. Je nach Erfolgsgraden dieses Willenskraftwettstreit sollte der Spielleiter erklären wann oder zu welchen Anlass diese Probe wiederholt wird. (siehe Spielleiterhinweis)
\item Illusionszauber: Da sie nicht direkt auf ein Ziel wirken sollte die Widerstandsprobe erst stattfinden wenn die Charaktere begründeten Anlass zum Zweifel haben oder mit der Illusion interagieren. Die Widerstandsprobe kann über Wahrnehmung und Aufmerksamkeit gemacht werden. Illusionen die direkt den Geisteszustand beeinflussen (z.B. Schmerzen zufügen) kann auch mit Willenskraft widerstanden werden.
\item Wahrnehmungszauber: Charaktere mit Ahnung von Magie können versuchen sich dieser Wahrnehmung (wie jeder anderen Wahrnehmung) entziehen indem sie entsprechende Sichtdeckungen nutzen (und verstehen das Sichtdeckungen für Wahrnehmungszauber anders funktionieren). Im Zweifel können hier Schleichenproben verwendet werden.
\item Schadenszauber: Diese werden behandelt wie Angriffe der Charakter kann dieselben Methoden verwenden um ihnen zu widerstehen. Zur Not muss er den Schaden aushalten. Eine weitere Widerstandsprobe ist nicht geplant (aber siehe nächster Punkt)
\item Veränderungszauber: Zauber die den Körper eines Charakters manipulieren können anstatt mit Willenskraft auch mit Konstitution widerstanden werden.
\end{itemize}

Um einen Zauber aufzulösen gibt es verschieden Möglichkeiten. Die Regel der Abstossung erlaubt es möglicherweise den
ursprünglichen Zauber durch einen ähnlichen Zauber zu ersetzen. Sollte das versucht werden so liefern sich beide
Zauberer ein Wettstreit auf Willenskraft, der Zauber des Siegers bleibt bestehen. Ansonsten kann mit passenden
Gegenzaubern versucht werden die Magie auszulöschen. Am einfachsten ist ein AN MAGI. Sobald der Gegenzauber
erfolgreich gewirkt wurde gehen die beiden Zauberer in einen Willenskraftwettstreit ob der Zauber bestehen bleibt
oder aufgelöst wird. Folgendes ist dabei noch zu beachten:

\begin{itemize}
\item Bei Gegenzauber mit niedriger Effektstärke bekommt der Originalzauber einen Willenskraftbonus in Höhe der Differenz der Effektstärken.
\item Für jeden Aspekt den der Gegenzauber weglässt bekommt der Originalzauber einen Punkt Willenskraftbonus.
\item Für fehlende Farbe oder Detailgrad der Farbe bekommt der Gegenzauber einen Willenskraftbonus von einem Punkt (bei deutlicher Diskrepanz sollte der Spielleiter mehr vergeben ORT AN AUM, gegen ein ORT AUM HUMI KONIU sollte eher drei Punkte sein da hier der Komplexität von HUMI nicht Rechnnug getragen wird).
\item Ein exakter Gegenzauber der entweder Form oder Farbe mit der AN Rune ins Gegenteil verkehrt und die gleiche Effektstärke hat neutralisiert den Zielzauber ohne das ein Willenskraftwettstreit stattfindet.
\end{itemize}

\begin{mdframed}[hidealllines=true, backgroundcolor=black!10]
\subparagraph{Spielleiterhinweis}

Generell ist die Idee das gute und komplexe Zauberbeschreibungen besser wirken, das ist auch bei Gegenzaubern der
Fall. Bei den Willenskraftboni für den Originalzauber muss deswegen mit Augenmass gemessen werden. Die Regeln oben
dienen als Orientierung können aber möglicherweise je nach Kampagne schwierig sein (wenn die Charaktere keine
Möglichkeiten der Kostenreduktion haben und sich gegen viele Effekte zur Wehr setzen müssen, kann es sein das sie
scheitern wenn sie nicht auch mit einfachen Zaubern sich zur Wehr setzen können).

Permanent gemachte Zauber sollten nicht zerstört sondern nur für die Wirkdauer des Gegenzaubers unterdrückt werden.
Als interessante Regelalternative kann hier die Willenskraft des Gegenstandes statt des Herstellers verwendet werden,
das heisst Verzauberungen bleiben bestehen sind aber sehr einfach zu unterdrücken.

Eine weitere Regelalternative ist die Komplexität von ORT AN MAGI ist automatisch die selbe wie des aufzulösenden
Zaubers. Es wird dann nur ein Willenskraftwettstreit ohne irgendwelche Boni gemacht. Weitere Antimagie sollte es dann
nicht geben.

\end{mdframed}
\begin{center}
\subsection{Kurzregeln und Beispiele}
\end{center}

Um einen schnellen Überblick über das Magiesystem zu bekommen:

\begin{itemize}
\item Magische Effekt bestehen aus Runenworten.
\item Es gibt Form und Farbe. Die eigentlich Form kommt vor der Farbe. Ohne Form gibt es nur eine primitive Hervorrrufung der Farbe.
\item Zauberbeginnworte modifizieren den Ort wo sie wirken, kein Zauberbeginnwort bedeuted dabei Berührungsreichweite.
\item Zauberbeginnwort können auch die Domäne der Magie ändern (Beschwörung statt herbeirufung).
\item Zauberendwort oder besser Zauberkontrollwörter spezifizieren oder erweitern die Zauberbeschreibung, sie kommen nach der Farbe.
\item Farbworte bestimmen das was. Sie können nicht beliebig gemischt werden, aber von der kompliziertesten Farbe beginnend mit zusätzlichen Farbworten genauer beschrieben werden (FERA NOX kann so z.B. Raubtier sein).
\item Zauberabstossung. Keine ''Shotgun'' Effekte. Man kann von einem Zauber nur einmal betroffen sein (man kann trotzdem mehrere Feuergeschoss aus unterschiedlichen Quellen abbekommen, oder Schaden über mehrere Runden bekommen). Nur ein Attributsstärkungszauber pro Ziel erlaubt.
\item Zauberauslöschung. Gegensätzlich Elemente löschen sich aus genauso wie z.B. FLAM und AN FLAM. Bannmagie und Zauberauflösung ist so möglich.
\end{itemize}

Beispiele

\begin{itemize}
\item ORT TEL FLAM (Fern, Geschoss, Feuer) Angriffszauber einzelnes Ziel
\item ORT TEL FLAM AUM, Derselbe Angriffszauber, Zauberkontrolle durch nachgestelltes AUM, der Zauber trifft besser
\item ORT TEL FLAM LEKO, (Fern, Geschoss, Feuer, Freisetzung) klassischer Feuerball
\item ORT LEP FLAM KONIU (Fern, Strahl, Feuer, Aufrechterhalten) Flammenpeitsche die mehrmals verwendet werden kann
\item UR FLAM (Schild, Feuer) Feuerschild das alle Arten von Schaden abhält, dabei anfällig für wasserbasierende Effekte (doppelter Schaden, Auslöschung) ist aber hilfreich gegen Feuereffekte (halber Schaden, Abstossung)
\item ORT VAS UR EBOR KONIU, (Fern, Verstärkt, Wand, Stein, Aufrecherhalten) Steinwand erschaffen. VAS vergrössert hier die Größe der Wand
\item ORT AUM SICR KONIU (Fern, Kontrolle, kleiner Gegenstand, Aufrechterhalten) Telekinesezauber. Kraft des Zaubers hängt von der Effektstärke ab
\item VAS SCIEN MAGI (Verstärk, tWahrnehmung, Magie) Verbesserte Magiewahrnehmung um Spuren von Magie finden zu können.
\item SYMA NOX KONIU (Illusion, Dunkelheit, Aufrechterhalten) Tarnzauber
\item SYMA FERA LITAX KONIU (Illusion, Tier, Luft, Aufrechterhalten) Ein Tier der Lüfte als Illusion darstellen
\item VAS AUM LITAX (Verstärkt, Kontrolle, Luft) Luftströme sehr genau lenken (Die Menge und Kraft wird über die Effektstärke geregelt)
\item ORT UR EBOR KONFAR (Fern, Schild/Wand, Erde/Stabilität, Stärkung) Eine Wand stabiler machen.
\item BET PRIX KONIU (Teilaspekt, Licht, Aufrechterhalten) Lichtzauber
\item GLADI FLAM KONFAR (Waffe, Feuer, Stärkung) Waffe verzaubern damit sie Bonussschaden macht.
\end{itemize}

Gegenbeispiele von ungültigen Zaubern

\begin{itemize}
\item ORT TEL FLAM LITAX. Vermischung Luft und Feuer nicht erlaubt (würde mit CONDUC gehen: ORT TEL FLAM CONDUC LITAX)
\item ORT FLAM TEL. Falsche Reihenfolge, Form dann Farbe
\item KONFAR FLAM. KONFAR ist ein Zauberendwort kein Formwort
\item TEL LEKO FLAM ORT, Rehenfolgen nicht richtig. LEKO muss ans Ende und ORT an den Anfang
\item SYMA NOX FERA, Falsche Reihenfolge der Beschreibung (könnte aber als Hausregel erlaubt sein)
\end{itemize}

Weitere Beispiele. Diese sollten möglicherweise diskutiert und nur mit Zustimmung der Spielleitung verwendet werden

\begin{itemize}
\item HORA CONDUC ORT HORA (Raum, Verbinden, Fern, Raum) Teleportations oder Torzauber. CONDUC verbindet hier zwei gleichfarbige Zauber
\item UR MAGI KONIU CONDUC UR MAGI KONFAR. Verstärktes Schild das eigene Rüstung hat. Das ist ein Beispiel für einen mehrstufigen komplizierten Zauber. Spielrunden können auch einfach festlegen solche Zauber nicht zuzulassen. In diesem Fall könnte der zweite Zauber (Stärkung des Schildes) einfach in Folge gesprochen werden
\item HUMI MUTAR FERA KONIU. Verwandlungszauber von Mensch zu Tier. Variante mit KONFAR kann auch möglich sein.
\end{itemize}

\begin{center}
\subsection{Übersicht über die wichtigen Runen}
\end{center}

Im nachfolgenden sind die Runen kurz in einer tabellarischen Übersicht zusammengefasst. Die Tabelle beinhaltet
die Rune, ihre Komplexität und an welcher Stelle im Zauber sie Verwendung findet. Dazu gibt es eine Kurzbeschreibung
sowie einen Beispielzauber. Für genauere Informationen ist im Abschnitt der jeweiligen Rune nachzulesen.


\begin{footnotesize}
\begin{tabular}{|m{1.7cm}|m{0.3cm}|m{1cm}|m{1cm}|m{1cm}|m{1cm}|m{3cm}|m{5cm}|}
\hline
\textbf{Rune}&\textbf{K}&\textbf{Form}&\textbf{Farbe}&\textbf{Begin}&\textbf{Ende}&\textbf{Beschreibung}&\textbf{Beispiel}\\
\hline
\hline
ORT&1&&&X&&Fernzauber (nicht Berührung)&ORT TEL FLAM, Fernkampfgeschoss\\
\hline
OBSEK&1&&&X&&Beschwörung/Ruf&OBSEK FLAM, Feuerbeschwörung\\
\hline
\hline
AUM&1&X&&&X&Kontrolle&ORT AUM SICR, Telekinese\\
\hline
SCIEN&0&X&&&&Wahrnehmung&SCIEN MAGI, Magiewahrnehmung\\
\hline
SYMA&0&X&&&&Illusion&SYMA FERA KONIU, Tierillusion\\
\hline
TEL&1&X&X&&&Geschoss&TEL FLAM KONFAR, Feuerpfeilverzauberung\\
\hline
LEP&1&X&&&&Strahl&ORT LEP POR, Blitzstrahl\\
\hline
UR&1&X&X&&&Schild/Wand&ORT SYMA UR WAKU KONIU, Wasserwandillusion\\
\hline
LEKO&2&X&&&X&Freisetzung, Explosion&ORT TEL FLAM LEKO, Feuerball\\
\hline
\hline
KONFAR&1&&&&X&Binden, Stärkungszauber&PRIX KONFAR, Intelligenzbuff\\
\hline
KONIU&!&&&&X&Anhaltender Effekt&BET PRIX KONIU, Lichtzauber\\
\hline
\hline
AN&1&X&X&&&Umkehr, Gegenteil&ORT AN MAGI, schwacher Gegenzauber\\
\hline
BET&-2&X&X&&&Verkleinern, Teilaspekt der nächsten Rune&BET FLAM KONFAR, Wärmeverzauberung\\
\hline
CONDUC&0&!&!&&&Verbindet Rune vor CONDUC mit nachfolgender&WAKU CONDUC LITAX, Eis\\
\hline
MANARE&1&!&&&&Leiten&ORT MANARE MANI, Leben entziehen\\
\hline
MUTAR&0&!&&&&Verwandeln&HUMI MUTAR FERA KONFAR, Mensch zu Tier verwandeln\\
\hline
VAS&2&X&X&&&Vergrößern, Verstärken der nächsten Rune&ORT VAS AN FLAM, Brand löschen\\
\hline
INGVA&4&X&X&&&wie VAS nur mehr&INGVA AUM EBOR KONIU, Skulpturen formen\\
\hline
\hline
EBOR&1&&X&&&Erde/Stein&OBSEK UR EBOR, Erdmauerbeschwörung\\
\hline
FLAM&1&&X&&&Feuer&BET FLAM, Anzünden\\
\hline
LITAX&1&&X&&&Luft&ORT AUM LITAX KONIU, Luftstrom\\
\hline
WAKU&1&&X&&&Wasser&ORT AUM WAKU KONIU, Wasser bändigen\\
\hline
NOX&1&&X&&&Dunkelheit&NOX KONFAR, Willenskraftstärkung\\
\hline
PRIX&1&&X&&&Licht&ORT PRIX, Lichtblitz\\
\hline
MAGI&1&&X&&&Magie&UR MAGI KONIU, Magieschild\\
\hline
MANI&2&&X&&&Leben&MANI, Heilungzauber\\
\hline
MORT&2&&X&&&Tod&SCIEN MORT KONIU, Krankheit/Gift spüren\\
\hline
\hline
FERA&4&&X&&&Tier&SCIEN FERA KONIU, Tiere aufspüren\\
\hline
FLORA&3&&X&&&Pflanze&ORT AN FLORA LEKO, Unkraut vernichten\\
\hline
SICR&3&&X&&&Gegenstand&ORT AUM SICR KONIU, Gegenstandslevitation\\
\hline
GLADI&2&&X&&&Waffe&GLADI FLAM KONFAR, Flammenschwertverzauberung\\
\hline
\hline
ANIMA&6&&X&&&Geist/Seele&OBSEK ANIMA MORT, Totengeist beschwören\\
\hline
CENSA&5&X&X&&&Essenz, bezieht sich nachfolgende Rune&ORT MANARE CENSA MAGI, Magie stehlen\\
\hline
HUMI&6&&X&&&Mensch&ORT VAS AUM HUMI KONIU, Gedankenkontrolle\\
\hline
HORA&6&&X&&&Raum&ORT AUM HORA, Raumkrümmung\\
\hline
TORA&6&&X&&&Zeit&SCIEN VAS TORA KONIU, Vergangenheitsblick\\
\hline
DIVIN&6&X&X&&&Erschaffung, gottgleiche Verstärkung&DIVIN HUMI CONDUC MANI, Wiederbelebung\\
\hline
TRIA&4&&&X&&Verdreifacht den Zauber&TRIA ORT TEL FLAM, 3 Feuergeschoße\\
\hline
PENTA&6&&&X&&Verfünffach den Zauber&HEPTA ORT UR MAGI KONIU, 5 magische Schilde\\
\hline
HEPTA&8&&&X&&Versiebenfacht den Zauber&HEPTA ORT FIR INGVA FIR KONIU, Eissturm mit 7 Zentren\\
\hline
\hline
ES 1&1&-&-&-&-&Effektstärke 1&Verwendung für Schaden, etc\\
\hline
ES 2&3&-&-&-&-&Effektstärke 2&Verwendung für Schaden, etc\\
\hline
ES 3&6&-&-&-&-&Effektstärke 3&Verwendung für Schaden, etc\\
\hline
ES 4&10&-&-&-&-&Effektstärke 4&Verwendung für Schaden, etc\\
\hline
ES 5&15&-&-&-&-&Effektstärke 5&Verwendung für Schaden, etc\\
\hline
\end{tabular}
\end{footnotesize}

\begin{center}
\subsection{Runen}
\end{center}

Im folgenden werden die magischen Runen beschrieben und mit eine paar Beispielzaubern versehen. Bei Farbworten wird
mehr auf die Effektveränderung (Stärkung, Illusion, etc.) eingegangen. Die Formrunen beschreiben dann ausführlicher
mögliche Zaubersprüche. Der angebene Kreis hat keine regeltechnische Bedeutung und dient zur Beschreibug der
Mächtigkeit der Rune. Magie des 7. Kreises, Erzmagier des 7. Kreises, etc.

\begin{mdframed}[hidealllines=true, backgroundcolor=black!10]
\subparagraph{Spielleiterhinweis}

Die Kombination von Runen zu Zaubern sollten einem gewissen Schema folgen. Ist dieses Schema gefühlt unlogisch so
kann mittels Hausregeln eine neue Grammatik etabliert werden. Z.B. könnte das KONFAR - Wort den selben Effekt auch
als Formwort statt als Zauberendwort hervorrufen. Das hat aber möglicherweise Einfluß auf die Spielbalanz (die meist
überbewertet ist). Mögliche Einschränkung wären das Stärkungszauber aufrechterhalten werden müssen (KONIU),
allerdings sind die ''Kosten'' für ''ewiges'' aufrechterhalten einfach zu bewerkstelligen. Ausserdem werden mit LEKO
Flächenstärkungen möglich.

Ein andere Problem sind komplexe Zaubersprüche, diese alle abzudecken ist nicht möglich. Hier sollte mit gesunden
Menschenverstand geurteilt werden (auch unter Gesichtpunkten wie Wirkung und Kosten).

Regeländerungen oder Erweiterungen sollten schriftlich notiert werden und als Referenz sowohl für den Spielleiter
als auch für die Spieler dienen, so das sich innerhalb einer Kampagne die Magie konsistent anfühlt!

Komplexität vs Kreis. Der Kreis hat in dem vorgeschlagenen Regelwerk keinen Einfluss. Die Idee ist es Farbworte
von der primären/elementaren Ebene bis hin zu vernunftbegabten Lebewesen zu kategorisieren. Falls es für die
Spielwelt anders benötigt wird kann hier beliebig geändert werden.

\end{mdframed}
\subsubsection{ANIMA (Geist, Seele)}

Farbrune, Schicksal, Kreis: 7, Komlpexität: 6

Beschreibt übernatürliche Wesenheiten (Geister, Gespenster, körperlose Seelen).

\begin{itemize}
\item Stärkungszauber: Kann Boni des Übernatürlichen geben (z.B. Glück).
\item Schadenszauber: Mit AN kann die Seele des Ziels verkrüppelt werden.
\item Illusionszauber: Kann übernatürliche Wesenheiten darstellen.
\item Kontrollzauber: Kann übernatürliche Wesenheiten manipulieren oder in den Dienst zwingen.
\item Wahrnehmungszauber: Kann zum aufspüren übernatürlicher Wesenheiten und Effekte verwendet werden.
\item Herbeirufungszauber: Ruft einen Geist/Geistwesen hervor sofern die Spielwelt das zulässt. (OBSEK ANIMA)
\end{itemize}

\subsubsection{AN (Umkehr, Negation)}

Spezialrune, Komplexität: 1

Kehrt das nachfolgende Wort um. Kann auf alle Runen angewendet werden. Meist werden damit Gegenzauber beschrieben. Die
Regeln für Gegenmagie befinden sich im Kapitel Antimagie in den Magieregeln. Bei den Farbworten ist zu beachten das
bei einigen Effekten es einen Unterschied macht ob das gegenteilige Elemente oder das negierte Element genutzt wird.
So wird ein ORT AN FLAM genauso gut Feuer bekämpfen wie ein ORT WAKU. Aber ein AN FLAM KONFAR senkt die Stärke während
ein WAKU KONFAR die Geschicklichkeit steigen lässt.

\begin{itemize}
\item Stärkungszauber: Wird entweder das KONFAR oder das Farbwort negiert so wird es ein Schwächungszauber. Anstatt eines Bonus gibt es einen Malus.
\item Schadenszauber: Haben im Prinzip eine ähnlich Wirkung. Sollte aber ein passendes Farbwort ins Gegenteil gekehrt werden so mach der Zauber doppelten Schaden (z.B. ORT TEL AN FERA). Diese Schaden kann auch katastrophal sein wenn weitere Verstärkung wie das CENSA Wort verwendet werden.
\end{itemize}

\subsubsection{AROL}

Zauberendrune.

Diese Rune hat keinen Enfluss auf den Effekt. Sie senkt die Kosten des Zaubers um 2 lässt dafür aber seine
Schwierigkeit um 2 steigen. Sie kann beliebig ans Ende des Zaubers gesetzt werden.

\subsubsection{AUM (Kontrolle)}

Formrune, Zauberendword, Komplexität: 1

Ermöglicht die Kontrolle über die beschriebene Farbe. Dabei kann die unverstärkte Kontrolle als Kraftausübung
verstanden werden. Es können so z.B. Tunnel in die Erde gedrückt werden. Wasser kann irgendwo hinfliessen und in
einfache Formen gedrückt werden. Ein Luftstrom kann erzeugt werden oder ein Gegenstand kann durch die Luft schweben.
Verstkärkt man die Kontrolle mit VAS so ist die Kontrolle auch in dem Ziel möglich. Eine Erdskulptur könnte so mit
ORT VAS AUM EBOR KONIU erschaffen werden. Ein Gegenstand der so kontrolliert wird dessen interne mechanische Mechanismen
werden so mitkontrolliert (so könnte man ein Kiste öffnen eine Klinke herunterdrücken oder einen Schlüssel drehen mit
einem ORT VAS AUM SICR KONIU). Mit INGVA verstärktes AUM kann feingranulare Bewegungen und Formen oder feine Eingriffe
in das innere ermöglichen. Z.B. ORT INGVA AUM EBOR KONIU würde nicht das Erschaffen einer Skulptur ermöglichen sondern
dieser auch feine Details wie Gesichtszüge, Narben, Falten oder Haare geben können. Prinzipiell ist der Einsatz von
Fertigkeiten mit INGVA AUM möglich (z.B Schloss auf Entfernung knacken). Sollte der Zauber als Hilfe verwendet werden
(z.B. INGVA AUM FERRUM KONIU beim Schmieden) so kann der Spielleiter einen Bonus in Höhe der Effektstärke (oder weniger)
auf den Fertigkeitswurf geben.

Die Effektstärke kann hier für die Kraft oder Geschwindigkeit der Kontrolle herangezogen werden. Verstärkungsworte
für die Farbe beschreibt die Menge. Das ist bei vielen Effekten Multiplikativ. Der Spielleiter ist angehalten Buch
über die Wirkungen zu führen. So könnte z.B. festgelegt sein das ein AUM LITAX KONIU ca 20kg pro Effektstärke
schweben lassen kann (und ein AUM VAS LITAX KONIU könnte so mit Effektstärke 2 ca 80kg schweben lassen).

Die AUM Rune kann auch als Zauberendwort (Zauberkontrollwort) verwendet werden. Sie gibt nachträglich Kontrolle über
den Zauber. So kann z.B. ein ORT SYMA FERA KONIU AUM Kontrolle über die Tierillusion geben. Diese Illusion kann so z.B.
auf Situationen reagieren in dem der Zauberer sie z.B. Drohgebärden machen lässt. Verstärkt man die AUM Rune am Ende
mit VAS oder INGVA so wird die Kontrolle genauer. AUM und LEKO am Zauberende widersprechen sich eigentlich aber das
Setting könnte so z.B. einem ORT TEL FLAM LEKO INGVA AUM erkauben befreundete Ziele im Wirkungsbereich zu verschonen.

Die Kontrolle über Lebewesen mittels AUM Rune ist etwas anders geartet. Zwar kann AUM FLORA wie ein primär Element
behandelt werden (Ranken und Gebüsche bewegen) allerdings werden die betroffenen Pflanzen auch auf Wunsch ein Stück weit
wachsen. Die Kontrolle über Mensch und Tier betrifft immer das Innenleben, abhängig von der Stärke der Kontrolle kann
das Innenleben modifiziert werden. So kontrolliert (hier am Beispiel Mensch, analog aber auch Tiere, Geister, etc.):

\begin{itemize}
\item AUM HUMI das Gefühlsleben (Änderung in Höhe der Effektstärke möglich).
\item VAS AUM HUMI die Gedankenwelt (Änderung in Höhe der Effektstärke, Gedanken werden schnell wieder normal nach Ablauf des Zaubers).
\item INGVA AUM HUMI die Erinnerung. Hier ist ein langfristiger Effekt möglich, allerdings repariert sich die Erinnerung selbst wenn der Betroffene Situationen ausgesetzt ist die einen Konflikt in seinen Erinnerungen auslöst. Mit KONIU und hohen Effektstärken könnten mehrere Erinnerungen stark beeinflusst werden um solche Situation zu vermeiden.
\item CENSA AUM HUMI das Schicksal. Hier wäre z.B. ein einfacher ''Wenn, Dann'' Fluch möglich oder je nach Effektstärke kann der Betroffene zu einer Aufgabe gezwungen werden.
\end{itemize}

Ein paar Beispiele:

\begin{itemize}
\item AUM EBOR KONIU: Einen Weg durch Erde und Gestein bahnen. (Es ist empfohlen das ein Kubikmeter Erde so pro Runde bewegt wird, weniger bei festem Gestein)
\item AUM NOX KONIU: Lässt die Dunkelheit aus den Schatten kriechen (Es wird weniger ausgeleuchtet, das Verstecken ist einfacher. Die Wirkung ist subtil und sollte z.B. Umstandsboni fürs Verstecken geben)
\item VAS AUM VAS LITAX: Fliegenzauber der mittels Luftströme funktioniert (Effektstärke 2 sollte für einen normalen Menschen knapp reichen)
\item VAS AUM HUMI KONIU: Übernimmt die Steuerung über das Ziel. Falls die Handlung dem Bezauberten zutiefst zuwider sind so kann es eine neue Widerstandsprobe durchführen.
\item ORT TEL FLAM VAS AUM: Nachgelenktes Geschoss. Das Zauberergebnis ist um 2 erhöht wenn es gegen die Verteidigung des Ziel gerechnet wird.
\end{itemize}

\subsubsection{Bet (Verkleinern/Teilaspekt)}

Spezialrune, Komplexität: -2

Verkleinert die nachfolgende Rune oder bezieht sich auf einen vom Zaubernden festgelegten Teilaspekt. So könnte mit
SYMA BET FLORA die Illusion des Geruchs von Blumen erschaffen werden ohne das man die ganze Pflanze als Illusion
benötigt. Ein BET FERA könnte aber auch einfach nur ein kleines Tier beschreiben.

Auf Formrunen angewandt weden diese stark eingeschränkt. Ob ein funktionierender Zauber zustande kommt legt die
Spielleitung anhand des Settings fest.

Ein paar Beispiele:

\begin{itemize}
\item ORT SYMA FERA LITAX KONIU, Eine Vogelillusion
\item BET SCIEN MAGI KONFAR, Zauber der für eine gewisse Zeit ein Magiegespür verleiht.
\item BET FLAM, Entzündenzauber
\item BET FLAM KONFAR, Wärmezauber der z.B. gegen drohende Unterkühlung hilft.
\item OBSEK BET ANIMA MORT KONFAR, schwachen Totengeist binden (hilfreich für die Erschaffung dummer Zombies)
\end{itemize}

\subsubsection{CENSA (Essenz)}

Farbrune, Formrune, Verstärkung, Komplexität: 5

Essenzmagie macht normale flüchtige Herbeirufungen oder Manipulationen zu realeren. Wie diese sich von normaler
Magie unterscheidet regelt das Setting. Es ist empfohlen die Energiekosten andersartig zu gestalten um den
zusätzlichen Kraftaufwand gerecht zu werden. Essenzmagie kann Substanzen erschaffen (CENSA WAKU erschafft echtes
Wasser) oder Lebewesen oder Gegenstände mit ihrer ureigensten Energie aufladen und so reparieren oder zerstören.

Es wird auch empfohlen das die Energiekosten erst dann regenerieren wenn der Zaubereffekt beendet (gebannt) ist.

Wie immer ist der Spielleiter angehalten Regeln für sein Setting zu erstellen und im Laufe des Spiels zu dokumentieren.
Eine Möglichkeit ist es die CENSA Rune vor den Spielercharakteren verborgen zu halten um weitere magische Effekte
zu haben die überirdisch erscheinen. Hier ein paar Beispiele wie die CENSA Magie gedacht ist:

\begin{itemize}
\item CENSA FLAM, Ruft elementares Feuer herbei das eine Zeit existieren wird. Kann zum Schmelzen von Metall verwendet werden.
\item CENSA MANI, Kann Verkrüppelungen heilen. Mit dem richtigen Weg kann so auch ein Körperteil wiederhergestellt werden.
\item MANARE CENSA MAGI, Könnte die grundlegende Energie eines Gegenstands absaugen und ihn so nichtmagisch machen.
\item ORT TEL CENSA AN MANI, Wenn es wirklich weh tun soll. Dieser Zauber wird katastrophalen und schwer heilbaren Schaden bei Lebewesen verursachen.
\item HUMI CENSA MUTAR FERA, Verwandlungsmagie die dank des CENSA anhält (permanent?). Ein KONIU ist nicht nötig
\item FLAM CENSA KONFAR. Sehr lang anhaltender Stärke verbessern Zauber. Andere Interpretationen wären auch möglich, so könnte der Zauber dem Ziel eine Feuerpägung geben die das zukünftige Zaubern von Feuermagie vereinfacht.
\item CENSA FLAM KONFAR. Sehr gefährlicher Zauber. Je nach Setting könnten neben Stärke weitere Boni wie Feuerimmunität Schadensaura, Nahkampfschaden dazu kommen. Allerdings wird die Ausrüstung des Charakters leiden und evtl auch er.
\end{itemize}

\subsubsection{CONDUC}

Spezialrune, Kreis: 3, Komplexität: 0

Verbindenrune. Diese Formrune kann zwei passende Runen zusammenziehen. Sie steht im Zauber zwischen den beiden Runen.
Beispiel:

\begin{itemize}
\item ORT TEL FLAM CONDUC LITAX: Feuer und Luft werden hier zu Blitz verbunden.
\item SYMA WAKU CONDUC LITAX: Wasser und Luft wird hier zu Eis verbunden.
\item ORT LEP CONDUC LEKO FLAM: Strahl und Freisetzung wird hier verbunden um einen kegelförmigen Angriffszauber zu erzeugen.
\end{itemize}

CONDUC ist auch als die Namensrunen bekannt. Magische Forschung versucht Effekte oder Farben aus den primären
Elementen mittels CONDUC zusammenzusetzen. So kann Metall zum Beispiel als FLAM CONDUC EBOR CONDUC EBOR beschrieben
werden. Gold könnte z.B. durch weiteres hinzufügen von PRIX beschrieben werden. Sollte ein solche Farbe erfolgreich
beschrieben werden so kann der Magier mit längerer Forschung den magischen Namen dieser Verbindung in Erfahrung
bringen. So ist FLAM CONDUC LITAX durch die Rune POR beschrieben. Dieser Name reduziert die Zauberlänge und evtl. auch
die Komplexität.

Hier ein paar Mischrunen. Je nach Kampagne können und sollten andere Interpretationen möglich sein.


\begin{small}
\begin{tabular}{|m{4cm}|m{3cm}|m{3cm}|m{3cm}|}
\hline
\textbf{ursprüngliche Farben}&\textbf{Runenwort}&\textbf{Komplexität}&\textbf{Beschreibung}\\
\hline
\hline
EBOR, FLAM&COC&2&Magma\\
\hline
EBOR, LITAX&HAREN&2&Sand\\
\hline
EBOR, NOX&SIDIAN&2&Obsidian\\
\hline
EBOR, PRIX&MARM&2&Marmor\\
\hline
EBOR, WAKU&LUTUM&2&Schlamm\\
\hline
FLAM, LITAX&POR&2&BLitz\\
\hline
FLAM, NOX&CINIS&2&Asche\\
\hline
FLAM, PRIX&SOL&2&Sonne\\
\hline
FLAM, WAKU&VAPOR&2&Dampf\\
\hline
LITAX, NOX&UMBRA&2&Schatten\\
\hline
LITAX, PRIX&CAEL&2&Himmel\\
\hline
LITAX, WAKU&FIR&2&Eis\\
\hline
NOX, PRIX&LUNA&2&Mond, Zwielicht\\
\hline
NOX, WAKU&NEB&2&Nebel\\
\hline
PRIX, WAKU&NUBI&2&Wolke\\
\hline
EBOR, EBOR, FLAM&FERRUM&2&Metall, Eisen\\
\hline
FLAM, LITAX, EBOR&METEO&3&Meteor\\
\hline
WAKU, LITAX, EBOR&AVE&3&Hagel\\
\hline
LITAX, FLAM, WAKU&TURBO&3&Tornado, Sturm\\
\hline
EBOR, FLAM, WAKU&MOTUS&3&Erdbeben\\
\hline
\end{tabular}
\end{small}

\begin{mdframed}[hidealllines=true, backgroundcolor=black!10]
\subparagraph{Spielleiterhinweis}

Wenn Neue Farbrunen erschaffen werden so sollten sie (eingekürzte) passende lateinische Worte verwenden, die 
existierenden Runen haben sich daran orientiert. Sollte zuviel Beschreibung in die Namensrune eingeflossen sein
so kann es sein das die Komplexität unverhältnismässig hoch ist. Hier sollte die Komplexität vom spielleiter auf
ein mit bereits existierenden Farbrunen vergleichbares Mass reduziert werden.

Mit der CONDUC Rune wäre es auch möglich Zauber oder Teilzauber zu verbinden. Denkbare Szenarien die aber von der
Spielleitung für die Welt vorgesehen sein müssen:

\begin{itemize}
\item ORT LEP CONDUC LEP POR: Kettenblitz der von einem Gegner zu einem weiter springt. (-2W und zwei Proben sind fällig)
\item ORT HORA CONDUC ORT HORA KONIU: Verbindet zwei Raumpunkte mittels eines Portals (Teleportation von allem das durchs Portal schreitet).
\item SYMA PRIX KONIU CONDUC SYMA AN MAGIE: Unsichtbarkeitszauber der von einfachen SCIEN Zaubern nicht aufgespürt wird
\item MANARE MANI CONDUC MANI MUTAR MAGI CONDUC AUM UR MAGI: Zauber der Lebensenergie in magische Energie umwandelt die zum reparieren eines magischen Schilds verwendet wird.
\end{itemize}

Das letzte Beispiel zeigt relativ komplexer Magie. Der Zauber wäre als Verzauberung oder KONFAR Magie denkbar. Dann
würde z.B. verursachter Waffenschaden ein Schutzschild aufladen. Bei dieser hochkomplexen Magie ist aber die Frage
wird sie für die Spielwelt gebraucht? Ist sie den Spielern zugänglich (kann hinter Komplexität verborgen sein oder
ist nur als Großmagie effektvoll, etc.). Desweiteren muss der Spielleiter bestimmte Sachen festlegen. Hier z.B.
welcher Schaden lädt das Schild wie auf? Wie schnell kann Schaden gemacht werden?

Weiter Ideen sind Schnee (FIR CONDUC LITAX).

\end{mdframed}
\subsubsection{DIVIN (göttliche Magie)}

Formrune, Farbe, Verstärkung, Komplexität: 6+

Beschreibt Gottgleiche Magie. Diese Magie kann wirklich erschaffen oder endgültig zerstören. Sie sollte dem Spielleiter
vorbehalten sein. Die Interpretation der Wirkweise kann weitergefasst sein als dies bei normaler Magie der Fall ist.
Die Kosten göttlicher Zauber sollten etwas permanentes beinhalten (genaueres regelt das Setting). Als Verstärkerrune
kann sie verwendet werden um Großmagie zu beschreiben. (ORT TEL DIVIN FLAM DIVIN LEKO um eine kleine Stadt zu vernichten)
Mit ihr können Landschaften umgestaltet werden oder die Toten zurück ins Leben geholt werden. Evtl ist nur göttliche
Magie in der Lage einige Wesenheiten in in ihre Schranken zu versetzen.

Ein paar Ideen bzw. Beispiele:

\begin{itemize}
\item HUMI DIVIN MANI: Tote zum Leben erwecken.
\item DIVIN AUM CENSA WAKU KONIU: Fluss versetzen (permanente Wirkung)
\item DIVIN AN ANIMA: Auslöschung eines Geistes.
\item DIVIN SCIEN DIVIN MAGI: Einen Eindruck davon bekommen wie die Welt hinter dem Schleier ist...
\item DIVIN CENSA EBOR CONDUC MANI: Land wieder fruchtbar machen.
\end{itemize}

\subsubsection{EBOR (Erde)}

Farbrune, Basis, Kreis: 1, Komplexität: 1

Erde und Gestein sowie feste Stoffe allgemein werden duch diese Rune beschrieben

\begin{itemize}
\item Stärkungszauber: Erhöht die Konstitution.
\item Schadenszauber: Verursacht normalen körperlichen Schaden.
\item Teilaspektzauber: Kann Stabilität vorgauckeln und kurzfristig einen möglichen Zusammenbruch verschieben.
\item Illusionszauber: Feste Stoffe wie Felsen oder Wände können erzeugt werden. Oberflächlich wird auch der Tastsinn leicht getäuscht (hier ist die Effektstärke entscheidend, ein unabsichtliches Durchschreiten der Illusion wird nicht verhindert).
\item Kontrollezauber: Erde bewegen um Löcher oder Tunnel zu erschaffen. Bewegte Erde ist aber zu langsam für effektive direkt Angriffe.
\item Wahrnehmungszauber: Gestein sehen. Ermöglicht es z.B. in begrenzten Umfang Hohlräume in der Erde / Gestein zu finden.
\end{itemize}


\subsubsection{FERA (Tier)}

Farbrune, Natur, Kreis: 5: Komlpexität: 4

Beschreibt Tiere. Durch Ergänzung von weiteren Runen können Tiere besser beschrieben werden (FERA LITAX könnten z.B.
Vögel sein). Die Größe des beschriebenen Tieres hängt von der Effektstärke ab.

\begin{itemize}
\item Stärkungszauber: Gibt eine Fähigkeit des Tieres (wie z.B. verbesserten Geruchssinn).
\item Schadenszauber: Mit Kombination mit der AN Rune sind tierische Wesen anfällig.
\item Teilaspektzauber: Mit BET könnten kleinere Tiere wie Insekten beschrieben werden.
\item Illusionszauber: Stellt Tiere dar. Mit nachgestellten AUM sind auch komplexere Bewegungen möglich.
\item Kontrollzauber: Kann Tiere kontrollieren.
\item Wahrnehmungszauber: Findet Tiere. Mit Teilaspekt wird der Gemütszustand des Tieres angezeigt.
\end{itemize}

\subsubsection{FLAM (Feuer)}

Farbrune, Basis, Kreis: 1, Komplexität: 1

Beschreibt das Feuer und Wärme. Im wissenschaftlicheren Sinne sind damit Plasmen und exotherme Reaktionen gemeint.

\begin{itemize}
\item Stärkungszauber: Erhöht die Stärke.
\item Schadenszauber: Verursacht normalen körperlichen (Energie/Feuer) Schaden.
\item Teilaspektzauber: Erzeugt Wärme und kann z.B. gegen Wettereinflüsse schützen.
\item Illusionszauber: Beschreibt Feuer und feurige/flackernde Lichteffekte.
\item Kontrollezauber: Abhängig von der Effektstärke kann ein Feuer vergrößert oder vekleinert werden. Auch Ausbreitungsrichtungen könnten so kontrolliert werden um ein Übergreifen zu unterbinden oder hervorzubringen.
\item Wahrnehmungszauber: Macht das Wärme sehen möglich. Je nach Kampagne hilft es entweder nur beim Aufspüren von Feuerquellen, magischen Feuerenergien/wesen oder ermöglicht Nachtsicht mittels Infrarotsicht.
\end{itemize}

Weitere Beispielzauber:

\begin{itemize}
\item ORT FLAM KONIU: Schadenszauber der ein Bereich blockiert. Auch geeignet zum Kochen oder zum Starten eines größeren Feuers.
\item ORT BET FLAM KONIU: Mit geringer Effektstärke geeignet Fackeln und Kerzen zu entzünden.
\end{itemize}

\subsubsection{FLORA (Pflanze)}

Farbrune, Natur, Kreis: 4, Komplexität: 3

Beschreibt Pflanzen. Durch Kombination mit weiteren Farbworten können die Pflanzen genauer bescrhrieben werden. Die
Effektstärke beschreibt die Größe der Pflanze.

Stärkungszauber: Gibt eine Fähigkeit von Pflanzen. Sofern das gewünscht ist...
Schadenszauber: Mit der AN Rune kann ein effektives Pflanzenvernichtungsmittel erstellt werden.
Teilaspektzauber: Beschreibt Bewuchs wie von Moos.
Illusionszauber: Erstellt Sinnestäuschung von Pflanzen.
Kontrollzauber: Kann Pflanzen bewegen und das Durchkommen durch Unterholz erleichtern. Unaufmerksame Gegner können
von evtl. vorhandenen Schlingpflanzen gefangen gesetzt werden.
Wahrnehmungszauber: Zum Aufspüren von Pflanzen geeignet.

\subsubsection{GLADI (Waffe)}

Farbrune, Gegenstand, Kreis: 4, Komplexität: 2

Beschreibt Waffen und ist somit streng genommen Teil der SICR Rune. Waffen bis zum Kurzschwert können ohne
Verstärkungsrune beschrieben werden. Alle von Menschenhand geführten Waffen werden mit VAS GLADI beschrieben. Noch
größere Waffen benötigen dann INGVA.

Stärkungszauber: Erhöht den Nahkampfschaden.
Schadenszauber: Verursacht physischen Schaden. Mit AN als Zauber zum Entwaffnen denkbar.
Teilaspektzauber: Beschreibt Waffeneigenschaften wie Schärfe.
Illusionszauber: Erzeugt Illusionen von Waffen.
Kontrollzauber: Telekinesezauber, als Kontrollzauber der mittels KONFAR gebunden ist verbessert er die Kampfeigenschaften
Wahrnehmungszauber: Zum aufspüren vonPflanzen geeignet.


\subsubsection{HORA (Raum)}

Typ: Farbrune, Universum, Kreis: 7, Komplexität: 6

Beschreibt den physikalischen dreidimensionalen Raum, Entfernungen und Gravitation. Kann im weiteren Sinne auch für
überdimensionale Beschreibungen (Astralebenen, Parallelwelten, etc.) verwendet werden.

\begin{itemize}
\item Stärkungszauber: Verankert den physikalischen Ort. Mit AN HORA KONFAR kann man sich von Raum Effekten lösen. Das kann z.B. zum Schweben verwendet werden.
\item Schadenszauber: Verursacht bei jeden Wesen katastrophale Schäden.
\item Teilaspekt: Beschreibt Entferungen und kann z.B. zum Entfernungsmessen mittels SCIEN verwendet werden.
\item Illusionszauber: Die HORA Rune kann mit weiteren Runen Großillusionen bzw. Illusionen mit mehr räumliche Tiefe erstellen.
\item Kontrollzauber: Kann die Bewegungsgeschwindigkeit verändern (um einen Multiplikator).
\item Wahrnehmungszauber: Kann multidimensionale Effekte aufspüren oder Raumanomalie oder dazugehörige Wesen und Effekte aufspüren. Kann auch für Fernsicht verwendet werden.
\end{itemize}

\subsubsection{HUMI (Mensch)}

Farbrune, Schicksal, Kreis: 7: Komlpexität: 6

Beschreibt den Menschen und in gewisser Weise Intelligenz oder intelligente Wesen.

\begin{itemize}
\item Stärkungszauber: Erhöht Zugang zu Wissen, Verständnis und Konzentration.
\item Schadenszauber: Kann mit der AN Rune intelligente Wesen verkrüppeln.
\item Teilaspekt: Beschreibt handwerkliche Tätigkeit (sofern das Sinn macht).
\item Illusionszauber: Stellt Menschen dar. Mit nachgestellten AUM kann der Mensch entsprechende Handlungen darstellen.
\item Kontrollzauber: Je nach Tiefe der Kontrolle können Gedanken, Emotionen und Erinnerungen manipuliert werden.
\item Wahrnehmungszauber: Kann zum Finden von Menschen oder dem Lesen von Gedanken und Gefühlen verwendet werden.
\end{itemize}

\subsubsection{INGVA (starke Verstärkung)}

Spezialrune, Komplexität: 4

Diese Rune ist die Fortsetzung der VAS Rune (siehe auch da). Sie bezieht sich auf die nachfolgende Rune. Ist Diese
ein Farbwort so wird es noch deutlich ''größer'' beschrieben. Z.B. kann ein Blumentopf mit SICR ein Stuhl oder Tisch
mit VAS SICR und eine Anbauwand oder eine Kutsche mit INGVA SICR beschrieben werden. Bezieht sich INGVA auf ein
Formwort so wird dieses deutlicher und stärker. So ist eine extrem genaue Kontrolle mit INGVA AUM oder ein sehr
detailliertes Wahrnehmen mit INGVA SCIEN möglich.

Ein paar Beispiele für stark verstärkte Zauber:

\begin{itemize}
\item INGVA SCIEN MAGI. Der Verzauberte ist in der Lage sehr feine Details in den Magieströmungen zu spüren. Er bekommt entweder mehr Bonuswürfel (als z.B. nur INGVA SCIEN MAGI) oder er kann Spuren von Magie wahrnehmen die vor langer Zeit gewirkt wurde. Oder bei magischen Spuren die recht jung sind Details wie Farbe und Form erkennen.
\item ORT INGVA TEL INGVA LITAX. Ein Sehr mächtiges Luftgeschoss. Normalgroße Gegner sollten mit Zustimmung des Spielleiters nach hinten und zu Boden geschleudert werden (Widerstandsprobe mit Stärke).
\item INGVA SYMA FERA. Die Illusion wird realler. Das Tier verhält sich wie ein entsprechendes Lebewesen und reagiert auf andere Charaktere (droht, flieht, etc.)
\item INGVA LEKO: Verdreifacht die Reichweite des Freisetzungseffekt.
\item INGVA LEP, INGVA TEL: Verdreifacht die Reichweite.
\end{itemize}

\subsubsection{KOL}

Zauberendrune.

Diese Rune hat keinen Enfluss auf den Effekt. Sie senkt die Schwierigkeit des Zaubers um 2 lässt dafür aber seine
Kosten um 2 steigen. Sie kann beliebig ans Ende des Zaubers gesetzt werden.

\subsubsection{KONFAR (Binden)}

Zauberendwort (Formwort), Komplexität: 1

Bindet die Struktur und Energie des Zaubers im Ziel. Das verändert die Wirkung stark. Wird ein reiner Farbzauber ins
Ziel gebunden so wirkt der Zauber auf das Ziel verstärkend. Zum Beispiel steigert FLAM KONFAR die Stärke des
verzauberten Charakters um die Effektstärke des Zaubers (In den Farbrunen ist je aufgeführt welche Farbe welchen
KONFAR Effekt hat).

Verdreht man entweder die Farbe oder das KONFAR ins Gegenteil (FLAM AN KONFAR oder AN FLAM KONFAR) so senkt der Zauber
das Attribut entsprechend. Gegen diesen Zauber ist aber Widerstand möglich!

Wirkt man einen Zauber der mit Form und Farbe und KONFAR beschrieben wird, so binden man den Zauber an ein passendes
Objekt. Die KONFAR Rune sorgt für eine weiterführende Wirkung (SYMA FLAM KONFAR könnte für eine anhaltende
Feuerillusion verwendet werden). Die Wirkung kann auch vom Gegenstand abhängen. Ein PRIX KONFAR auf einen Stein zu
sprechen macht ihn nicht intelligenter sonder lässt ihn leuchten (ein EBOR KONFAR macht ihn härter). Hier ist der
Spielleiter gefragt die Wirkung zu beschreiben.

Ein Objekt kann immer nur von einer Art KONFAR Zauber belegt sein.

KONFAR beschreibt auch die Wirkdauer und kann so nicht mit KONIU im selben Zauber stehen.

\begin{itemize}
\item PRIX KONFAR: Intelligenz oder Persönlichkeit des Charakters um Effektstärke verbessern.
\item AUM GLADI KONFAR: Gibt bessere Kontrolle (AUM) über die verzauberte Waffe (GLADI -> bis Größe Kurzschwert) und bindet den Zauber an das Schwert. Die Waffe trifft um die Effektstärke besser.
\item BET FLAM KONFAR: Verzaubert ein Ziel so das es über ein gewisse Zeit Wärme abgibt (BET FLAM macht aus Feuer Wärme).
\item GLADI FLAM KONFAR: Verzaubert eine Waffe das sie zusätzlich Feuerschaden macht. Hier wird GLADI als Formwort verwendet so das jede Waffe verzaubert werden kann (im Vergleich zu AUM GLADI, wo GLADI als Farbe verwendet wird und damit in der Größe eingeschränkt ist).
\item TEL FIR KONFAR: 10 Pfeile können so verzaubert werden das sie zusätzlich Eisschaden in Höhe der Effektstärke machen.
\item AUM FERRUM KONFAR: Verbessert die Fähigkeit des verzauberten Charakters Metalle zu kontrollieren. Boni auf alle Fertigkeitswürfe zur Metallbearbeitung (z.B. Schmieden).
\item AN FLAM KONFAR: Stärke beim Ziel senken. Analog würde auch FLAM AN KONFAR funktionieren.
\item SCIEN MAGI KONFAR: Statt einer Magiesicht bleibt nur ein Magiegespür.
\end{itemize}

\begin{mdframed}[hidealllines=true, backgroundcolor=black!10]
\subparagraph{Spielleiterhinweis}

Mittels der KONFAR Rune sind einige starke Verbesserung für die Spielercharaktere möglich. Es wird deswegen dringend
empfohlen nur eine Verstärkung pro Gegenstand und Charakter zu erlauben. Für verzauberte Gegenstände sollte man diese
Regel möglicherweise aufweichen (Verzauberung ist nicht eingeschränkt, aber höchstens eine weiter Gesprochene kann
dazu kommen). Alternativ kann man Charakteren auch zugestehen eine Attributs und ein Fertigkeitsverbesserung
gleichzeitig zu besitzen. 

Sollten elementare Mischrunen eingesetzt werden (z.B. FIR für Eis das aus WAKU und LITAX zusammengesetzt ist). So
teilt man die erzeugten Attributspunkte zu gleichen Teilen auf die Attribute auf übrig gebliebene werden vom Zaubernden
verteilt. Alternativen:

\begin{itemize}
\item Die Attributspunkte werden aufgeteilt, aber es wird aufgerundet (Belohnung für die höhere Komplexität)
\item Der Zaubernde hat jedesmal die Wahl wie er die Punkte auf die betroffenen Attribute verteilt (z.B. alle Punkte in ein Attribut)
\item Es werden die Attributspunkte aufgeteilt, dabei wird abgerundet. Aber Verbesserung durch Wege werden erst im Nachhinein zu jedem betroffen Attribut addiert.
\end{itemize}

Einige nützliche Regelerweiterung könnten sein:

\begin{itemize}
\item Illusionen können auch mit KONFAR an Ort und Stelle gebunden und wirken dann eine gewisse Zeit nach. Zauber so kurzfristig zu verankern sollte nur für Illusionszauber möglich sein. (Beispiel INGVA SYMA FLAM KONFAR: Verankert Illusionsflammen an einem brennenden Gegenstand. Die Flammen breiten sich aus).
\item VAS KONFAR, INGVA KONFAR und CENSA KONFAR Effekt können sich entweder auf die Dauer beziehen oder aber andere Effekte wie elementare Immunität und elementare Anfälligkeit geben. Ebenso könnten Fähigkeiten passend zu den Elementen oder beschriebenen Kreaturen gegeben werden. Beispiele wären Nachtsicht, tierischer Geruchssinn, Wasseratmung, etc.
\end{itemize}

\end{mdframed}
\subsubsection{LEKO (Freisetzung)}

Formrune, Zauberendwort, Komplexität: 2

Die Rune bringt das hervorgerufene Element zur Freisetzung (Explosion). Um die Explosion ins ein entferntes Ziel zu
bringen ist die Kombination mit einem Trägerzauber nötig (z.B. TEL). Die Rune LEKO wird dann als Zauberendwort
eingesetzt. Der klassische Feuerball ist ORT TEL FLAM LEKO. Für die Stärke der Explosion und den Schaden wird wie
folgt gerechnet:

\begin{itemize}
\item Das Feld auf dem der Zauber explodiert ist Radius 0 entfernt.
\item LEKO hat für die Berechnung von Schaden die volle Effektstärke in Radius 0 (Egal welche Verstärkungsstufe von LEKO)
\item Die sechs Felder darum sind Radius 1 entfernt. Die angrenzenden zwölf Felder sind Radius 2 entfernt, usw.
\item LEKO hat in Radius 1 volle Effektstärke. In Reichweite 2 reduziert diese sich um 1, in Reichweite 3 um einen weiteren Punkt, usw.
\item Vertärkung des LEKO Worts (durch VAS oder einen Weg) erhöhen die diese Reichweite jeweils um 1. So hat ein ORT TEL FLAM VAS LEKO volle Effektstärke im Radius 0 bis 2 im Radius 3 bis 4 ist die Effektstärke um 1 reduziert usw.
\item Verstärkungen addieren sich für die Reichweitenerhöhung. INGVA zählt als 2 Erhöhungen.
\end{itemize}

\subsubsection{LEP (Strahl)}

Formrune, Komplexität: 1

Formt den Zauber zu einen Strahl. In Ausnahmefällen könnte das Wort auch als Farbwort verwendet werden um einen
Strahl zu beschreiben. Als Beispiel könnte so ein Wasserstrahl beschrieben werden, allerdings wird dieser besser über
die WAKU Rune beschrieben.

Die Wirkung von Kampfzaubern wird im Kapitel Kampf behandelt. Eine Strahl hat eine deutlich kürzere Wirkung (6m) wird
dafür aber nicht beim Treffen auf ein Ziel beendet (und kann für kontinuierlichen Schaden verwendet werden).

Ein paar Beispiele:

\begin{itemize}
\item ORT LEP FLAM KONIU: Aufrechterhaltener Schadenszauber der jede Runde versuchen kann ein Ziel zu treffen.
\item ORT LEP CONDUC LEKO FLAM: Ein Flammenfächer mit einem Öffnungswinkel von 60 Grad (VAS LEKO hätte 120 Grad) und einer Reichweite von 6m oder 4 Felder, ein VAS LEP würde das verdoppeln.
\item ORT LEP CONDUC LEP (Farbwort): Könnte mit Zustimmung des Spielleiters eine Umschreibung der MANARE Rune sein.
\item ORT LEP CONDUC LEP POR: Könnte als alternative Variante ein Kettenblitz (zwei Ziele) beschreiben.
\end{itemize}

\subsubsection{LITAX (Luft)}

Farbrune, Basis, Kreis: 1: Komlpexität: 1

Beschreibt die Luft und Gase oder gasförmige Stoffe allgemein.

\begin{itemize}
\item Stärkungszauber: Erhöht die Wahrnehmung.
\item Schadenszauber: Verursacht körperlichen Betäubungsschaden.
\item Teilaspektzauber: Kann vor überlastung der Sinnesorgane durch Druck helfen.
\item Illusionszauber: Kann Geräusche beeinflussen und somit erheblich zum Erfolg beim Schleichen beitragen.
\item Kontrollezauber: Kann Luftströmungen erzeugen. In höheren Effektstärken ist auch das Fliegen einer Person denkbar.
\item Wahrnehmungszauber: Ermöglicht ''Luft sehen''. Giftige aber farblose Dämpfe können so z.B. gefunden werden. Vertärkt wäre auch eine Schallsicht denkbar.
\end{itemize}

\subsubsection{MAGI (Magie)}

Farbrune, Basis, Kreis: 1, Komplexität: 1

Beschreibt Magie und magische Strömungen. Die MAGI Rune hat im Elementarkreis keine Gegenrune.

\begin{itemize}
\item Stärkungszauber: Der Magie ist kein Attribut zugeordnet.
\item Schadenszauber: Verursacht körperlichen Betäubungsschaden.
\item Illusionszauber: Kann magische Effekte verschleiern oder vorgauckeln.
\item Kontrollezauber: Kontrolle von Magieströmungen, starke Kontrolle kann auch einfache magische Effekte manipulieren.
\item Wahrnehmungszauber: Wahrnehmung aller Magie und magischer Strömungen und Formen.
\end{itemize}

\begin{mdframed}[hidealllines=true, backgroundcolor=black!10]
\subparagraph{Spielleiterhinweis}

Zum Einen ist die MAGI Rune in einer magischen Welt extrem wichtig zum Anderen als beschreibende Rune greift sie
nicht viel. Als Regelvariante könnte eine Stärkung mit Magie das Glücksattribut erhöhen. Eine andere Idee ist es
Schadenszauber mit der MAGI Rune zu erlauben Schaden auf dem geistigen Monitor zu machen, das hat aber gravierende
Auswirkung auf Kämpfe. Kriegercharaktere erleiden durch ihre meist geringere Willenskraft mehr Schaden, während
Magier durch den zusätzlichen Schaden auf ihrem ''Magiemonitor'' ebenfalls gehindert sind.

\end{mdframed}
\subsubsection{MANARE (Leiten)}

Formrune, Komplexität: 1

Mit Hilfe dieser Rune ist es möglich magische Energien zwischen Zaubernden und Ziel hin und herzuleiten.

Diese Rune könnte auch durch AUM CONDUC LEP beschrieben werden. Als Schadenszauber ist diese Rune weniger potent,
so verursachen solche Zauber Minimalschaden und haben keinen Bonusschaden durch Willenskraft oder Erfolgsgrade. Je
nach Setting und in Absprache mit dem Spielleiter könnten MANARE Effekte mit MUTAR Effekten oder aber auch anderen
Zaubern kombiniert werden (CONDUC RUNE) um Zaubereffekte zu bezahlen oder andere Energien verfügbar zu machen.

Bekommt ein Ziel mit Hilfe eines solchen Zaubers Lebensenergie so heilt es, allerdings auch hier nur mit der
minimalen Menge ohne Willenskraftboni.

Diese Rune findet meist Anwendung bei verzauberten Gegenständen die über keine eigene innere Macht verfügen und auf
Energien der Umgebung angewiesen sind.

Beispiel:

\begin{itemize}
\item ORT MANARE MANI KONIU: Leben entziehen. Je nach Setting kann es sein das man Heilzauber nur mit magischer Energie des Lebens bezahlen kann, dieser Zauber wäre eine Möglichkeit an solche Energie zu kommen oder direkt zu heilen.
\end{itemize}

\begin{itemize}
\item ORT MANARE MAGI KONIU: Magie entziehen. Falls es das Setting erlaubt könnte so geistige Erschöpfung geheilt werden.
\end{itemize}

\begin{itemize}
\item ORT MANARE PRIX CONDUC SYMA FLAM KONFAR: Die anhaltene Feuerillusion könnte jetzt noch länger anhalten da sie mit Sonnenenergie gespeist wird.
\end{itemize}

\begin{itemize}
\item ORT MANARE FLAM CONDUC FLAM MUTAR MAGI KONIU: Verwandelt Feuerenergie in nutzbare ''graue'' Magie. Das wäre hilfreich um z.B. einen Gegenstand zu erschaffen dessen Verzauberung aufgeladen wird indem er ins Feuer geschmissen wird (und dabei das Feuer schwächt)
\end{itemize}

\subsubsection{MUTAR (Verwandeln)}

Formrune, Spezialrune Komplexität: 0

Diese Rune benötigt zwei Farben. Die Farbe die vor MUTAR beschreiben wird, wird in die Farbe nach MUTAR verwandelt.
Ein lebendes Ziel kann sich auch mit seiner Konstitution gegen eine unfreiwillige Verwandlung wehren.

Ziele verwandeln sich nach Ablauf des Zaubers zurück. Solche Rückverwandlung schützt in gewissen Rahmen das Ziel. So
würde ein Mensch der als Tier verwandelt in einen Drahtkäfig gesteckt wurde, diesen bei der Rückverwandlung
zerreissen. Ein Stahlkäfig könnte dem aber widerstehen. Verwandlungszauber werden pro Szene und nicht pro Kampfrunde
aufrecht erhalten. Eine Alternative ist diese Zauber auch mit KONFAR zu binden (Entscheidung des Spielleiters).

Beispiel:

\begin{itemize}
\item HUMI MUTAR FERA KONIU: Mensch in ein Tier verwandeln. Für spezielle Tiere müsste dieses weiter beschreiben werden.
\item FERRUM NOX MUTAR FERRUM PRIX KONFAR: Silber in Gold verwandeln (zumindest für einen Moment).
\item WAKU MUTAR FIR KONIU: Wasser zu Eis erstarren lassen.
\item SICR AN MUTAR SICR: Gegenstand zurück verwandeln (reparieren).
\item EBOR MORT MUTAR EBOR MANI CENSA KONFAR: Durch das CENSA KONFAR sollte hier die tote Erde in fruchtbare verwandelt werden. Die Wirkung sollte ausreichend lange halten um einen natürlichen Kreislauf zu starten. (Aber wie immer hat der Spielleiter das letzte Wort ob das zu seinem Setting passt oder nicht.)
\end{itemize}

\subsubsection{MANI (Leben)}

Farbrune, Metaphyisch, Kreis: 3, Komplexität: 2

Beschreibt rudimentäre Lebensenergie. Kann zur Heilung von körperlichen Schaden verwendet werden. Teilaspekt ist
Wohlbefinden. Mit Essenzmagie ist die Wiederherstellung von körperlichen Beeinträchtigung möglich.

\begin{itemize}
\item Stärkungszauber: Erhöht Widerstandfähigkeit gegen körperliche Beeinträchtigung.
\item Schadenszauber: Kann gegen Untote oder ähnliche verwendet werden (evtl. Anfälligkeit ausnutzen). Gegen lebende Ziele wirkt ein Angriffszauber mit der Farbe MANI nicht.
\item Teilaspekt: Beschreibt körperliches Wohlbefinden.
\item Illusionszauber: Kann sehr rudimentäres Wohlbefinden (wie Drogenrausch) vorspielen.
\item Kontrollzauber: Kann Lebensenergie umlenken. Für aktive Schädigung ist aber ein MANARE Zauber nötig.
\item Wahrnehmungszauber: Zeigt Lebewesen.
\end{itemize}

\subsubsection{MORT (Tod)}

Farbrune, Metaphyisch, Kreis: 3, Komplexität: 2

Beschreibt rudimentär das Beenden des Lebens. Teilaspekte sind Symptome von Gift und Krankheit. Mit Essenzmagie
können Verkrüpplung und Ähnliches verursacht werden.

\begin{itemize}
\item Stärkungszauber: Verursacht Vergiftungen (mit CENSA verstärkt auch ernste Krankheiten).
\item Schadenszauber: Verursacht schwerer heilbaren Schaden gegen Lebewesen und ist ansonsten wirkungslos.
\item Teilaspekt: Beschreibt Symptome von Vergiftungen und Krankheit.
\item Illusionszauber: Kann Unbehagen vorspielen.
\item Kontrollzauber: Kann verwendet werden um Einflüsse von Todesmagie zu manipulieren. So kann beispielsweise die  Ausbreitung einer Krankheit oder eines Gifts behindert werden.
\item Wahrnehmungszauber: Kann Todesmagie erkennen (z.B. Untote, magische Gifte und Krankheiten).
\end{itemize}

\subsubsection{NOX (Dunkelheit)}

Farbrune, Basis, Kreis: 1, Komplexität: 1

Beschreibt die Abwesenheit und Auslöschung von Licht.

\begin{itemize}
\item Stärkungszauber: Erhöht die Willenskraft.
\item Schadenszauber: Verursacht körperlichen Betäubungsschaden.
\item Teilaspektzauber: Beschreibt Halblicht und Schattenränder.
\item Illusionszauber: Schatten und Unschärfe. Ermöglicht sich ohne weiter Sichtdeckung im Schatten zu verstecken.
\item Kontrollezauber: Ermöglich Schatten und Dunkelheit zu kontrollieren und Licht zu verdrängen.
\item Wahrnehmung: Ermöglicht das Finden von Dunkelheit, Effekten magischer Dunkelheit. Mit Zustimmung der Spielleitung kann ein Zurechtfinden ähnlich einer Sonarsicht in absoluter Dunkelheit möglich sein.
\end{itemize}

\subsubsection{OBSEK (Beschwörung/Ruf)}

Formrune, Komplexität: 1

Ruft die beschriebene Form herbei. Ob Beschwöreungsmagie erlaubt ist und wie sie funktioniert regelt das Setting.
Der Spielleiter muss für seine Welt mehrere Entscheidungen treffen wie Beschwörung funktioniert.

\begin{itemize}
\item Dürfen Spieler und einfache Charaktere herbeirufen?
\item Erscheint ein beschworener Gegenstand aus dem Nichts. Wie real ist er wie lange existiert er?
\item Können Menschen mit OBSEK HUMI beschworen werden? Wenn ja sind sie echt? Bleiben sie? Tauchen sie aus dem Nichts auf oder kommen sie angelaufen? Woher stammen sie?
\item Können Tiere beschworen werden. Hier stellen sich ähnlich Fragen wie bei den Menschen.
\item Können Geister (ANIMA) beschworen werden. Woher kommen Sie? Wie feindlich sind sie? Haben Sie eine Persönlichkeit? Ist diese Persönlichkeit persistent?
\end{itemize}

Wenn Wesenheiten beschworen werden könnte das z.B. nur in einem Ritual erlaubt sein. Beschworene Wesenheiten könnten
Päkte anbieten und wären dann an ihren Wortlaut gebunden. Beschworene Wesenheiten könnten eine Agenda haben und den
Charakter mit magischer Macht oder Beistand belohnen wenn er ihre Agenda vorwärts bringt. Wenn Beschwörungsmagie eine
Rolle spielen soll so muss der Spielleiter das bei der Ausarbeitung des Settings berücksichtigen. Generell wird
empfohlen die Macht die eine Wesenheit für Spielercharaktere bereithält mit entsprechend Hürden zu versehen. Ein
paar Beispiele:

\begin{itemize}
\item Beschwören des Geistes des Ortes. Dieser hat keinen großen Einfluss auf die reale Welt kann aber den Charakteren nützliche Tips geben (Wo kann ich mich verstecken, Wo finde ich was zu essen, Wer ist hier vor mir gewesen, etc.). Solche Wesenheiten können vom Spielleiter wie normale NSCs gehandhabt werden. Sind die Charaktere sympathisch so helfen ihnen die Geister. Hier könnte der Spielleiter Geisterfraktionen haben und den Charakteren haften der Geruch von z.B. Stadtgeister oder Ahnengeistern an. Darauf könnten Naturgeister unwirsch reagieren. (OBSEK ANIMA)
\item Beschwören von Elementargeistern (OBSEK ANIMA FLAM) könnten einfache Kampfgefährten für den Kampf herbeirufen. Einziges Manko ist sie müssen kontrolliert werden (OBSEK ANIMA FLAM AUM erlaubt eine vergleichende Willenskraftprobe, gewinnt das Elementar so greift es alles und jeden an). Natürlich könnte man auch unkontrollierte ELementare beschwören (z.B. als Ablenkung im feindlichen Lager).
\item Beschwören von Tier oder Menschengeistern (OBSEK ANIMA FERA, OBSEK ANIMA HUMI). Sollte die normale Beschwörung von Tieren (OBSEK FERA) erlaubt sein so sollten sich die Geistwesen in ihren Fähigkeiten und ihrer Macht deutlich von den deutlich einfacher herbeirufbaren unterscheiden. Diese Beschwörung könnten auch um weitere Farben erweitert sein und somit halbgöttliche Wesen beschreiben (OBSEK ANIMA HUMI SOL könnte eine Engelsbeschwörung sein). Diese könnten den Charakter ohne weiteres unterstützen. Diese Wesenheiten werden aber Buch darüber führen wie der Charakter zu ihrer Agenda passt und ihm evtl auch den Dienst versagen.
\item Beschwören von Tieren (OBSEK FERA) könnte nur erlaubt sein wenn es in einem Ritual gesprochen wird. Das Tier wird irgendwann dem Ruf folgen und am Ritualplatz sein. Im Kampf ist das nicht hilfreich, aber evtl können so Tiergefährten herbeigerufen werden. Oder es kommt ein mystisch aufgeladenes Wesen (sprechende Tiere, oder ähnliches).
\end{itemize}

Weitere unkomplizierte Beispiele die relativ einfach zum restlichen Magiesystem passen:

\begin{itemize}
\item OBSEK GLADI FLAM: Waffenbeschwörung (Kurzschwert oder kleiner) die statt physischen Feuerschaden macht. Effektstärke erhöht die Qualität der Waffe.
\item OBSEK UR FERRUM: Beschwört Eine Rüstung oder Schild. Effektstärke bestimmt den Rüstungsschutz.
\item OBSEK WAKU: Wasserbeschwörung für den durstigen Reisenden.
\item OBSEK BET SICR: Mit hohere Effektsärke falls man mal wieder den Schlüssel vergessen hat.
\item OBSEK UR EBOR: Drei Kampffelder mit einer Mauer versehen.
\item OBSEK GLADI CONDUC TEL: Bogen beschwören.
\end{itemize}

\subsubsection{PRIX (Licht)}

Farbrune, Basis, Kreis: 1, Komplexität: 1

Beschreibt Licht und Helligkeit. In wissenschaftlichen Sinne elektromagnischte Strahlung oder Photonen.

\begin{itemize}
\item Stärkungszauber: Steigert Intelligenz oder Persönlichkeit nach Wahl des Zaubernden.
\item Schadenszauber: Verursacht körperlichen Betäubungsschaden.
\item Teilaspektzauber: Kann vor Lichteffekten schützen.
\item Illusionszauber: Erzeugt Helligkeit und Licht (auch farbiges). Ein Ziel kann ''durchsichtiger'' werden.
\item Kontrollezauber: Licht von Lichtquellen manipulieren, starke Kontrolle könnte z.B. auch verwendet werden um einen ''Spiegel'' zu erschaffen.
\item Wahrnehmungszauber: Lichteffekte (insbesondere Illusionen) erkennen. Kann je nach Kampagne auch verwendet werden um Restlichtsicht zu ermöglichen.
\end{itemize}

\subsubsection{SCIEN (Wahrnehmung)}

Formrune, Komplexität: 0

Ermöglicht es dem Magier seine Sicht auf die verschiedenen Energieströme bzw. Farben einzustellen. Mittels ORT
Zauber kann die Sicht auch jemand anderem gegeben werden. Verstärken der SCIEN Rune ermöglicht tieferen Detailgrad
oder den Schatten/Nachhall von Energieströmen zu untersuchen. Es empfiehlt sich solche Zauber als KONIU Zauber zu
sprechen.

Solche Wahrnehmungszauber können teilweise durch Materie hindurchschauen. Die Reichweite ist aber begrenzt. Eine
dicke Burgmauer sollte einfache Wahrnehmungszauber blocken.

Mittels KONFAR Magie wird die Wahrnehmung weg von der Sicht und mehr zu einem Gefühl. Mittels KONFAR ist es dann
auch möglich andere Wahrnehmung z.B. aus dem Tierreich zu bekommen. Hier muss aber die Farbe entsprechend beschrieben
sein.

Ein paar Beispiele

\begin{itemize}
\item SCIEN MAGI KONIU: Klassisches Magie entdecken.
\item VAS SCIEN WAKU KONIU: Wasser finden. Erlaubt auch das sehen von unterirdischen Wasseradern.
\item SCIEN UR KONIU: Wände aufspüren. Der Zauber erlaubt grob die Form der nächsten Räume zu erahnen. (Spielleiterentscheid!)
\item SCIEN MAGI KONFAR: Magiegespür. Man spürt die Magie ohne das sie die Sicht stört.
\item SCIEN FERA NOX KONFAR: Das Ziel bekommt den Sinn eines Raubtiers.
\end{itemize}

\subsubsection{SICR (Gegenstand)}

Farbrune, Gegenstand, Kreis: 4, Komplexität: 3

Beschreibt von vernunftbegabten hergestellte Gegenstände. Um größere Gegenstände zu beschreiben sind
Verstärkungsworte nötig. So könnten Möbel mit VAS SICR, Kutschen und ähnliches mit INGVA SICR beschrieben werden.
Letztes Wort hat wie immer der Spielleiter. Auch die Effektstärke kann herangezogen um die Beschreibung zu
unterstützen.

Stärkungszauber: Zusammen mit AUM kann SICR verwendet werden um den Gegenstand besser nutzen zu können.
Schadenszauber: Kann physischen Schaden beschreiben. In Kombination mit AN könnte ein Gegenstand auflösen Zauber
beschrieben werden.
Teilaspektzauber: Beschreibt kleine Gegenstände wie Münzen oder Schlüssel.
Illusionszauber: Erstellt Täuschung von Gegenständen
Kontrollzauber: Telekinesezauber die die Gegenstände bewegen. Effektstärken und Verstärkungsworte legen die
physikalische Kraft fest. Könnte auch als Schadenszauber gewirkt werden (und bewirkt dann physischen Schaden in Höhe
der Effektstärke).
Wahrnehmungszauber: Kann Gegenstände finden (verstärkte Wahrnehmungszauber könnten spezifische Gegenstände finden).

\subsubsection{SYMA (Illusion)}

Formrune, Komplexität: 0

Ermöglicht es die die durch die Farbe dargestellte Sache als Illusion zu erschaffen. Der Magier hat hier ein bischen
mehr Einfluss auf die Beschreibung des Effekts, insbesondere wenn er Verstärkungsrunen für das SYMA verwendet.
Illusionen sind nicht wirklich real und können so keine echt physische oder andere Interaktion durchführen. Starke
Illusionszauber können aber durchaus mehrere Sinne täuschen (bis hin zum Tastsinn).

Illusion wirken echt bis mit ihnen ernsthaft interagiert wird, dann können getäuschte Charaktere versuchen der Magie
zu widerstehen (Willenskraftprobe) oder sie zu durchschauen (Wahrnehmungsprobe). Sollte die Illusion unlogisch sein
(weil sie z.B. nicht interagieren kann) so wird sie automatisch durchschaut.

Damit die Illusion sich von der Stelle bewegen kann ist die Verstärkung mittels VAS nötig. Damit sie realistisch auf
ihre Umgebung durch Steuerung des Zaubernden reagieren kann ist das AUM Wort am Zauberende vonnöten. Ein mit INGVA
verstärkter Zauber agiert halbwegs selbstständig auf seine Umgebung (ein Hund knurrt, Feuer breitet sich aus)

Beispiele:

\begin{itemize}
\item SYMA FLAM KONIU: Der Zaubernde erschafft Flammen die zwar keinen echten Schaden anrichten aber Feinde oder wilde Tiere in die Flucht schlagen können.
\item SYMA EBOR oder SYMA UR EBOR: Erschafft eine Erdwand oder eine Steinmauer als Illusion.
\item VAS SYMA BET FERA LITAX AUM KONIU: Erschafft einen Illusionsvogel. Der könnte aber z.B. eine Botschaft übermitteln.
\item SYMA PRIX oder SYMA NOX: Mittels SYMA PRIX wird der Charakter durchsichtiger (abhängig von der Effektstärke) mit SYMA NOX ist er in Schatten gehüllt. Je nachdem kann das beim Schleichen und Verstecken hilfreich sein.
\end{itemize}

\subsubsection{TEL (Geschoss)}

Formrune, Farbrune (Kreis: 4), Komplexität: 1

Formt einen Zauber zu einem Geschoss. Kann auch als Farbwort für Geschoss verwendet werden. Die Wirkung von
Geschossen wird im Kapitel Kampf beschrieben.

Geschosszauber enden mit Auftreffen. Sie setzen dann ihre schädliche Energie frei. Geschosse fliegen in einer
Runde 300m.

Ein paar Beispiele:

\begin{itemize}
\item ORT TEL FLAM: Verursacht einen Geschossangriff der gegen ein Ziel gerichtet ist (siehe Kampf).
\item ORT TEL FLAM LEKO: Der klassische Feuerball, kann ein Ziel direkt treffen und verursacht Flächenschaden.
\item ORT VAS AUM TEL KONIU: Versucht die Kontrolle über ein Geschoss zu bekommen. Ein normales AUM würde reichen um das Geschoss abzulenken, mit einem VAS AUM ist auch ein zurücklenken möglich. Die Effektstärke wird direkt mit der Stärke des Geschosses verglichen und sollte ausreichend sein damit der Zauber wirkt. Einen geschleuderten Speer sollte aber besser mit einem VAS TEL beschrieben sein.
\item ORT TEL FLAM AUM: Geschosszauber der die Verteidigung des Ziels reduziert indem er das Geschoss nachsteuerbar macht. Ein VAS AUM würde die Verteidigung stärker reduzieren.
\item ORT TEL FLAM AUM KONIU. Ein Geschoss das zwar nur einmal trifft aber mehr als 300m Reichweite hat (und dabei pro Runde 300m zurücklegt). Das AUM erlaubt es die Geschossrichtung anzupassen.
\end{itemize}

\subsubsection{TORA (Zeit)}

Farbrune, Universum, Kreis: 7, Komplexität: 6

Beschreibt die Zeit und das Vergehen von Zeit.

\begin{itemize}
\item Stärkungszauber: Kann die Handlungsgeschwindigkeit manipulieren.
\item Schadenszauber: Verursacht katastrophalen Schaden bei Lebewesen.
\item Teilaspekt: Kann zum Messen von Zeit verwendet werden (z.B SCIEN BET TORA).
\item Illusionszauber: Beschreibt die Zeit und zeitliche Effekte. Ist für Illusionen relativ ungeeignet.
\item Kontrollzauber: Kann die Bewegungsgeschwindigkeit verändern (um einen Multiplikator).
\item Wahrnehmungszauber: Kann zeitliche Abläuf wahnehmen, je nach Effektstärke und zusätzlicher Runen kann auch ein Blick in die Vergangenheit möglich sein.
\end{itemize}

\subsubsection{TRIA, PENTA, HEPTA (Dreifach, Fünfach, Siebenfach)}

Zauberbeginn, Komplexität: 4,6,8

Die jeweilige Rune wird zu Beginn des Zaubers gesetzt und lässt den Zaubernden den Zauber auf drei, fünf oder sieben
Ziele gleichzeitig wirken. Dabei ist aber zu beachten das ein Ziel von einem magischen Effekt immer nur einmal
betroffen ist auch wenn sich z.B. LEKO Effekte überschneiden.

Beispiele:

\begin{itemize}
\item HEPTA ORT MANI INGVA LEKO: Massenhafte Flächenheilung mit 7 Zentren der Freisetzung.
\item TRIA WAKU KONFAR: Gibt drei Zielen in Berührungreichweite eine verbesserte Geschicklichkeit.
\item TRIA ORT VAS LEP FLAM KONIU: Drei aufrechterhaltende Feuerstrahlen die jede Runde drei unterschiedliche Ziele bekämpfen können.
\end{itemize}

\subsubsection{UR (Schild, Wand)}

Formrune, Farbrune (Kreis: 4), Komplexität: 1

Erschafft Schilde und Wände. Kann als Farbwort für die Beschreibung von Wänden, Schilden und Rüstungen verwendet
werden. Regeln für die Nutzung von Schildzaubern befinden sich im Kapitel Kampf. Schilde und Wände können durch das
Erleiden von Schaden zerstört werden, ein zerstörter Zauber endet automatisch (wieder siehe Kapitel Kampf).
Hervorgerufene oder Beschworene Wände decken drei Kampffelder (nach Wahl des zaubernden) ab. Ein VAS UR erhöht diese
auf 6 und INGVA UR auf 9 Felder.

Ein paar Beispiele:

\begin{itemize}
\item UR MAGI KONIU: Schützt ein Ziel mit einem magischen Schild.
\item UR FLAM KONFAR: Verpasst dem Ziel eine Rüstung gegen Feuerschaden. Erde oder Metall als Farbwort würden Schutz gegen physische Angriffe geben.
\item ORT UR FLAM KONIU: Verschafft entweder einem einzelnem Ziel ein magisches Schild oder erzeugt eine magische Flamenwand (die hier Schaden in Höhe der Effektstärke verursacht).
\item (ORT) UR MAGI KONFAR: Kann auch auf ein magisches Schild gesprochen werden um ihm einen Rüstungsschutz zu geben.
\item 
\end{itemize}

\subsubsection{VAS (Vergrößern, Verstärken)}

Spezialrune, Komplexität: 2

Verstärkt das nachfolgende Wort. Bei Farbworten kann damit das Vergrössern gemeint sein. So kann GLADI bis zum
Kurzschwert als Farbwort gemeint sein, VAS GLADI kann dann jede von Menschen führbare Nahkampfwaffe meinen. Ein
VAS FERA ist deutlich größer als ein FERA, etc. Der Faktor hier ist etwa 2. Noch größere Verstärkung sind mit dem
INGVA Wort möglich.

Wird die Rune auf eine Formrune angewandt so verbessert sie deutlich die Wirkung. Z.B. lässt ein ORT AUM SICR einen
Gegenständen schweben und macht ihn beweglich. Ein ORT VAS AUM SICR lässt aber auch Manipulation an dem Gegenstand zu
(Tasten drücken, ein Handtuch auswringen, einen Deckel hochklpappen etc., für Manipulation die geschickte Finger
benötigen muss aber INGVA verwendet werden). Boni die durch Farbrunen erzeugt werden, werden verdoppelt.

Ein paar Beispiele für verstärkte Zauber:

\begin{itemize}
\item VAS SCIEN MAGI oder SCIEN VAS MAGI: Der Verzauberte kann auch nicht mehr präsente Magie spüren. Als so eine Art astrale Spuren. Weitere Verstärkungen bringen ihm Bonuswürfel oder erlauben auch sehr alte oder feine Spuren zu sehen.
\item ORT VAS TEL VAS LITAX: Sowohl Geschoss als auch Luft ist verstärkt. Mit Zustimmung des Spielleiters kann hier vom Getrofffenen eine Niederschlagsprobe gegen die Willenskraftsprobe oder Erfolgsprobe des Zaubernden zu leisten sein.
\item VAS SYMA FERA: Die Illusion wird realer, das Tier kann Bewegung ausführen.
\item VAS SYMA FLAM: Die Illusion wird realer, Hitze wird von dem Feuer ausgestrahlt. (Auch eine normale Feuerillusion täuscht alle Sinne. Die Hitze hier verursacht aber auch Schmerzen, Gerüche etc.)
\item VAS AUM: Verbessert die Kontrolle oder die Tiefe der Kontrolle.
\item VAS SCIEN: Verbessert die Wahrnehmungstiefe und Schärfe.
\item VAS UR: Vergrössert die Hervorgerufenen Wände. Kann auch verwendet werden um grossen Kreaturen ein magisches Schild zu geben.
\item VAS LEKO: Verdoppelt die Reichweite des Freisetzungseffekt.
\item VAS LEP, VAS TEL: Verdoppelt die Reichweite.
\end{itemize}

\subsubsection{WAKU (Wasser)}

Farbrune, Basis, Kreis: 1, Komplexität: 1

Beschreibt das Wasser und Flüssigkeiten. In wissenschaftlichem Sinne is damit der flüssige Aggregatzustand gemeint.

\begin{itemize}
\item Stärkungszauber: Erhöht das Geschicksattribut.
\item Schadenszauber: Verursacht körperlichen Betäubungsschaden.
\item Teilaspektzauber: Beschreibt Feuchtigkeit, kann vor demm Austrocknen schützen.
\item Illusionszauber: Wasser und wellenartige Bewegung. Kann dem Tastsinn und Geschmack oberflächlich täuschen.
\item Kontrollezauber: Ermöglicht Wasser und wässrige Lösungen zu bewegen.
\item Wahrnehmung: Ermöglicht das Finden von Wasser (auch magisch) und Feuchtigkeit oder die bessere Wahrnehmung unter Wasser.
\end{itemize}

\begin{center}
\subsection{Grimoirs}
\end{center}

Hier sind Vorschäge zu Zaubersprüchen thematisch zusammen gefasst und in Zauberbüchern dargestellt. Die Idee ist das
diese als Handouts verwendet werden können. Zusätzlich bieten sie weitere Anhaltspunkte für die Interpretation der
Zauber. Alternativ kann das freie Magiesystem auch mittels dieser Grimoirs in ein nicht freies verändert werden.
Dann gibt es nur diese Zauber und die Charaktere müssen Zugang zu einem Grimoir haben um daraus die Zaubersprüche zu
lernen. Hier müssten ggfs weitere Grimoirs vom Spielleiter erstellt werden (Z.B. Kampfmagie mit anderen Elementen).

Für eine einfache Handhabung ist die Effektstärke (ES), der Mindestwurf (MW) und die Komplexität (K) angegeben. Die
Komplexität (K) lässt sich auch als Kosten interpretieren. Diese Werte können durch Wege modifiziert werden.

\subsubsection{Adeptgrimoir der Heilung}

Die Heilzauber hier können je nach Setting zu stark sein. Um diese abzuschwwächen kann die Verfügbarkeit eingeschränkt
sein oder die Energiekosten müssen anders/aufwendiger bezahlt werden.

\begin{itemize}
\item Wundheilung (MANI, ES: 2, MW: 15, K: 5). Heilt mit einer Effektstärke von 2 (genaueres im Kapitel Kampf).
\item Giftheilung (AN MORT KONFAR, ES: 1, MW: 15, K: 5). Anstatt Wunden wird in gleicher Stärke Gift geheilt.
\item Leben spüren (SCIEN MANI KONIU, ES: 2, MW: 15, K: 5). Wahrnehmungszauber für Leben. Ermöglicht es sehr begrenzt durch schwache Sichthindernisse wie Türen zu schauen und dort Lebewesen (und nichts anderes) zu sehen.
\end{itemize}

\subsubsection{Adeptgrimoir der Illusion}

Die drei Heimlichkeitszauber sind ein Beispiel wie unterschiedliche Illusion situationsbedingt Boni geben. Es spricht
auch nichts dagegen die Bonuswürfel immer für Heimlichkeitsproben zu geben egal welcher Zauber wirkt.

\begin{itemize}
\item Schattenumhang (SYMA NOX KONIU, ES: 2, MW: 14, K: 4). Verbessert Schleichenproben um 2W. Erfordert aber Schatten in die man sich hüllen kann.
\item Tarnung (SYMA PRIX KONIU, ES: 2, MW: 14, K: 4). Verbessert Versteckenproben (wenig Bewegung des Charakters in direkter Sicht) um 2W.
\item Stille (SYMA LITAX KONIU, ES: 2, MW: 14, K: 4). Dämpft die Geräusche des Charakters und gibt ihm für Schleichenproben 2W.
\item Illusionswand (SYMA UR EBOR KONIU, ES: 2, MW: 15, K: 5). Erzeugt eine Erd oder Steinwand die ca drei. Felder (1,5m) belegt und menschengroß ist. Ihre Form kann auch eine einfache geometrische Formen von ca 8 Quadratmetern gebrach werden.
\item Lichtblitz (ORT SYMA PRIX, ES: 1, MW: 13, K: 3). Ein kurzer heller weithin sichtbarer Lichtblitz dessen Farbe von Zaubernden bestimmt wird und gut für die Übermittlung einfacher Signale geeignet ist.
\item Geräusch (ORT SYMA LITAX, ES: 1, MW: 13, K: 3). Erzeugt ein einfaches Geräusch um z.B. Gegner abzulenken.
\end{itemize}

\paragraph{Adeptgrimoir des Kampfes}

Die Zaubersprüche in diesem Buch können mit beliebigen primären Elementarfarben beschrieben werden. So liesse sich
das Feuergeschoss auch als Erdgeschoss (ORT TEL EBOR) wirken. Die Wirkung von Kampfzaubern wird im Kapitel Kampf
beschrieben.

\begin{itemize}
\item Feuergeschoss (ORT TEL FLAM, ES: 2, MW: 16, K: 6). Erzeugt ein Feuergeschoss das auf einen Gegner geschleudert werden kann. Um den Gegner zu treffen muss Die Erfolgsprobe beim Zaubern auch die Verteidigung erreichen.
\end{itemize}

\begin{itemize}
\item Erdschild (UR EBOR KONIU, ES: 2, MW: 15, K: 5). Erzeugt ein magisches Schild das vor Schaden schützt. Erdmagie richtet nur halben, Luftmagie richtet doppelten Schaden gegen das Schild an. Zauber muss auf Berührung gewirkt werden und jede Runde aufrechterhalten werden. Das Aufrechterhalten muss aber nicht mehr auf Berührung stattfinden.
\end{itemize}

\begin{itemize}
\item Feuerpfeile (TEL FLAM KONFAR), ES: 1, MW: 14, K: 4). Verzaubert 10 Pfeile so das sie beim Treffen zusätzlichen Feuerschaden machen (Willenskraft wird hier aber nicht als Bonus verwendet). Der Zauber wirkt für diese Szene.
\end{itemize}

\begin{itemize}
\item Waffenverzauberung (AUM VAS GLADI KONFAR, ES: 1, MW: 16, K: 6). Verzaubert eine von Menschen geführte Nahkampfwaffe so das sie um 1 besser trifft. Kleinere Waffen bis hin zu Kurzschwertern könnten ohne das VAS Wort verzaubert werden. (AUM GLADI KONFAR, ES: 1, MW: 14, K: 4)
\end{itemize}

\begin{itemize}
\item Flamenschwert (GLADI FLAM KONFAR, ES: 2, MW: 17, K: 7). Verzaubert eine Waffe so das sie zusätzlichen Feuerschaden (auch hier ohne Willenskraftbonus!) veruracht.
\end{itemize}

\subsubsection{Adeptgrimoir der Stärkung}

\begin{itemize}
\item Stärke steigern (FLAM KONFAR, ES: 2, MW: 15, K: 5). Steigert das Attribut um 2.
\item Geschick steigern (WAKU KONFAR, ES: 2, MW: 15, K: 5). Steigert das Attribut um 2.
\item Konstitution steigern (EBOR KONFAR, ES: 2, MW: 15, K: 5). Steigert das Attribut um 2.
\item Intelligenz/Persönlichkeut steigern (PRIX KONFAR, ES: 2, MW: 15, K: 5). Steigert entweder Intelligenz oder Persönlichkeit um 2.
\item Willenskraft steigern (NOX KONFAR, ES: 2, MW: 15, K: 5). Steigert das Attribut um 2.
\item Wahrnehmung steigern (LITAX KONFAR, ES: 2, MW: 15, K: 5). Steigert das Attribut um 2.
\item Mauer verstärken (UR EBOR KONFAR, ES: 2, MW 16, K: 6). Erhöht drastisch die Stärke einer Mauer. Die Fläche sollte auf 8-10 Quadratmeter begrenzt sein.
\end{itemize}

\subsubsection{Magusgrimoir der Heilung}

\begin{itemize}
\item Regeneration (ORT MANI KONIU, ES: 2, MW: 16, K: 6). Heilt das Ziel jede Runde, allerdings gibt es nur die erste Runde Bonusheilung durch Willenskraft.
\item Stärkung (MANI KONFAR, ES: 3, MW: 19, K: 9) Stärkt den Charakter und reduziert die Schwere von Krankheiten und Gifte um 3.
\item Flächenheilung (ORT MANI LEKO, ES: 2, MW: 18, K: 8). Heilt eine Fläche wie ein entsprechender Schadenszauber. Hier gibt es keinen Bonus durch Willenskraft.
\end{itemize}

\subsubsection{Magusgrimoir der Illusion}

Der ''Magie verbergen'' Zauber ist ein Sonderfall bei dem der Spielleiter sich genaue Regeln zurecht legen sollte.
Vorgeschlagen ist das ein einfaches SYMA ein einfaches SCIEN bei gleichen Effektstärken täuschen kann. Ein VAS SYMA
kann dann ein VAS SCIEN täuschen und so weiter. Ist der Charakter mit dem Wahrnehmungszauber im Vorteil so sollten
ihm Wahrnehmungsproben gestattet werden. Je nach Situation kann hier durch mehr Vorteil der Malus bei der Probe
reduziert oder aufgehoben werden. Magie verbergende Zauber sollten von klugen Charakteren mittels mundaner
Heimlichkeit verbessert werden und eher dazu dienen zufälligen Entdeckungen zu entgehen.

\begin{itemize}
\item Magie verbergen (ORT SYMA AN MAGIE KONIU, ES: 2, MW: 16, K: 6). Erschafft ein Feld von illusionärer Antimagie. Das kann einfache Wahrnehmungszauber täuschen sofern die Effektstärken der Magie verbergen und aktiver Zauber gleich sind. Ist ein Ungleichgewicht so gibt es den zu täuschenden ein Bonus.
\end{itemize}

\begin{itemize}
\item Raubtierillusion (ORT SYMA FERA NOX KONIU, ES: 2, MW: 19, K: 9). Erschafft ein bedrohliches lebensechtes Raubtier. Das Raubtier ist in seiner Bewegung und Handlung eingeschränkt. Für den ersten Schreck reicht es allemal. Knurren und drohen kann es auch.
\end{itemize}

\begin{itemize}
\item Dickicht (ORT SYMA VAS FLORA KONIU, ES: 2, MW: 20, K: 10). Erschafft eine Illusion von undurchdringlichen Gestrüpp (oder beliebig anderen großen Pflanzen). Die Illusion kann 6 Felder abdecken und ist dabei ca 5m groß.
\end{itemize}

\begin{itemize}
\item Melodie (VAS SYMA VAS LITAX KONIU, ES: 2, MW: 18, K: 8). Erschafft eine weit in die Landschaft tragende Melodie. Töne können dabei jedem Instrument nachgeahmt werden. Sprache ist nicht möglich. Kommunikation wäre aber z.B. über Melodiefolgen möglich.
\end{itemize}

\begin{itemize}
\item Unsichtbarkeit (SYMA PRIX KONIU, ES: 3, MW: 17, K: 7). Macht den Charakter quasi unsichtbar (aber nicht unhörbar) solange er sich nicht zu schnell bewegt. Ein Charakter der sich so im Kampf befindet und von seinen Gegnern gesehen wurde (z.B. weil er angegriffen hat) bleibt entdeckt solange er in Sicht bleibt.
\end{itemize}

\subsubsection{Magusgrimoir des Kampfes}

Auch hier kann man die elementaren Farbworte austauschen um die Zauber zu modifizieren.

\begin{itemize}
\item Feuerball (ORT TEL FLAM LEKO, ES:2, MW: 18, K: 8). Feuerball der Flächenschaden verursacht. Sollte das primär Ziel von dem Geschoss getroffen werden so richtet der Zauber dort Bonusschaden durch Willenskraft und Erfolgsgrade an.
\end{itemize}

\begin{itemize}
\item Flammenstrahl (ORT LEP FLAM KONIU, ES: 3, MW 19, K: 9). Flammenstrahl mit Effektstärke 3 der bei einem einzelnen Ziel Feuerschaden verursacht. Der Strahl kann auffrecht erhalten werden er muss aber jede Runde auf ein Ziel gerichtet werden (erneute Probe zum Treffen). Der Zauber verursacht nur in der ersten Runde Bonusschadden durch Willenskraft. Sollte noch eine andere Handlung vom Zaubernden gemacht werden während er mit dem Flammenstrahl Feinde bekämpft, so macht er mehrere Handlungen gleichzeitig und bekommt entsprechend Würfelabzüge (Z.B. Angriff mit Schwert und Flammenstrahl reduziert den Würfelpool für beide Proben um 2W).
\end{itemize}

\begin{itemize}
\item Flammenkegel (ORT LEP CONDUC LEKO FLAM, ES: 2, MW: 18, K: 8). Erzeugt einen Flamenkegel mit einem Öffnungswinkel von 60 Grad und einer Reichweite von 4 Feldern. Betroffene werden von einem Flächeneffekt durch Feuer der Stärke 2 getroffen.
\end{itemize}

\begin{itemize}
\item Eislanze (ORT TEL VAS FIR, ES: 3, MW: 21, K: 11). Erzeugt ein Geschoss mit Effektstärke 3. Das getroffene Ziel bekommt den Zustand ''gelähmt'' in Höhe von 3. (Modifizierung des Farbworts kann andere Zustände produzieren).
\end{itemize}

\begin{itemize}
\item Magisches Schild (UR MAGI KONIU, ES: 3, MW 18, K: 8). Erzeugt ein Schild der Effektstärke 3 ohne weitere Schwächen oder Stärken.
\end{itemize}

\subsubsection{Magusgrimoir der Stärkung}

Die Zauber hier können durch austauschen der Farbworte modifiziert werden. So können z.B. analog die andern Attribute
gesteigert werden.

\begin{itemize}
\item Willenskraft steigern (NOX KONFAR, ES: 3, MW: 18, K: 8). Steigert das Attribut um 3.
\item Flammenrüstung (UR FLAM KONFAR, ES: 3, MW: 19, K: 9). Erschafft eine Flammenrüstung die Feuerschaden um 6 reduziert.
\item Magische Rüstung (UR FERRUM KONFAR, ES: 3, MW: 20, K: 10). Schütz das Ziel wie eine Plattenrüstung indem physischer Schaden um 6 reduziert wird (aber keine Behinderung, ersetzt getragene Rüstung).
\item Magischer Schutz (ORT UR MAGI KONFAR, ES: 3, MW: 20, K: 10). Das so verzauberte Ziel reduziert eingehenden Schaden um 3 (Das ist kompatibel mit Rüstung). Kann auch auf magische Schilde gesprochen werden, diese reduzieren eingehenden Schaden um 3 bevor sie den Schaden von ihren Schildpunkten abziehen.
\end{itemize}

\subsubsection{Meistergrimoir der Heilung}

Dieses Grimoir enthält mächtige Heilmagie die je nach Setting nur eingeschränkt oder garnicht funktioniert. Auch
andere oder zusätzliche Energiequellen wären denkbar.

\begin{itemize}
\item Ernähren (CENSA MANI KONFAR, ES: 1, MW: 18, K: 8). Gibt dem menschengroßen Ziel genug Nahrung und Wasser (Lebensenergie) für einen Tag. Bei extremer Belastung weniger. Hunger und Durst spührt das Ziel trotzdem.
\item Luftloser Atem (CENSA LITAX CONDUC MANI KONFAR, ES: 1, MW: 19, K: 9). Der Charakter kann für 10min auf das Atmen verzichten ohne dabei Einschränkungen zu erleiden.
\item Große Heilung (CENSA MANI, ES: 2, MW: 20, K: 10). Starke Heilung die auch Körperteile bis zur Größe einer Hand oder eines Fußes wiederherstellt. Mit höherer Effektstärke sind auch größere Körperteile möglich.
\item Wiederbelebung (DIVIN MANI CONDUC DIVIN HUMI, MW: 30, ES: 2, K: 20). Göttliche Magie die einen Toten zurück bringen kann. Die Effektstärke regelt wie ''weit'' der zu Erweckende weg ist (Zeit, Alter, will er aus einem Totenreich zurück?). Dieser Zauber bleibt klar unter Spielleitervorbehalt. Es könnte aber auch auf das zweite DIVIN verzichtet werden (Mindestwurf und Kosten um 6 erleichtert). Ob es permanente Folgen am Körper oder Psyche gibt regelt ebenfalls das Setting.
\end{itemize}

\subsubsection{Meistergrimoir der Illusion}

Für Illusionen die den Geist betreffen kann je nach Setting auch die HUMI Rune (im Aspekt Geist, Verstand oder
Vernunft) Rune verwendet werden.

\begin{itemize}
\item Böse Ahnung (VAS SYMA ANIMA KONIU, ES: 1, MW: 19, K: 9). Erschafft ein übernatürliches gespenstisches Gefühl. Die Illusion wirkt auch unterbewusst und kann z.B. bewirken das Charaktere einen Ort meiden es sei denn das sie gute Gründe haben diesen aufzusuchen. Dieser Zauber könnte auch verwendet werden um andere Gefühle zu beschreiben.
\end{itemize}

\begin{itemize}
\item Illusionskutsche (SYMA INGVA SICR CONDUC VAS FERA KONIU, ES: 2, MW: 27 K: 17). Erschafft die Illusion einer Kutsche inklusive Zugtiere. Die Kutsche kann sich auch nach Wahl des Zaubernden bewegen.
\end{itemize}

\begin{itemize}
\item Doppelgänger (ORT VAS SYMA HUMI KONIU, ES: 2, MW: 22, K: 12) Erschafft einen Doppelgänger einer Person. Dieser kann vom Zaubernden gesteuert werden und so z.B. als Ablenkung oder Bote dienen.
\end{itemize}

\begin{itemize}
\item Spiegelbilder (TRIA SYMA HUMI KONIU, ES: 2, MW: 23, K: 13). Erschafft drei Illusionen von sich selbst. Diese simplen Illusionen führen alle Bewegungen des Charakters aus. Sie können im Kampf verwendet werden um die echte Position zu verschleiern und z.B. Gelegenheitsangriffe auszulösen. Illusionen die vom Feind getroffen werden sind von diesem automatisch durchschaut.
\end{itemize}

\begin{itemize}
\item Traumbild (ORT HORA CONDUC INGVA SYMA ANIMA, ES: 2, MW: 30, K: 20). Sendet eine Traumbotschaft die zwar nicht räumlich jedoch in ihrer Komplexität durch die Effektstärke beschränkt ist.
\end{itemize}

\begin{itemize}
\item Maske (VAS SYMA HUMI KONFAR, ES: 2, MW: 22, K: 12). Ändert das Aussehen der verzauberten Person in das einer Anderen. Sollte es starke Unterschiede zwischen Beiden geben so ist eine höhere Effektstärke nötig. Sollten Details wie Gang und Gestik mitgetäuscht werden so sollte INGVA statt VAS verwendet werden. (Der prachklang ist in diesem Zauber enthalten.
\end{itemize}

\subsubsection{Meistergrimoir der Kampfmagie}

Modifizierung der Farbworte sind auch hier möglich.

\begin{itemize}
\item Feuersturm (TRIA ORT TEL FLAM INGVA LEKO, ES: 4, MW: 36, K: 26). Falls es einem Magier gelingt diesen Zauber zu wirken so kann er drei Feuerbälle mit Effektstärke 4 und verdreifachten Radius schleudern. Gegner die direkt getroffen werden bekommen zusätzlichen Schaden durch Effektstärke und Willenskraft. Gegner die im Explosionsradius verschiedener Feuerbälle stehen bekommen aber nur einmal Schaden.
\end{itemize}

\begin{itemize}
\item Eissturm (ORT VAS FIR VAS LEKO KONIU, ES: 3, MW: 25, K: 15). Der Zaubernde erschafft eine Zone starker Kälte (Explosionsregeln für die Effektstärke). Jeder Gegner der sich in der Zone am Ende seiner Runde aufhält bekommt Schaden als wenn er von der Explosion getroffen wurde). Ausserdem erleidet er des Zustand gelähmt in Höhe der für ihn geltenden Effektstärke. Der Zauber muss aufrecht erhalten werden.
\end{itemize}

\begin{itemize}
\item Meteorbeschwörung (OBSEK INGVA METEO, ES: 3, MW: 24, K: 14). Falls das Setting erlaubt stürzt nach 3 Runden (Effektstärke) ein Meteor auf das vom Zaubernden beim Zaubern bestimmte Feld und verursacht elementaren Explosionsschaden mit verdreifachten Radius (INGVA).
\end{itemize}

\begin{itemize}
\item Wall (OBSEK INGVA UR VAS EBOR, ES: 3, MW: 25, K: 15). Falls das Setting Beschwörung erlaubt so kann der Magier 27 (OBSEK UR,3 multipliziert mit INGVA, 3 und der Effektstärke) Felder mit einer Erdwand versehen. Charaktere werden dabei zur Seite gedrückt. Alternativ könnten 9 Felder mit einer Mauer in Höhe von 6m belegt werden. Eventuell gibt es weitere Einschränkung an die Form oder Geometrie der Erdwand.
\end{itemize}

\begin{itemize}
\item Schildregeneration (TRIA ORT AUM UR MAGI KONIU, ES: 3, MW: 24, K: 14). Kann drei magische Schilde regenerieren. Diese erhalten jede Runde 6 Schildpunkte zurück.
\end{itemize}

\subsubsection{Meistergrimoir der Stärkung}

Einige hier vorgestellten Zauber können durch Austausch der Farbe ein anderes Attribut verbessern.

\begin{itemize}
\item Magierstärkung (LUNA KONFAR, ES: 4, MW: 23, K: 13). Steigert Intelligenz und Willenskraft um je 2.
\item Hand des Schicksal (ANIMA KONFAR, ES: 2, MW: 20, K: 10). Gibt dem Ziel zwei Wiederholungswürfe. Die müssen aber in dieser Szene verbraucht werden.
\item Sammeln (HUMI KONFAR, ES: 2, MW: 19, K: 9). Gibt dem Ziel zwei Fokuspunkte. Diese müssen in der Szene verbraucht werden.
\item Nase des Spürhunds (FERA NOX KONFAR, ES: 2, MW: 19, K: 9). Gibt einen Raubtiersinn (z.B. Geruchssinn). Dieser Sinn ermöglicht es Sinneseindrücke wahrzunehmen die vorher nicht wahrnehmbar waren. Für alle Wahrnehmungsproben mit dem verbesserten Sinn gibt es 2W.
\item Schnelligkeitszauber (TORA KONFAR ES: 3, MW: 23, K: 13). Gibt eine zusätzlich Aktionshandlung (aber keine zusätzliche Bewegungshandlung oder reaktive Handlung). Geringere Effektstärken geben eine zudätzliche Bewegungsandlung (Effektstärke 2) oder reaktive Handlung (Effektstärke 1).
\end{itemize}

\subsubsection{Sammelgrimoir der Beschwörung}

Die Zauber hier müssen ans Setting angepasst werden. Dazu müssen Fragen der Natur von Beschwörungsmagie geklärt werden.
Die Vorschläge hier sind generisch. Beschworene Gegenstände halten einen Tag. Übernatürliche Kreaturen müssen beim
nächsten Sonnenaufgang oder Untergang wieder zurück. Sie helfen nicht freiwillig und müssen überzeugt oder gezwungen
werden.

\begin{itemize}
\item Langschwert beschwören (OBSEK VAS GLADI FERRUM, ES: 2, MW: 19, K: 9). Beschwört ein qualitativ gutes Langschwert (oder eine beliebige andere Nahkampfwaffe).
\end{itemize}

\begin{itemize}
\item Bogen beschwören (OBSEK GLADI CONDUC TEL, ES: 2, MW: 18, K: 8). Beschwört einen Kurzbogen oder eine vergleichbare Geschosswaffe.
\end{itemize}

\begin{itemize}
\item Feuerpfeile beschwören (OBSEK TEL FLAM, ES: 2, MW: 16, K: 8). Beschwört 20 Pfeile die Feuer statt physischen Schaden verursachen.
\end{itemize}

\begin{itemize}
\item Eisdolch beschwören (OBSEK GLADI FIR, ES: 1, MW: 16, K: 6) Beschwört einen Dolch der Eisschaden statt physischen verursacht. Die Qualität des Dolches ist mangelhaft.
\end{itemize}

\begin{itemize}
\item Kleines Tier rufen (OBSEK BET FERA, ES: 1, MW: 14, K: 4). Ruft ein kleines Tier. Der Ruf ist relativ schwach, das Tier wird sich nur vorsichtig nähern.
\end{itemize}

\begin{itemize}
\item Raubtier rufen (OBSEK VAS FERA NOX, ES: 3, MW: 24, K: 14). Ruft ein Raubtier. Der Ruf ist stark das Tier wird schnell und direkt erscheinen.
\end{itemize}

\begin{itemize}
\item Geist beschwören (OBSEK ANIMA, ES: 2, MW: 20, K: 10). Ruft einen lokal vorhandenen Geist und gibt ihm etwas Substanz (Effektstärke). Wie der Geist reagiert bleibt dem Geist überlassen.
\end{itemize}

\begin{itemize}
\item Feuerelementar beschwören (OBSEK ANIMA FLAM, ES: 2, MW: 21, K: 11). Beschwört ein menschengroßes Feuerelementar. Dieses wird nicht glücklich sein über den Ruf...
\end{itemize}

\begin{itemize}
\item Gebundenes Luftelementar beschwören (OBSEK ANIMA LITAX AUM, ES: 2, MW: 22, K: 12). Beschwört ein Luftelementar und gibt dem Beschwörer einen Hebel das Elementar in den Dienst zu zwingen.
\end{itemize}

\begin{itemize}
\item Todesgeist rufen (OBSEK ANIMA MORT, ES: 2, MW: 22, K: 12). Beschwört einen Geist der Toten. Ideal um Zombies zu erschaffen, weniger für ein Gespräch.
\end{itemize}

\begin{itemize}
\item Geist der Menschen rufen (OBSEK ANIMA HUMI, ES: 2, MW 26, K: 16). Beschwört einen Geist der Menschen. Dieser kann ein Ahnengeist, ein Schutzgeist oder vielleicht sogar direkt ein Verstorbener sein.
\end{itemize}

\begin{itemize}
\item Tiergeist beschwören (OBSEK ANIMA FERA NOX AUM, ES: 2, MW: 26, K: 16). Ruft einen Raubtiergeist und gibt ihm etwas Substanz. Durch die Kontrollrune AUM sollte der Beschwörer den Geist auf seine Feinde hetzen können.
\end{itemize}

\subsubsection{Sammelgrimoir der Schwächungsmagie}

Die hier vorhandenen Zauber können durch Tauschen der elementaren Farbe andere Attribute betreffen. Den Zaubern kann
mit Willenskraft widerstanden werden. Es wird die Willenskraft vor etwaiger Modifizierung durch den Zauber verwendet.
Durch die Anpassung der Effektstärken können die Zauber ebenfalls modifiziert werden.

\begin{itemize}
\item Willenskraft senken (ORT AN NOX KONFAR, ES: 2, MW: 17, E: 7) Senkt das Attribut um 2.
\item Intelligenz senken (ORT AN PRIX KONFAR, ES: 2, MW: 17, E: 7) Senkt das Attribut um 2.
\item Wahrnehmung senken (ORT AN LITAX KONFAR, ES: 2, MW: 17, E: 7) Senkt das Attribut um 2.
\item Stärke senken (ORT AN FLAM KONFAR, ES: 2, MW: 17, E: 7) Senkt das Attribut um 2.
\item Geschicklichkeit senken (ORT AN WAKU KONFAR, ES: 2, MW: 17, E: 7) Senkt das Attribut um 2.
\item Konstitution senken (ORT AN EBOR KONFAR, ES: 2, MW: 17, E: 7) Senkt das Attribut um 2.
\item Persönlichkeit senken (ORT AN PRIX KONFAR, ES: 2, MW: 17, E: 7) Senkt das Attribut um 2.
\item Vergiften (ORT MORT KONFAR, ES: 2, MW: 17, E: 7). Wird dem Zauber nicht widerstanden so bekommt das Ziel 4 Punkte des Zustands vergiftet.
\item Rüstung brechen (ORT AN UR KONFAR, ES: 3, MW: 20, E: 10). Jeglicher Schutz (Schadensreduzierung durch z.B. Rüstung) ist um 3 Punkte reduziert
\item Verrostete Rüstung (ORT AN UR FERRUM KONFAR, ES: 2, MW: 19, E: 9) Rüstungsschutz ist um 4 Punkte reduziert. (Ziel ist die Rüstung und nicht der Charakter, der Charakter kann trotzdem eine Widerstandsprobe für von ihm verwendete Gegenstände versuchen).
\item Unhandliche Waffe (ORT AN AUM GLADI KONFAR, ES: 2, MW: 19, E: 9). Die verzauberte Waffe trifft um 2 Punkte schlechter. (Widerstand durch den Charakter der die Waffe führt möglich).
\item Schwächen (ORT AN MANI KONFAR, ES: 2, MW: 18, E: 8). Die Behinderung des Charakters steigt um 2 Punkte (-2W auf alle aktive Proben und reduziert Geschwindigkeit um 2).
\end{itemize}

\subsubsection{Sammelgrimoir der Telepathie}

Das Anwenden dieser Zauber kann unter Strafe stehen. Bedenkt das vierte Gesetz der Magie! Die Zauber hier sind mit
dem Ziel Mensch vorgestellt. Analoge Zauber für Tiere und Geister wären ebenfalls denkbar.

\begin{itemize}
\item Emotion kontrollieren (AUM HUMI KONIU, ES: 2, MW: 20, K: 10). Kontrolliert die Emotionen des Ziels. Dieser Zauber wirkt auch nach dem Beenden nach. Die Effektstärke 2 sagt wie weit der Gemütszustand verändert werden kann. Z.B. Wut über Verärgerung zu Ruhig ändern.
\end{itemize}

\begin{itemize}
\item Gedankenkontrolle (VAS AUM HUMI KONIU, ES: 2, MW: 22, K: 12). Kontrolliert die Gedanken und Handlungen des Ziels. Nach Beenden des Zaubers wird dem Ziel klar das es kontrolliert wurde (oder etwas nicht stimmt, evtl ist das Ziel verwirrt).
\end{itemize}

\begin{itemize}
\item Erinnerung manipulieren (INGVA AUM HUMI KONIU, ES: 2, MW: 24, K: 14). Verändert die Erinnerung des Ziels. Die Effektstärke gibt das Ausmaß die Dauer der Zaubers die Menge der veränderten Erinnerungen an.
\end{itemize}

\begin{itemize}
\item Empathie (SCIEN HUMI, ES: 2, MW: 19, K: 9). Ermöglicht das ''sehen'' von Emotionen.
\end{itemize}

\begin{itemize}
\item Gedanken lesen (VAS SCIEN HUMI, ES: 2, MW: 21, K: 11). Ermöglicht es die bewussten Gedanken zu ''sehen''.
\end{itemize}

\begin{mdframed}[hidealllines=true, backgroundcolor=black!10]
\subparagraph{Spielleiterhinweis}

Die Zauber hier haben das Potential ganze Plots zu zerstören. Eine Möglichkeit diese Zauber dennoch zu nutzen wäre
es sie unzuverlässig zu machen. D.H. jeder Mensch ist anders und reagiert anders auf die Zauber. Beispiele:

\begin{itemize}
\item Generische NSCs lassen sich kontrollieren und lesen, NSCs mit einem Namen nicht.
\item Verdeckter Wurf des Spielleiters und dann entscheiden. So können wichtige NSCs geschützt werden.
\item Relativ weit verbreitete Talismane und Schutzzauber die die Wirkung unterbinden.
\end{itemize}

\end{mdframed}
\subsubsection{Sammelgrimoir der Umweltkontrolle}

Effektstärken skalieren linear die veränderte Umwelt. Ebenso können Verstärkungsworte die ''Menge'' der
manipulierten Umwelt multiplikativ verändern (VAS = 2, INGVA = 3, BET = 1/2).

\begin{itemize}
\item Erde bewegen (ORT AUM EBOR KONIU, ES: 1, MW: 14, E: 4). Bewegt ca 1 Kubikmeter lose Erde pro Runde. Besondere Kraft besitzt der Zauber nicht. Bei härteren Gestein wird die Menge entsprechend reduziert (z.B. Granit sollte nur 1/5 Kubikmeter sein)
\end{itemize}

\begin{itemize}
\item Einfrieren (ORT WAKU MUTAR FIR KONFAR, ES: 2, MW: 18, E: 8). Verwandelt eine Fläche von ca 10 Quadratmeter in Eis das einen Menschen tragen kann (und evtl wegschwimmt oder bricht wenn die Geometrie schlecht gewählt ist). Im Tausch für eine dünnere Eisdecke könnte eine größere Fläche eingefroren werden.
\end{itemize}

\begin{itemize}
\item Windstoss (ORT AUM VAS LITAX, ES: 3, MW: 21, E: 11). Erzeugt einen Windstoss der die selbe Kraft wie ein Mensch mit Stärke 3 hat. Dieser Zauber kann verwendet werden um einen Gegner von den Füssen zu reissen. Andere Elemente (z.B EBOR) könnten die Stärke erhöhen sind aber eingeschränkt auf Ziele mit Bodenkontakt. Fliegende Ziele können Abstürzen (und sollten einen Malus von 3W auf Proben zum Stabilisieren haben, zumindest solange sie im Einflussbereich sind).
\end{itemize}

\begin{itemize}
\item Funke (BET PRIX KONIU, ES: 1, MW: 10, E: 0). Erzeugt ein Funken Licht der ca ein Kampffeld direkt und die Nachbarfelder schwach beleuchtet.
\end{itemize}

\begin{itemize}
\item Lichtzauber (SYMA PRIX KONIU, ES: 2, MW 14, E: 4). Erzeugt ein Zauber der innerhalb von 6m gut und innerhalb von 12m schach leuchtet. Bei den Lichtzaubern ist zu bemerken das der Zaubernde die Wahl der Farbe hat und das die Aufrechterhaltungskosten nicht jede Kampfrunde sondern für jede Szene zu zahlen sind.
\end{itemize}

\begin{itemize}
\item Wärmezauber (BET FLAM KONIU, ES: 2, MW: 12, E: 2). Wärmt das Ziel so als wenn die Umgebungstemperatur 20 Kelvin wärmer wäre. Ene Variante mit KONFAR statt KONIU wäre auch denkbar (MW: 13, E: 3)
\end{itemize}

\begin{itemize}
\item Telekinese (ORT AUM SICR KONIU, ES: 2, MW: 18, E: 8). Bewegt einen Gegenstand bis zur Größe eines Stuhls. Die Effektstärke regelt die Geschwindigkeit (z.B.: 100 Felder pro Runde pro Kilogramm Gewicht oder einfach 4 Felder pro Runde).
\end{itemize}

\begin{itemize}
\item Manipulation (ORT VAS AUM SICR KONIU, ES: 1, MW: 18, E: 8). Kann Gegenstände manipulieren. Z.B. eine Klinge drücken einen Bolzen lösen oder eine Gürtelschnalle öffnen. Effektstärke beschreibt dabei Geschick oder Stärke. Komplexe Manipulationen sind nicht möglich.
\end{itemize}

\begin{itemize}
\item Reparieren (SICR AN MUTAR SICR KONIU, ES: 1, MW: 16, E: 6). Verwandelt den Gegenstand zurück und repariert ihn so. Wie lange der Zauber aufrechtzuerhalten ist oder welchen Schaden man reparieren kann muss mit dem Spielleiter abgesprochen werden.
\end{itemize}

\subsubsection{Sammelgrimoir des Waldläufers}

Ein paar Zauber zum Überleben in der Natur.

\begin{itemize}
\item Pflanzenkontrolle (ORT AUM FLORA KONIU, ES: 2, MW: 18, K: 8). Kann Pflanzen bewegen und so Wege durchs Unterholz ermöglichen oder Spuren verbergen.
\end{itemize}

\begin{itemize}
\item Tierfreund (ORT AUM FERA, ES: 2, MW: 19, K: 9). Verändert die Einstellung des Ziels um ''2'' Punkte. Das sollte reichen um Raubtiere ihrer Wege ziehen zu lassen.
\end{itemize}

\begin{itemize}
\item Unterstand (OBSEK VAS SICR, ES: 2, NW: 19, K: 9). Beschwört ein Zelt für zwei Personen.
\end{itemize}

\begin{itemize}
\item Augen des Adlers (FERA LITAX KONFAR, ES: 2, MW: 19, K: 9). Gibt Adleraugen die übermenschliche Weitsicht ermöglichen und bei entsprechenden Wahrnehmungsproben 2w Bonus liefern.
\end{itemize}

\begin{itemize}
\item Wasser finden (VAS SCIEN WAKU KONIU, ES: 3, MW: 19, K: 9). Ermöglich Wasser zu sehen. Wasseradern können ebenfalls gesehen werden. Wasserquellen hinterlassen Spuren in der Vegetation die mit diesem verstärkten Wahrnehmungszauber gesehen und verfolgt werden können.
\end{itemize}

\subsubsection{Verbotenes Buch}

Diese finstere BUch ist größtenteils zerstört. Zwischen den Seiten finden sich neben den Hinweisen zu grausamen Zaubern
auch genug Warnungen das ein Magier schnell zum Geächteten oder Gejagten wird sollte er diese Magie einsetzen und sich
erwischen lassen.

\begin{itemize}
\item ORT CENSA MORT KONFAR: Permanente, chronische und tödliche Krankheit.
\item ORT DIVIN CENSA MORT KONFAR: Eine solche Krankheit nur mit der Fähigkeit sich zu verbreiten
\item ORT TEL AN HUMI. Zerstört Menschen und verkrüppelt Diese falls sie das Geschoss überleben.
\item ORT TEL AN ANIMA. Zerstört den Geist oder Geistwesen. Verursacht schwere psychische Schäden bei Menschen.
\item ORT CENSA AN HUMI DIVIN KONFAR. Tödlicher Fluch.
\item ORT SYMA MORT CONDUC HUMI. Spruch für die Folter.
\item ORT AN DIVIN HUMI. Entfernt den göttlichen Schutz eines Menschen.
\end{itemize}

\begin{center}
\section{Verzauberung und magische Gegenstände}
\end{center}

Diese Kapitel beschäftigt sich damit wie Charaktere verzauberte Gegenstände nutzen oder auch selbst erstellen können.
Verzauberte Gegenstände stellen einen magischen Effekt bereit, für die Funktionsweise und Regeln dieser Effekt wird
das Kapitel Magie empfohlen.

\begin{itemize}
\item Die Magie in Verzauberungen kann nur funktionieren wenn der magische Effekt mit Energie versorgt wird (analog von Zaubersprüchen die Kosten auf dem geistigen Monitor verursachen). Diese Energie kann verschieden bereitgestellt werden.
\item Das Auslösen kostet eine Handlung (falls nicht anderweitig im Gegenstand beschrieben). Diese Handlung kann aber auch Teil einer anderen Handlung sein (falls es Sinn Macht. Z.B. Feuerschwert aktiviert sich beim Waffe bereit machen). Diese Auslösehandlung kann Gelegenheitsangriffe provozieren falls sie nicht irgendwie den Nahkampf bedroht.
\item Gegenstände haben eigene Attribute. Für magische Effekte ist meist Willenskraft nötig. Gegenstände ohne Willenskraft haben eine von 1 falls sie sonst nicht funktionieren.
\item Gegenstände haben wie Zaubersprüche Effektstärken. Diese beschreiben wie stark die Verzauberung ist. (Willenskraft kann Boni geben oder bestimmt wie lange ein Effekt wirkt).
\item Für die Beschreibung der magischen Effekte wird auf das Runensystem zurück gegriffen. Beschreibung sind im entsprechenden Kapitel (Regeln der Magie und Runen) zu finden. Im Kapitel Kampf werden magische Kampfeffekte behandelt.
\end{itemize}

\begin{center}
\subsection{Benutzen eines magischen Gegenstands}
\end{center}

Sofern der nicht anders bestimmt muss der Gegenstand vom verwendenden Charakter bereitgemacht werden und mit einer
aktiven willentlichen Handlung ausgelöst werden. Einige Gegenstände wirken passiv oder haben unbegrenzte
Wirkungsdauer. Sofern passend kann man davon ausgehen das Charaktere diese aktiviert haben.

In einigen Fällen wird für die Wirkung des Effekts eine Würfelprobe benötigt. Falls beispielsweise bestimmt werden
muss ob ein magisches Geschoss einen Feind trifft. Bei diesen Proben kann als Fertigkeit ''Verzaubern'' verwendet werden.
Als Attribut für die diese Probe kommen neben Intelligenz auch andere Attribute in Frage. Fernkampfmagie könnte aus
einem verzauberten Gegenstand zum Beispiel mit Wahrnehmung gewürfelt werden.
Magische Waffen werden natürlich mit ihren passenden Fertigkeiten geführt. Ihre Effekte könnten mit Zustimmung der
Spielleitung aber auch mit der passenden Waffenfertigkeit ausgelöst werden. So könnte ein Feuerball aus einem
verzauberten Schwert mit Wahrnehmung und Schwertkampf ins Ziel gebracht werden. Die Spielleitung kann (und je nach
Setting sollte) so ermöglichen das nichtmagische Charaktere Zugang zu nutzbaren magischen Effekten haben.

\begin{center}
\subsection{Energie für Verzauberungen}
\end{center}

Energiekosten können wie folgt bereitgestellt werden

\begin{itemize}
\item Interner Energiespeicher. Dieser kann mit Splittern aufgeladen werden (maximal 2 Splitter pro Kampfrunde).
\item Ein in die Verzauberung eingebauter Kristall. Nur ein Kristall pro Verzauberung möglich und ein Kristall kann nicht in mehreren Verzauberungen integriert sein. Ein Kristall bringt seine Stärke an Energie jede Runde. Die Kristallkosten können sowohl die Auslöse- als auch die Aufrechterhaltungskosten reduzieren oder ganz bezahlen.
\item Beide Methoden oben können beliebig kombiniert werden.
\item Einfache Verzauberungen können auch ohne Kristall und Speicher direkt mit Splittern aufgeladen werden. Sie können so nur einmal ausgelöst werden. Möglicherweise dauert dann das Bereitstellen der Energie auch mehr als ein Runde.
\item Fortführende Effekte haben eigene Kosten (meist 1/3 der Auslösekosten) diese können aber nur von Speichern oder Kristallen bezahlt werden.
\item Es ist möglich mit Splittern auszulösen und mit Kristall den fortlaufenden Effekt zu zahlen.
\item Energiekosten können nicht unter 1 fallen.
\end{itemize}

\begin{center}
\subsection{Erstellung eines Artefakts}
\end{center}

Um ein Artefakt zu erstellen sind zwei Schritte nötig: Planung und Durchführung. In der Planung überlegt sich der
Charakter (und sein Spieler) was er für einen Gegenstand erschaffen möchte. Dann wird bei der Durchführung mittels
der Fertigkeit Verzaubern bestimmt ob der Gegenstand erfolgreich erstellt wurde.

\subsubsection{Planung}

Plannung läuft in verschiedenen Schritten. Dabei sammelt man Komplexität (K) für den magischen Effekt an.
Analog zur Runenmagie werden daraus die Energiekosten (E) und Schwierigkeit ermittelt. Schwierigkeit und
Energiekosten können dann seperat modifiziert werden.

Nachfolgende Punkte müssen nacheinander abgegangen werden:

\paragraph{Effektbeschreibung.}

Der Effekt wird mit Hilfe des Runenmagiesystems beschrieben. Wie Zaubersprüche funktionieren und welche Runen wie
einen Effekt beschreiben ist in den entsprechenden Kapiteln zu finden. Sobald der Effekt beschrieben ist hat der
Zauber eine Komplexität. Diese Komplexität ist die Komplexität der Verzauberung, sie wird in den nächsten Schritten
modifiziert. Es versteht sich von selbst das der verzaubernde die Runen des Zaubers beherschen muss.

\paragraph{Effektstärke}

Damit ein Zauber funktioniert benötigt er eine Effektstärke. Zauber mit Effektstärke von 0 sind eher als Gimmick zu
sehen und haben nur sehr eingeschränkten Nutzen/Effekt. Bei Effektstärke 0 sollte vorher diskutiert werden ob die
Verzauberung wie gewünscht funktioniert (FLAM mit Effektstärke 0 -> Feuerzeug, FIR mit Effektstärke 0 zum Kühlen von
Getränken, etc. Alternativ könnte meist auch BET und Effektstärke 1 verwendet werden).

Mit Festlegung der Effektstärke erhöht sich die Komplexität wie folgt


\begin{small}
\begin{tabular}{|m{3cm}|m{1cm}|m{3cm}|}
\hline
\textbf{Effektstärke(X)}&\textbf{K}&\textbf{Vorraussetzung}\\
\hline
\hline
0&0&Verzaubern 1\\
\hline
1&1&Verzaubern 2\\
\hline
2&3&Verzaubern 3\\
\hline
3&6&Verzaubern 4\\
\hline
4&10&spezieller Weg\\
\hline
5&15&spezieller Weg\\
\hline
\end{tabular}
\end{small}

\paragraph{Kosten und Schwierigkeit}

Aus der Komplexität wird jetzt Schwierigkeit und Energiekosten ermittelt. Alle Modifikatoren die die Komplexität wie
z.B. erlernte Wege werden hier berücksichtigt. Jede Modifikation der Komplexität modifiziert auch die Energiekosten
und Schwierigkeit entsprechend.

\begin{itemize}
\item Schwierigkeit ist 10 + Komplexität.
\item Energiekosten sind gleich der Komplexität
\end{itemize}

\paragraph{Energiemanagement}

An dieser Stelle muss entschieden werden ob der Gegenstand einen Energiespeicher hat. Falls ein solcher eingebaut
wird erhöht sich die Schwierigkeit (und nur die Schwierigkeit). Zusätzlich kann sich entschieden werden das die
Verzauberung einen eingebauten Kristallen hat. Das erhöht nicht den Schwierigkeitsgrad, reduziert aber die Kosten für
das Auslösen und etwaiges Aufrechterhalten um seine Stufe.


\begin{small}
\begin{tabular}{|m{3cm}|m{4cm}|m{3cm}|}
\hline
\textbf{Speicher}&\textbf{Schwierigkeitserhöhung}&\textbf{Vorraussetzung}\\
\hline
\hline
10&+1&Verzaubern 2\\
\hline
20&+2&Verzaubern 2\\
\hline
30&+3&Verzaubern 3\\
\hline
40&+4&Verzaubern 4\\
\hline
50&+5&Verzaubern 5\\
\hline
\end{tabular}
\end{small}

\paragraph{Feintuning}

Folgende Techniken kann der Verzauberer verwenden um die Verzauberung anzupassen. Weiter Techniken können über Wege
gelernt werden.


\begin{small}
\begin{tabular}{|m{4cm}|m{3cm}|m{4cm}|m{2cm}|}
\hline
\textbf{Technik}&\textbf{Schwierigkeit}&\textbf{Beschreibung}&\textbf{Rune}\\
\hline
\hline
Expertenverzauberung&+1&-1 benötigte Energie&?\\
\hline
Meisterverzauberung&+2&-2 benötigte Energie&AROL\\
\hline
kontrollierte Verz.&-1&+1 benötigte Energie&?\\
\hline
sichere Verzauberung&-2&+2 benötigte Energie&KOL\\
\hline
starke Verzauberung&+1&Gegenstand hat Will 1&\\
\hline
sehr starke Verzauberung&+2&Gegenstand hat Will 2&\\
\hline
extrem starke Verzauberung&+4&Gegenstand hat Will 3&\\
\hline
\end{tabular}
\end{small}

\subsubsection{Herstellen eines Artefakts}

Nach dem der Charakter den Gegenstand geplant hat so muss er fertig gestellt werden. Prinzipiell ist dabei die
Schwierigkeit mit einer Fertigkeitsprobe auf Intelligenz und Verzaubern zu erreichen. Je nach Setting wären auch
andere Attribute wie Geschicklichkeit denkbar. Folgende Regeln gelten:

\begin{itemize}
\item Das Herstellen des Artefakts dauert pro Punkt Komplexität 2h. Für kleine oder sehr kleine Gegenstände kann der Zeitraum verdopplet, verdreifach oder ähnlich modifiziert werden.
\item Es wird eine Verzauberungswerkstatt benötigt. Fehlt diese so kommt ein Malus von 3W zu tragen wenn der Gegenstand überhaupt hergestellt werden kann. Die verwendete Werkstatt kann je nach Qualität den Würfelpool modifizieren.
\item Es werden weitere Werkzeuge oder Werkstätten benötigt um den Gegenstand zu bearbeiten (z.B. Juwellierwerkstatt bei einer Ringverzauberung). Auch hier kann ein Malus von bis zu 3W zum Tragen kommen wenn die Werkstatt von minderer Qualität ist.
\item Schlägt die Probe fehl so werden Materialien beschädigt oder zerstört. Bei einem kritischen Fehlschlag kann man von der Zerstörung der Materialien ausgehen. Bei anderem Misslingen lässt sich nach Entscheidung der Spielleitung noch etwas retten (10\%-50\% der Materialien sollten ersetzt oder repariert werden).
\item Bei aussergewöhnlichem Gelingen kann die Spielleitung kleinen Bonuseffekten zustimmen. So könnte die Willenskraft des Gegenstands um 1 steigen oder zusätzliche optische Effekte den Gegenstand und seine Nutzung verschönern.
\end{itemize}

\newpage

\begin{center}
\subsection{Artefaktliste}
\end{center}


\begin{small}
\begin{tabular}{|m{2cm}|m{3cm}|m{6mm}|m{6mm}|m{1cm}|m{15mm}|m{2cm}|m{3cm}|}
\hline
\textbf{Gegenstand}&\textbf{Effekt}&\textbf{ES}&\textbf{Wk}&\textbf{E max}&\textbf{E jetzt}&\textbf{Laden}&\textbf{Sonstiges}\\
\hline
\hline
1.&&&&&&&\\
\hline
2.&&&&&&&\\
\hline
3.&&&&&&&\\
\hline
4.&&&&&&&\\
\hline
5.&&&&&&&\\
\hline
6.&&&&&&&\\
\hline
7.&&&&&&&\\
\hline
8.&&&&&&&\\
\hline
9.&&&&&&&\\
\hline
10.&&&&&&&\\
\hline
11.&&&&&&&\\
\hline
12.&&&&&&&\\
\hline
13.&&&&&&&\\
\hline
14.&&&&&&&\\
\hline
15.&&&&&&&\\
\hline
16.&&&&&&&\\
\hline
17.&&&&&&&\\
\hline
18.&&&&&&&\\
\hline
19.&&&&&&&\\
\hline
20.&&&&&&&\\
\hline
21.&&&&&&&\\
\hline
22.&&&&&&&\\
\hline
23.&&&&&&&\\
\hline
24.&&&&&&&\\
\hline
25.&&&&&&&\\
\hline
26.&&&&&&&\\
\hline
27.&&&&&&&\\
\hline
28.&&&&&&&\\
\hline
29.&&&&&&&\\
\hline
30.&&&&&&&\\
\hline
\end{tabular}
\end{small}

\newpage

\begin{center}
\section{Alchemie}
\end{center}

Mit Hilfe der Alchemie ist es möglich Verbrauchsgegenstände herzustellen die einen magischen Effekt wirken (Tränke).
Alchemistische Erzeugnisse wirken ihre magischen Effekte über die Regeln der Runenmagie. Die grundsätzlichen Regeln
der Runenmagie sollten für das Verständnis der Tränke bekannt sein. Allerdings gibt es einige Ausnahmen:

\begin{itemize}
\item Grundsätzlich fängt die Wirkung einer Substanz an wenn sie freigesetzt wird. Z.B wenn sie geworfen, ausgegossen oder getrunken wird.
\item Zauberbeginnworte fürs Ziel entfallen. Der magische Effekt wirkt dort wo die alchemistische Substanz hinkommt.
\item Grundsätzlich wird die Effektstärken durch die verwendeten Substanzen gestellt. D.H. für jedes alchemistische Erzeugnis wird bei der Herrstellung die Effektstärke festgelegt.
\item Die Willenskraft der Substanz ist wird durch die Erfolgsgrade beim Herstellen bestimmt.
\item Die Dauer der Effekte hängt von der Willenskraft ab (für diesen Fall ist sie immer mindestens 1, auch wenn kein Erfolgsgrad bei der Herstellung vorliegt). KONIU Effekte wirken Willenskraft * 5 Runden lang.
\end{itemize}

\begin{center}
\subsection{Notation einer alchemistischen Substanz}
\end{center}

Es sollten entweder der relevante Effekt exakt aufgeschrieben werden (z.B. Trank macht für 3 Runden unsichtbar) oder
der magische Effekt inklusive Willenskraft und Effektstärke (z.B. Heiltrank MANI, ES 2, Wk 3 der dann 2W4+3
Lebenspunkte heilen würde)

\begin{center}
\subsection{Herstellung einer alchemistischen Substanz}
\end{center}

Bei der Herstellung werden mehrere Zutaten miteinander gemischt. Prinzipiel hat man mindestens eine Grundsubstanz und
eine oder mehrerere Additive.

\paragraph{Grundsubstanz}

Grundsubstanzen sind meist einfach zu beschaffene Flüssigkeiten. Sie stellen im Prinzip das Volumen des
alchemistische Erzeugnis dar. Ausserdem bestimmen sie wie lange man für die Herstellung des Erzeugnis benötigt.

\paragraph{Additiv}

Die schwerer zu beschaffenen Zutaten. Im alchemistischen Prozess werden sie der Grundsubstanz hinzugefügt.

\subsubsection{Schwierigkeit und Kritikalität}

Hat sich der Alchemist entschieden welche Substanzen er mischen möchte, so kann der Mindestwurf für das Gelingen
festgelegt werden.

\begin{itemize}
\item Schwierigkeit ist 10 plus Summe der Schwierigkeiten der Zutaten.
\end{itemize}

Eine Prozedur kann aufgrund der beteiligten Effekt kritisch werden. Bei der Effektbestimmung werden deswegen
Kritikalitätspunkte verteilt. Es wird ein zweiter Mindeswurf mit den Kritikalitätspunkten gebildet. Ist die
Kritikalität nicht geschafft geht etwas gründlich schief! Abhängig wie weit die Probe fehlschlägt. Der Prozess
misslingt in jedem Fall und das Ergebnis ist unbrauchbar (möglicherweise auch der Alchimist oder seine Ausrüstung).

Es empfiehlt sich nur eine Probe zu würfeln die gleichzeitig gegen Kritikalität und Schwierigkeit
geht.

\begin{itemize}
\item Schwierigkeit der Kritikalität ist 10 + 2 * Kritikalitätspunkte.
\end{itemize}

\paragraph{Herstellungsdauer}

Herstellungsdauer ist Summe der Schwierigkeiten der Grundsubstanzen in Stunden.

\paragraph{Effektbestimmung}

Bei der Effektbestimmung gibt es folgendes zu beachten:

\begin{itemize}
\item Es werden nur Runen gewertet die sowohl in den Grundsubstanzen als auch in den Additiven vertreten ist.
\item Ergeben die Formworte einen ungültigen Effekt so ist die Prozedur ein Fehlschlag. Es werden ausserdem Kritikalitätspunkte in Höhe der Komplexität der vorhandenen Formworte vergeben.
\item Jedes Farbwort hat Stärken, gekennzeichnet durch `+` oder `-`. `-` nimmt hier die Rolle des AN Worts ein.
\item Alle Farben werden untereinander addiert (Jedes `-` neutralisiert ein `+` oder andersrum, bis eine Farbe nur aus `+` oder  `-` besteht).
\item Positive Farbworte reduzieren ihren elementaren Gegenpart. Negative ihren negativen elementaren Gegenpart. Dabei wird das schwächere Wort vom stärkeren Wort abgezogen und dann auf null gesetzt. Die ursprüngliche Stärke des schwächeren Worts wird ausserdem der Kritikalität hinzugefügt.
\item Die Anzahl der Plus oder der Minus bestimmen die Effektstärke der dazugehörigen Farbe.
\item Das Farbwort mit der größten Effektstärke ist das dominante und kommt zur Geltung. Gibt es zwei oder mehr gleichstarke so ist der Prozess ein Fehlschlag.
\item übrig gebliebene Farbworte bringen je einen Punkt Kritikalität.
\item Die Effektstärke des Erzeugnis ist die Effektstärke der dominanten Farbe.
\item Die Willenskraft sind die Erfolgsgrade gegen die Schwierigkeit.
\end{itemize}

\begin{center}
\subsection{Notation von Rezepten}
\end{center}

Ein Rezept notiert folgende Punkte:

\begin{itemize}
\item Zutaten
\item Schwierigkeit
\item Kritikalität
\item das alchemistische Result
\end{itemize}

\begin{center}
\subsection{Sonderregeln der Alchemie}
\end{center}

Da die Wirkung durchaus von der Menge der alchemistischen Substanz abhängen kann, gibt es folgende Regeln zu beachten.

\subsubsection{Gifte}

Gifte werden mit Waffen beigebracht und fügen einen Zustand zu (siehe Kampfregeln).

\paragraph{Waffen}

Waffen haben ein Reservoir an Gift und fügen pro verletzendem Treffer Giftpunkte zu die dann das Reservoir reduzieren.
Diese Reduzierung findet auch statt wenn nur Rüstungs oder Schildtreffer gelanded wurden.

\begin{itemize}
\item Dolche u.ä.: Die Klinge kann bis 6 Punkte aufnehmen und fügt 2 Punkte beim verletzenden Treffer hinzu.
\item größere Schwerter: Waffe nimmt Punkte in Höhe ihre Würfels auf (Kurzschwert W6 -> 6 Punkte) und fügt beim Treffer 1 Punkt zu
\item Äxte und ähnliche scharfe Waffen haben nur ein halb so großes Reservoir wie Schwerter und fügen auch nur ein Punkt zu.
\item Pfeile und Bolzen fügen und halten 2 Punkte Gift.
\item Meisterhafte und oder Spezialwaffen können diese Werte verbessern.
\end{itemize}

\paragraph{Herstellung}

\begin{itemize}
\item Bei der Giftherstellung (negative KONFAR Effekte) werden 3 * Effektstärke plus Willenskraft an Giftdosen hergestellt.
\end{itemize}

\subsubsection{Detailbeschreibung}

Die Detailbeschreibung einer alchemistischen Substanz kann durch Zugabe nicht alchemistischer Stoffe
geschehen. So kann z.B ein SYMA FERA das Tier darstellen dessen Fell der Substanz beigefügt wurde.
Kontrolltränke benötigen solche Detailbeschreibung oder das Ziel bekommt einen starken Bonus auf die vergleichende
Willenskraftprobe. (ein AUM FERA Trank könnte einem Dieb beträchtlich Helfen wenn er
etwas Fell von dem Wachhund bekommt den er unter Verwendung des Tranks kontrollieren möchte).

\newpage

\begin{center}
\subsection{Alchemieblatt}
\end{center}
Phiolen (2 Splitter):


\begin{small}
\begin{tabular}{|m{2cm}|m{2cm}|m{6mm}|m{2cm}|m{3cm}|m{6mm}|m{6mm}|m{6mm}|m{6mm}|m{6mm}|m{6mm}|}
\hline
\textbf{Rezept}&\textbf{Zutaten}&\textbf{MW}&\textbf{krit. MW.}&\textbf{mag. Effekt}&\textbf{ES}&\textbf{W0}&\textbf{W1}&\textbf{W2}&\textbf{W3}&\textbf{W4}\\
\hline
\hline
1.&&&&&&&&&&\\
\hline
2.&&&&&&&&&&\\
\hline
3.&&&&&&&&&&\\
\hline
4.&&&&&&&&&&\\
\hline
5.&&&&&&&&&&\\
\hline
6.&&&&&&&&&&\\
\hline
7.&&&&&&&&&&\\
\hline
8.&&&&&&&&&&\\
\hline
9.&&&&&&&&&&\\
\hline
10.&&&&&&&&&&\\
\hline
11.&&&&&&&&&&\\
\hline
12.&&&&&&&&&&\\
\hline
13.&&&&&&&&&&\\
\hline
14.&&&&&&&&&&\\
\hline
15.&&&&&&&&&&\\
\hline
16.&&&&&&&&&&\\
\hline
17.&&&&&&&&&&\\
\hline
18.&&&&&&&&&&\\
\hline
19.&&&&&&&&&&\\
\hline
20.&&&&&&&&&&\\
\hline
21.&&&&&&&&&&\\
\hline
22.&&&&&&&&&&\\
\hline
23.&&&&&&&&&&\\
\hline
24.&&&&&&&&&&\\
\hline
25.&&&&&&&&&&\\
\hline
26.&&&&&&&&&&\\
\hline
27.&&&&&&&&&&\\
\hline
28.&&&&&&&&&&\\
\hline
29.&&&&&&&&&&\\
\hline
30.&&&&&&&&&&\\
\hline
31.&&&&&&&&&&\\
\hline
32.&&&&&&&&&&\\
\hline
33.&&&&&&&&&&\\
\hline
34.&&&&&&&&&&\\
\hline
35.&&&&&&&&&&\\
\hline
36.&&&&&&&&&&\\
\hline
37.&&&&&&&&&&\\
\hline
38.&&&&&&&&&&\\
\hline
39.&&&&&&&&&&\\
\hline
40.&&&&&&&&&&\\
\hline
41.&&&&&&&&&&\\
\hline
42.&&&&&&&&&&\\
\hline
43.&&&&&&&&&&\\
\hline
44.&&&&&&&&&&\\
\hline
45.&&&&&&&&&&\\
\hline
46.&&&&&&&&&&\\
\hline
47.&&&&&&&&&&\\
\hline
48.&&&&&&&&&&\\
\hline
49.&&&&&&&&&&\\
\hline
50.&&&&&&&&&&\\
\hline
\end{tabular}
\end{small}

\newpage


\begin{small}
\begin{tabular}{|m{3cm}|m{3cm}|m{4cm}|m{6cm}|}
\hline
\textbf{Zutat}&\textbf{Schwierigkeit}&\textbf{Preisvorschlag}&\textbf{Effekte oder sonstige Notizen}\\
\hline
\hline
BASIS&&&\\
\hline
Harz&2&2&\\
\hline
Königswasser&4&6&\\
\hline
Lauge&3&2&\\
\hline
Menschenblut&6&10&\\
\hline
Schlamm&3&3&\\
\hline
Spiritus&3&3&\\
\hline
Säure&3&3&\\
\hline
Tierblut&4&3&\\
\hline
Wachs&3&4&\\
\hline
Wasser&2&1&\\
\hline
Wein&3&2&\\
\hline
Öl&3&5&\\
\hline
ADDITIVE&&&\\
\hline
Amethyst&4&6&\\
\hline
Asche&3&1&\\
\hline
Bittergrün&3&2&\\
\hline
Blei&3&3&\\
\hline
Blätterpilz&3&2&\\
\hline
Brennnessel&3&2&\\
\hline
Diamant&7&12&\\
\hline
Eisen&3&2&\\
\hline
Elektrum&6&20&\\
\hline
Eulenkappe&4&3&\\
\hline
Fliegenpilz&3&3&\\
\hline
Gold&3&10&\\
\hline
Granit&3&2&\\
\hline
Honiggras&5&4&\\
\hline
Kalk&3&3&\\
\hline
Kohle&3&3&\\
\hline
Kupfer&3&2&\\
\hline
Löwenzahn&2&1&\\
\hline
Marmor&3&4&\\
\hline
Minze&2&2&\\
\hline
Mondkraut&4&2&\\
\hline
Nachtschatten&3&2&\\
\hline
Obsidian&6&4&\\
\hline
Phosphor&3&3&\\
\hline
Platin&4&15&\\
\hline
Pyrit&4&5&\\
\hline
Quecksilber&3&3&\\
\hline
Rinde&3&1&\\
\hline
Risspilz&5&7&\\
\hline
Rubin&5&10&\\
\hline
Salz&3&3&\\
\hline
Saphir&5&10&\\
\hline
Schwarzwurz&3&2&\\
\hline
Schwefel&3&2&\\
\hline
Silber&3&4&\\
\hline
Silberwurz&3&2&\\
\hline
Smaragd&5&10&\\
\hline
Sonnenstein&5&4&\\
\hline
Steinpilz&3&2&\\
\hline
Tintling&3&2&\\
\hline
Topaz&5&10&\\
\hline
Trughut&4&2&\\
\hline
Wegkraut&6&2&\\
\hline
Wolfsblatt&3&2&\\
\hline
Zinn&3&3&\\
\hline
\end{tabular}
\end{small}

\newpage
\begin{center}
\section{Charakterblatt}
\end{center}
\begin{tabular}{m{6cm} m{6cm} m{6cm}}
Name:&Herkunft:&Beruf:\\
Beschreibung:&&\\
&&\\
&&\\
&&\\
\end{tabular}
\hline
\begin{center}
\subsection{Attribute}
\end{center}
\begin{tabular}{m{6cm} m{6cm} m{6cm}}
Stärke&Wahrnehmung&Persönlichkeit\\
Geschicklichkeit&Intelligenz&Glück\\
Konstitution&Willenskraft&\\
\end{tabular}
\newline\hline
\begin{center}
\subsection{Kampfdaten}
\end{center}
\begin{tabular}{m{10cm} m{10cm}}
Verteidigung (14+Ges+Boni-Behinderung):&Waffe:\\
Geschwindigkeit (3 + (Stärke+Geschick)/2):&Waffe:\\
Rüstungsschutz:&Waffe:\\
&\\
\end{tabular}
\begin{tabular}{ m{9cm} m{9cm} }
\textbf{Körperlicher Monitor (5+Konstitution)}&\textbf{Geistiger Monitor (5+Willenskraft)}\\
\begin{tabular}{| c | c | c | c | c | c | c | c | c | c | c | c | c | c |}
\hline
0&&&&&&&&&&&&&\\
\hline
-1&&&&&&&&&&&&&\\
\hline
-2&&&&&&&&&&&&&\\
\hline
-3&&&&&&&&&&&&&\\
\hline
-4&&&&&&&&&&&&&\\
\hline
\end{tabular}
&\begin{tabular}{| c | c | c | c | c | c | c | c | c | c | c | c | c | c |}
\hline
0&&&&&&&&&&&&&\\
\hline
-1&&&&&&&&&&&&&\\
\hline
-2&&&&&&&&&&&&&\\
\hline
-3&&&&&&&&&&&&&\\
\hline
-4&&&&&&&&&&&&&\\
\hline
\end{tabular}
\\
\end{tabular}
\newline

Zustände:
\newline\hline
\begin{center}
\subsection{Fertigkeiten}
\end{center}
\begin{tabular}{ m{4cm}  m{4cm}  m{4cm}  m{4cm} }
\textbf{Akademisch}&Werfen:&\textbf{Magie}&Einschüchtern:\\
Geschichte:&\_\_\_\_\_\_&Alchemie:&Umgangsformen:\\
Magiekunde:&\_\_\_\_\_\_&Kristallisieren:&Verhandeln:\\
Magiekunde:&&Runenmagie:&\_\_\_\_\_\_\\
Navigation:&\textbf{Handwerk}&Verzaubern:&\_\_\_\_\_\_\\
\_\_\_\_\_\_&Beruf:&\_\_\_\_\_\_&\\
\_\_\_\_\_\_&Heilkunde:&\_\_\_\_\_\_&\textbf{Wildnis}\\
&\_\_\_\_\_\_&&Jagdkunst:\\
\textbf{Diebesfertigkeiten}&\_\_\_\_\_\_&\textbf{Nahkampf}&Nebelkreaturen:\\
Feinmechanik:&&Handgemenge:&Nebelkunde:\\
Taschendiebstahl:&\textbf{Körperlich}&Hiebwaffen:&TierFühren:\\
Verkleiden:&Aufmerksamkeit:&Spezialwaffen:&Wildniskunde:\\
\_\_\_\_\_\_&Gleiterflug:&Stangenwaffen:&\_\_\_\_\_\_\\
\_\_\_\_\_\_&Heimlichkeit:&\_\_\_\_\_\_&\_\_\_\_\_\_\\
&Sportlichkeit:&\_\_\_\_\_\_&\\
\textbf{Fernkampf}&\_\_\_\_\_\_&&\\
Armbrust:&\_\_\_\_\_\_&\textbf{Sozial}&\\
Bögen:&&Anführen:&\\
\end{tabular}

\newline\hline
\begin{center}
\subsection{Sonstiges}
\end{center}
\begin{tabular}{ m{10cm} m{10cm}}
Erfahrungspunkte / Gesamt:&\textbf{Ausrüstung:}\\
Anzahl gekaufter Attributspunkte:&\\
Anzahl erlernter Wege / Meisterwege:&\\
\textbf{Weg / Stufe}&\\
\end{tabular}
\end{document}
nt}
